\chapter{\LaTeX}
\label{latexchapter}
\LaTeX{}, or more properly ``\LaTeXe{}'', is a very useful document processing program. It is very widely used, widely available, stable and free. Famously, \TeX, upon which \LaTeX{} is built, was originally developed by the eminent American mathematician Donald Knuth because he was tired of ugly mathematics books \cite{shustek2008interview}. Although it has a learning curve (made much less forbidding by online tools and resources -- see below), it allows the writer to concentrate more fully on the content, and takes care of most everything else.

While it can be used as a word processor, it is a \emph{typesetting} system, and Knuth's idea was that it could be used to produce beautiful looking books:
\begin{quote}
\emph{\LaTeX{} is a macro package which enables authors to typeset and print their work at the highest typographical quality, using a predefined, professional layout.}\footnote{This is from \cite{oetiker2001not}. Did we mention that you should minimise your use of footnotes?}
\end{quote}
\LaTeX{} has great facilities for setting out equations and a powerful and very widely supported bibliographic system called BibTeX, which takes the pain out of referencing.

Three useful online resources make \LaTeX~much better:
\begin{enumerate}[(1)]
\item An excellent online \LaTeX{} environment called ``Overleaf'' is available at \url{http://www.overleaf.com} and runs in a modern web browser. It's got this template available -- search for a TCD template. Overleaf can work in conjunction with Dropbox, Google Drive and, in beta, GitHub.
\item Google Scholar, at \url{http://scholar.google.com}, provides BibTeX entries for most of the academic references it finds.
\item An indispensable and very fine introduction to using \LaTeX{} called \emph{``The not so short introduction to LATEX 2$\varepsilon$''} by \cite{oetiker2001not} is online at \url{https://doi.org/10.3929/ethz-a-004398225}. Browse it before you use \LaTeX~for the first time and  read it carefully when you get down to business.
\end{enumerate}
Other tools worth mentioning include:
\begin{itemize}
\item \texttt{Draw.io} -- an online drawing package that can output PDFs to Google Drive -- see \url{https://www.draw.io}.
\end{itemize}