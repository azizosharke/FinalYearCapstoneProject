\documentclass[a4paper,oneside,12pt]{book}

%----------------------------------------------------------------------------------------
%	README!
%   Welcome. It's worth having a read through this file
%   to set up the broad parameters, such as the name of
%   the degree, the school/department, the type of work
%   (dissertation/Final Year Project/report, etc. as well
%   as your own details.
%----------------------------------------------------------------------------------------

%----------------------------------------------------------------------------------------
%	COVER PAGE
%   The cover page is laid out in title/title.tex. You can choose a colour
%   or black and white logo
%----------------------------------------------------------------------------------------

%----------------------------------------------------------------------------------------
%	THESIS INFORMATION
%   Put title, author name, supervisor name, degree, type of work, school, department in here
%   It will be used for the title page and for the embedded PDF information
%----------------------------------------------------------------------------------------

\newcommand{\thesistitle}{DNS Abuse Transparency} % Your thesis title, this is used in the title and abstract
\newcommand{\degree}{Computer Science and Business} % Replace with your degree name, this is used in the title page and abstract
\newcommand{\typeofthesis}{dissertation} % dissertation, Final Year Project, report, etc.
\newcommand{\authorname}{Abdelaziz Abushark} % Your name, this is used in the title page and PDF stuff
%% Do not put your Student ID in the document, as TCD will not publish
%% documents that contain both your name and your Student ID.
\newcommand{\supervisor}{ Dr. Stephen Farrell} % replace with the name of your supervisor
%\newcommand{\cosupervisor}{Dr Alex Lee} % replace with the name of your co-supervisor if you have one
\newcommand{\keywords}{this, that, more} % Replace with keywords for your thesis
\newcommand{\school}{\href{https://www.tcd.ie/scss/}{School of Computer Science and Statistics}} % Your school's name and URL, this is used in the title page
%Edited by HS for engineering

%% Comment out the next line if you don't want a department to appear
%\newcommand{\department}{\href{http://researchgroup.university.com}{Department Name}} % Your research group's name and URL, this is used in the title page


%% Language and font encodings
\usepackage[T1]{fontenc} 
\usepackage[utf8]{inputenc}
\usepackage[english]{babel} 
%% Bibliographical stuff
\usepackage[]{cite}
%% Document size
% include showframe as an option if you want to see the boxes
\usepackage[a4paper,top=2.54cm,bottom=2.54cm,left=2.54cm,right=2.54cm,headheight=16pt]{geometry}
\setlength{\marginparwidth}{2cm}
%% Useful packages
\usepackage{amsmath}
\usepackage[autostyle=true]{csquotes} % Required to generate language-dependent quotes in the bibliography
\usepackage[pdftex]{graphicx}
\usepackage[colorinlistoftodos]{todonotes}
\usepackage[colorlinks=true, allcolors=black]{hyperref}
\usepackage{hyperxmp}
\usepackage{caption} % if no caption, no colon
\usepackage{sfmath} %use sans-serif in the maths sections too
\usepackage[parfill]{parskip}    % Begin paragraphs with an empty line rather than an indent
\usepackage{setspace} % to permit one-and-a-half or double spacing
\usepackage{enumerate} % fancy enumerations like (i) (ii) or (a) (b) and suchlike
\usepackage{booktabs} % To thicken table lines
\usepackage{fancyhdr}
\usepackage{xcolor} % to get TCD colour on headings
\usepackage{colortbl}
\numberwithin{equation}{chapter} %HS edit for (chapter.equation)
\pagestyle{plain} % Embrace simplicity!
\usepackage{caption}
\usepackage{pdflscape}
\usepackage{longtable}
\usepackage[none]{hyphenat}
\usepackage{booktabs}
\usepackage{float}
\usepackage{subcaption}




\definecolor{tcd_blue}{RGB}{5, 105, 185}
%% use [H] to make the pics do not float around. 
%% It's personal taste but...
%% Uncomment the following block if you want your name and ID at the top of
%% (almost) every page.

%\pagestyle{fancy}
%\fancyhf{} % sets both header and footer to nothing
%\renewcommand{\headrulewidth}{0pt}
%\cfoot{\thepage}
%\ifdefined\authorid
%\chead{\it \authorname\ (\authorid)}
%\else
%\chead{\it \authorname}
%\fi
%% End of block

%% It is good practise to make your font sans-serif to improve the accessibility of your document.  Comment out the following line to disable it (but you really should not)
\renewcommand{\familydefault}{\sfdefault} %use the sans-serif font as default

%% If you insist on not using sans-serif (please don't), consider using Palatino instead of the LaTeX standard
%\usepackage{mathpazo} % Use the Palatino font by default if you prefer it to Computer Modern


%% Format Chapter headings appropriately
\usepackage{titlesec}
\titleformat{\chapter}[hang]{\normalfont\huge\bfseries\color{tcd_blue}}{\thechapter}{1cm}{}{}

\title{\thesistitle}
\author{\authorname}


\hypersetup{
   pdftitle=\thesistitle, % Set the PDF's title to your title
   pdfauthor=\authorname, % Set the PDF's author to your name
   pdfkeywords=\keywords, % Set the PDF's keywords to your keywords
   pdfsubject=\degree, % Set the PDF's keywords to your keywords
   pdfinfo={
     pdfsupervisor=\supervisor, % Set the PDF's supervisor to your supervisor
     %pdfcosupervisor=\cosupervisor, % Set the PDF's cosupervisor to your cosupervisor if using
   }
}

\usepackage{natbib} %harvard style 
\frontmatter
\begin{document}
\input{title/title.tex}
\pagenumbering{roman}
\section*{\Huge\textcolor{tcd_blue}{Declaration}}
\vspace{1cm}
I hereby declare that this \typeofthesis\ is entirely my own work and that it has not been submitted as an exercise for a degree at this or any other university.

\vspace{1cm}
I have read and I understand the plagiarism provisions in the General Regulations of the University Calendar for the current year, found at \url{http://www.tcd.ie/calendar}.
\vspace{1cm}

I have completed the Online Tutorial on avoiding plagiarism `Ready Steady Write', located at \url{http://tcd-ie.libguides.com/plagiarism/ready-steady-write}.
\vspace{1cm}

I consent / do not consent to the examiner retaining a copy of the thesis beyond the examining period, should they so wish (EU GDPR May 2018).
\vspace{1cm}

I agree that this thesis will not be publicly available, but will be available to TCD staff and students in the University’s open access institutional repository on the Trinity domain only, subject to Irish Copyright Legislation and Trinity College Library conditions of use and acknowledgement.  \textbf{Please consult with your supervisor on this last item before agreeing, and delete if you do not consent}
\vspace{3cm}

Signed: \underline{Abdelaziz Abushark} \hfill Date: \underline{14/04/2024}


\chapter*{Abstract}

This research project explores the subject of DNS (Domain Name System) abuse, which is widespread and jeopardises the reliability and security of the Internet. The integrity of DNS operations has always been critical due to the growing reliance on the Internet for both personal and professional activity. To promote a safer online environment, this project investigates whether and if so, additional transparency related to DNS abuse mitigations might help improve the overall DNS ecosystem. Every type of abuse puts users at risk by making identity theft, money loss, data breaches, and system intrusions easier, as well as by undermining confidence in online services. The reseach project emphasises how urgent it is to address these problems because new technologies like IoT and AI have the potential to make them worse.

The methodology used in this study included a thorough examination of the DNS ecosystem, vulnerability identification, and an assessment of mitigation initiatives that are currently being implemented by important parties, such as registries and registrars. The survey of DNS infrastructure providers and stakeholders was part of the selective survey to determine the existing level of transparency in the mitigation of DNS abuse. This involved assessing the usefulness of transparency reports and how well they work to stop DNS abuse.Key findings point to a serious weakness in DNS abuse mitigation initiatives' openness. Although several organisations have taken positive steps towards transparency, there is still a lack of standardisation and fragmentation in the industry as a whole. The report makes a number of suggestions to improve openness and transparency, such as creating uniform reporting guidelines, encouraging greater cooperation between DNS stakeholders, and implementing best practices to deal with DNS abuse in an open and transparent manner.

This research project adds to the current conversation on DNS abuse by providing a practical reform plan and a detailed grasp of its complexities. It establishes the foundation for more successful mitigation of DNS abuse by promoting transparency, which hopefully will result in more secure and reliable Internet.




\newpage
\onehalfspacing\raggedright %\raggedright turns off justification and hyphenation

\section*{\Huge\textcolor{tcd_blue}{Acknowledgements}}

In the name of God, the most Gracious, the most Merciful.

First, I would like to thank God. Everything I do is only done with his permission. I sincerely thank everyone who helped me along the way with this thesis. First, I express my sincere gratitude to Dr. Stephen Farrell, my supervisor, whose knowledge, compassion, and tolerance greatly enhanced my graduate experience. From the beginning to the end of my inquiry, your advice was very helpful.

I also would like to express my gratitude to the study participants who enthusiastically engaged with my work and offered valuable insights into the dynamics of DNS abuse.

I must express my sincere gratitude to my family for their unwavering support and unceasing encouragement during my years of education, as well as during the process of conducting research and composing this project. Without them, this achievement would not have been feasible. I am grateful to my parents for their guidance and experience.

I want to take this time to show my sincere thanks to friends and college friends who have made good company, understanding the times of stress and relief during our college years. Their presence and support have been a great joy and motivation.

Finally, I would like to thank Trinity College Dublin and my lecturers for giving me opportunities over the past four years.
I am appreciative of what they have offered.




\tableofcontents
\listoffigures
\listoftables
\newpage



\mainmatter
% maintaining separate .tex files for each chapter is good practice
\chapter{Introduction}


\section{Brief Context for the Problem}

The Domain Name System (DNS), as seen in Figure \ref{fig:dnsIntro}, translates domain names into IP addresses. This will definitely affect the daily digital interaction of each user, along with the smooth running of the Internet. Unfortunately, this one is not resistant to abuse. Meanwhile, malicious actors use DNS domains for a variety of abusive and sometimes illegal activities, such as sending malware, phishing websites, and controlling botnets \cite{so2022}. Therefore, such activity undermines the reliability and security of the Internet, posing very serious risks to cybersecurity and user trust \cite{bayer2022}. Addressing this issue requires a robust response from DNS infrastructure providers, including registrars and registries, who play a role in the management of abuse complaints. Registries refer to organisations that function to manage the top-level domains (TLDs) of the Internet, such as ".com" and ".net". On their part, registrars play a role as some form of intermediaries in the sale of domain names to the members of the public. Entities of this kind have the power to disable or deny the registration of DNS names that have been found to be abusive. But it will also consider proactive measures, such as turning down registrations that may facilitate "typosquatting", and potentially regulating permissible domain names to censor registration or renewal based on content. This would enhance the efficiency of such interventions since they shall have been transparent on the measures and the justification thereof. Although the trend is in a manner that issuing transparency reports would be able to shed light on the practices, this happens on rare occasions.

\begin{figure}[H]
    \centering
    \includegraphics[width=0.4\linewidth]{introduction/dnsWork.jpg}
    \caption{How DNS works. Adapted from \cite{blanche2018understandingDNS}. }
    \label{fig:dnsIntro}
\end{figure}

\section{Motivation}

With the growing opportunities for DNS abuse for malicious and sometimes even illegal activities such as confusable domains and phishing, the figure \ref{fig:dnsintro2} of honesty and protection is at issue. This has been particularly highlighted by the harshness and regularity with which this occurs in recent studies: such as the "Study on Domain Name System (DNS) Abuse: Technical Report” by Bayer et al., which represents the need for more surveillance and moderation activities \cite{bayer2022}. Many instances of DNS abuse have not only jeopardised user protection but also shattered public trust in the digital market. As citizens become more aware of these hazards, their confidence declines, and there is a pressing need to mitigate the challenges to restore trust and a secure online environment. Hesselman et al. \cite{hesselman2020}, proposed the development of a "responsible Internet" by improving community-level transparency, as well as the responsibility to increase confidence and control. Mathew and Cheshire's analysis and self-reporting Trust and Community Practice in the Context of Network Security explore the importance of trusting interactions and communities online and depict how the danger of DNS abuse undermines it \cite{mathew2016}.

\begin{figure}[H]
    \centering
    \includegraphics[width=0.5\linewidth]{introduction/maliciousActivity.png}
    \caption{ increase in DNS abuse incidents over time. Adapted from \cite{Rich2023Cyberpsychology}.}
    \label{fig:dnsintro2}
\end{figure}

Registries and registrars are leading the way in this issue, especially DNS infrastructure providers such as registrars and registries. However, their policies are probably clear and transparent to themselves, just not to outsiders.  This clear approach to handling DNS abuse allegations and their accompanying actions worsens the ongoing lack of confidence. Identified in this case are the credibility and a necessary need to protect the Internet \cite{cerf2022}. It also covers the moral and legal implications, in addition to technical aspects of DNS abuse and how one can mitigate it. This is the void to which the project is motivated to fill by exploring ways to increase the transparency of mitigation of DNS abuse. To understand current efforts, the research also sought to find the difficulties in the way of more transparent practices through an evaluation of the current landscape on transparency reports and practices among DNS infrastructure providers. The ultimate goal is to contribute to a system that can facilitate, promote, and enable better and more efficient approachable transparency in the mitigation of DNS abuse.


\section{Research Question/Project \& Personal objective} 
\subsection{Research Question}

Primary research question: "How does the work of registries, registrars, and other DNS infrastructure participants, as it appears in transparency reports, help mitigate DNS abuse, and what can be learnt from it, in terms of best practices for transparency when it comes to handling complaints related to DNS abuse?" This question seeks to uncover the mechanisms, policies, and practices in place to mitigate DNS abuse and to what extent these efforts are transparent to the public and stakeholders.

\subsection{Project Objectives}

Assess handling of abuse complaints

\begin{itemize}
  \item Examine the protocols and measures that DNS infrastructure providers use to respond to abuse complaints.
  \item Report the most common types of DNS abuse complaint received and the mechanisms applied in most cases.
\end{itemize}

Assess Transparency Levels:

\begin{itemize}
  \item Evaluate the level of transparency available in the actions taken by carriers/infrastructure providers against DNS abuse.
  \item Identify what information is made public, how it is communicated, and the frequency of disclosure.
\end{itemize}

Evaluating Against Best Practices:

\begin{itemize}
  \item Measure the results against best practices in the field to establish areas of success and failure. 
  \item Identify examples of good and poor transparency or abuse mitigation measures.
\end{itemize}

Develop recommendations :

\begin{itemize}
  \item Suggest actionable recommendations to DNS infrastructure providers to improve their abuse handling and transparency.
  \item Propose that some policy changes or initiatives be implemented to normalise and enhance such practices across the industry.
  \item Contribute to work in the future on how best practices for transparency may be developed.
\end{itemize}

Contribute to stakeholder understanding: 

\begin{itemize}
  \item Give stakeholders, including consumers, policymakers, and other providers, an understanding of the handling and transparency of DNS abuse.
  \item Develop a roadmap for further research and discussion on improving DNS trust and security.
\end{itemize}

\section{Scope}	
The Scope of this project will consider the transparency measures that registrars and registries take to mitigate DNS abuse. It will look at the collection and characterisation of transparency reports that can be obtained from registries and registrars, among others that are given the same responsibilities to mitigate abuse. Furthermore, this work will review the transparency reports developed in the current year, thus forming future work on the ways through which practices for transparency could be developed. The project discusses with different actors, as summarised in figure \ref{fig:dnsintrointro}, in the DNS ecosystem to obtain their opinions and insights on what they are currently practising and the challenges they face. This will involve establishing criteria that can be used to measure the way in which the same transparency can affect the perception of the Internet user, which then relates to trust and safety. The new system will not include the creation of new transparency tools or systems, but will be based on the review of current procedures and the recommendation of changes for the better. The key objective of the study is to learn more about transparency and its impacts.  

\begin{figure}[H]
    \centering
    \includegraphics[width=0.7\linewidth]{introduction/diagram (8).png}
    \caption{ DNS ecosystem.}
    \label{fig:dnsintrointro}
\end{figure}

\section{ Outline of the Project Work} 

The goal of this project, "DNS Abuse Transparency", is to better understand and increase the transparency of the efforts of the registrars and registries to mitigate DNS abuse. The research will first examine the different aspects of DNS abuse, such as popular forms like phishing, confusable domains, etc. and their broader consequences. 

The questionnaire will explore the scope and effectiveness of current practices implemented on the aspect of transparency in mitigating abuse associated with the DNS. At the same time, the study will also unveil current transparency reports that reflect the landscape, frequency, scope, and accessibility of the reports to users. Critical evaluation of the handling of DNS abuse reports forms the core of the project.

Critical evaluation of the handling of DNS abuse reports forms the core of the project. This will involve a review of proactive security controls that may be in place, procedures for mitigation, and avoidance of abusive domain registrations. Thereafter, these will be assessed in terms of how transparency influences not only user trust, but also provider reputation, and overall the effectiveness of techniques applied in mitigating abuse. The best practices for mitigating DNS abuse.

The project will discover and clarify best practices for transparency in the mitigation of DNS abuse, based on the data and insights obtained. The careful balance between security, privacy, and transparency will be taken into account by these best practices. It is under this background that the research will, therefore, with these findings in mind, develop a set of practical recommendations for the DNS infrastructure providers seeking to increase transparency for better security and, therefore, trust in the digital ecosystem.

A comprehensive timeline will guide the progress, guaranteeing an organised study of the subject. The project, upon completion, would have contributed to a collection of recommendations and considerations for further study and policy creation in this area of Internet governance. This, in turn, would give a further comprehensive understanding of where the current state of DNS abuse transparency lies.

\section{Outline of the report}

This report offers a comprehensive account of the steps performed, decisions made, and research carried out during the project's development. The format of the report is as follows:

\textbf{Chapter 2 - Background }

This chapter discusses the foundation of DNS, its importance, its weaknesses, and several types of abuse. Explaining the methods formulated in combating DNS abuse gives a specific look at the work done by ICANN and the DNS Abuse Institute.
\vspace{25px}

\textbf{Chapter 3 -  State of the art }

This chapter critically looks at some of the existing strategies towards mitigation of DNS abuse and their effectiveness. This chapter will give an account of the complex relationship that international governments have with DNS and outline efforts to openness made by companies such as Google and Cloudflare. In addition to stressing the difficulties in striking a balance between user privacy and compliance requirements, the chapter emphasises the importance of DNS in internet governance. Investigate how various tactics are used and their effects on the larger online ecosystem through critical analysis.

\textbf{Chapter 4 -  Research methodology }

This chapter describes techniques to investigate DNS abuse and transparency from the infrastructure provider. The author goes on how the questionnaires were made, how the responses from stakeholders were analysed, and what kinds of DNS abuse were found. This chapter describes the methodology used to collect and examine data to understand DNS abuse reporting procedures and transparency policies.

\textbf{Chapter 5 -  Implementation }

This chapter is the practical part of the project, where all other findings are implemented, including integrating the system, the back-end, and front-end implementations, and implementation technologies that will form the system. The area that this part covers is how the DNS data are visualised with the term of abuse and how the system implementation challenges are addressed. The next section also includes the testing and validation that are performed.

\textbf{Chapter 6 -  Evaluation \& Discussion }

This chapter evaluates the way the project meets the mitigation of DNS abuse and the improvement of transparency, as well as the chapter was also used to evaluate the effectiveness of transparency in mitigating DNS abuse while evaluating security issues. Furthermore, the chapter will present the limitations of the study and the extent to which the project achieved its objectives.

\textbf{Chapter 7 -  Conclusion }

The chapter provided a summary of the project results and recommendations for improving the transparency of DNS abuse mitigation. It is important to note that no matter how difficult the situation, it is important to continue to try to improve the overall security of the DNS. As a result, there are opportunities for further research in the area discussed. More importantly, the chapter underlines the essence of collaboration and transparency in tackling DNS abuse.

\chapter{Background}
\label{Chapt2}


This chapter will explore the fundamental information relevant to this project, with an emphasis on the world of DNS abuse and transparency. It will include a detailed investigation of the domain name system (DNS), its function in the online community, and the variety of abuses it faces , the history of widely used policies and organisations aimed at mitigating DNS abuse, including a thorough examination of the DNS Abuse Institute and its achievements. A 'competition landscape' providing an examination of current market choices, from automated solutions to human tactics, will be provided as we navigate through the current methodology and technology deployed to mitigate DNS abuse. The reader will obtain a detailed understanding of the current situation of DNS abuse and the need for a more open, strong, and proactive strategy by analysing these various techniques and appreciating their strengths and weaknesses. This chapter emphasises the importance of the suggested solution in an era where digital authenticity is required, not only by providing information, but also by laying the groundwork for its presentation as a better and essential progression in the battle against DNS abuse.

\section{Understanding DNS \& Its Vulnerabilities}

The Domain Name System (DNS) is a significant part of the Internet infrastructure, serving as the key to converting computer-understandable IP addresses into human-friendly domain names. Although the DNS plays a vital role in maintaining ongoing online activities, privacy and security problems still arise. The ScienceDirect paper "Domain Name System Security and Privacy: A Contemporary Survey" provides a detailed analysis of these concerns that highlights the fundamental importance of DNS while illuminating the weaknesses that malicious actors may take advantage of \cite * {Sciencedirect2023dns}. There are a variety of security threats, ranging from DNS infrastructure targeting distributed denial-of-service (DDoS) assaults to cache poisoning and hijacking. Each of these attacks has the potential to do significant harm, including interruptions in service and the promotion of theft and spying. Due to the standard DNS design's lack of encryption, users' query data is vulnerable to abuse and eavesdropping, raising serious privacy problems. However, weaknesses do not mark the end of the story. In the same survey, new approaches are examined to improve DNS security and privacy. The use of DNSSEC (DNS Security Extensions), which authenticates DNS data and guarantees its integrity while repelling some types of attack, is an example of these advances in security measures. In addition, privacy-enhancing technologies are being used to encrypt DNS queries, preventing eavesdropping and manipulation, such as DNS over HTTPS (DoH) and DNS over TLS (DoT). The environment of DNS threats and defences is always changing in sync with the Internet. For systems to be robust and resilient, it is essential to understand these weaknesses and the continuous efforts being made to mitigate them. In this section, we provide an in-depth discussion of DNS vulnerability details, the effects of these safety concerns, and creative solutions that aim to bring in a new era of DNS security and privacy.

In a usual DNS lookup, three types of queries come into play to streamline the process and minimise the data journey. The first type is a recursive query, where the DNS client expects a direct answer or an error if the record cannot be found from the DNS server. Then there is an iterative query, which means if the server doesn't have the answer, it points the client to another server that might know, and the client keeps asking down the line until it gets an answer or hits a dead end. Lastly, a non-recursive query happens when the DNS server already knows the answer either because it is directly responsible for that piece of information or it has it saved from earlier inquiries. This method helps to reduce unnecessary internet traffic and reduce the load on the servers involved.

\section{Strategies \& Collaborations in Addressing DNS Abuse}

The DNS Abuse Institute, which will focus on DNS abuse to help increase safety and security through the domain name system, will be catered on these efforts to address DNS abuse with a comprehensive approach throughout the internet infrastructure. It helps the Internet community identify, report and mitigate DNS abuse in its mission to make the online environment more secure. Efforts by the institute, such as Compass Dashboards, provide vital data to registries and registrars that will enable proper decisions on combating DNS abuse. They show the commitment to transparency and education by issuing publications such as the "DNSAI 2022 Annual Report" or "DNSAI Bulletin 2023 04; Account Takeovers," which provide information on DNS abuse and how recommended mitigation practices \cite{dnsabuseinstitute2023}. Another such global strategy against DNS abuse has been contributed by the Internet Corporation for Assigned Names and Numbers (ICANN)\cite{icann2022dnsabuse} in collaboration with the entire DNS community, ICANN supports a synchronised method in the development of policies and standards on how to mitigate DNS abuse while ensuring the openness of the Internet. These participatory pillars hint at concerted efforts through policy development, technological developments, and stakeholder engagement as a central component in this collective approach to combating DNS abuse \cite{dnsai2022report}. 



\section{Different Forms of DNS Abuse}

DNS abuse takes many forms, each with its procedures and effects on users and the Internet as a whole. It is essential to understand these various pieces of evidence to create responses and regulations that work. This section will examine the comprehensive analysis of DNS abuse presented, describing the description, mechanism, and impact of each kind \cite{dotmagazine2022dnsabuse}.

\subsection{Phishing}
\begin{itemize}
    \item \textbf{Description:} Phishing is a technique aimed at deceiving individuals by creating website addresses that mimic those of companies, to trick users into revealing sensitive information such as login credentials, credit card numbers, or personal identification information \cite{webinarcare2023dnsstats}.
    \item \textbf{Mechanism:} This deception often occurs through emails or messaging services that direct users to websites similar to authentic ones \cite{jakobsson2006phishing}.
    \item \textbf{Impact:} Victims may suffer identity theft, financial fraud, and security compromise.
\end{itemize}

\subsection{Confusable Domains (Typosquatting)}
\begin{itemize}
    \item \textbf{Description:} Registering domain names that look visually similar to popular websites, taking advantage of typing errors or character similarities \cite{inta2023dnstypo}.
    \item \textbf{Mechanism:} Users may accidentally visit these websites when making a typo in a URL, which can expose them to malware or phishing attempts.
    \item \textbf{Impact:} Deception of users and potential harm to brand reputation \cite{edelman2008typosquatting}.
\end{itemize}

\subsection{Domain Hijacking}
\begin{itemize}
    \item \textbf{Description:} Unauthorised acquisition of domain names by exploiting security vulnerabilities in the domain registration system \cite{inta2023dnstypo}.
    \item \textbf{Mechanism:} Attackers may use tactics like social engineering, phishing, or exploiting security loopholes to gain control over a domain.
    \item \textbf{Impact:} Loss of control of the website, redirection to malicious sites, and potential data breaches.
\end{itemize}

\subsection{Botnets}
\begin{itemize}
    \item \textbf{Description:} Botnets involve controlling a group of computers infected with malware, used to carry out attacks or spread spam and malware \cite{citpyour}.
    \item \textbf{Mechanism:} Malware infects computers of unsuspecting users, incorporating them into a network under the attacker's control.
    \item \textbf{Impact:} Can result in large-scale DDoS attacks, mass spam campaigns, and widespread malware dissemination.
\end{itemize}

\subsection{Fast Flux Hosting}
\begin{itemize}
    \item \textbf{Description:} A technique used to conceal the location of websites associated with phishing and malware distribution \cite{lin2013genetic}.
    \item \textbf{Mechanism:} Involves a network of compromised hosts that regularly modify DNS records to avoid detection.
    \item \textbf{Impact:} Makes tracking and shutting down malicious sites difficult.
\end{itemize}

\subsection{Domain Generation Algorithms (DGA)}
\begin{itemize}
    \item \textbf{Description:} DGAs generate domain names that act as meeting points for botnets \cite{antonakakis2012throw}.
    \item \textbf{Mechanism:} Malicious software uses algorithms to generate a sequence of domain names for command-and-control servers.
    \item \textbf{Impact:} Adds complexity to efforts to disrupt botnet command and control channels.
\end{itemize}
\captionsetup{font= footnotesize}
\begin{figure}[H]
\centering
\includegraphics[width=\textwidth]{background/DNSabuseForms.png}
\caption{Different Forms of DNS Abuse.}
\label{fig:figureThree}
\end{figure}



\section{How DNS Abuse Harms Users}

DNS abuse has serious and detrimental effects for both users and organisations, going beyond basic technological disruptions. Identity theft is among the most direct and direct effects. Phishing attacks, a common type of DNS abuse, use realistic websites to trick visitors into revealing sensitive data. Such attacks can produce information that results in financial theft, unauthorised access to accounts, and long-term damage to a person's reputation and credit \cite{godaddy2023dnsabuse}.

\subsection{Identity Theft}
\begin{itemize}
    \item \textbf{Phishing:} Phishing attacks often use domain names that imitate legitimate websites, fooling users into providing sensitive information such as usernames, passwords, or financial details, leading to potential identity theft.
\end{itemize}

\subsection{Financial Loss}
\begin{itemize}
    \item \textbf{Deceptive Transactions:} Users may be tricked into making payments to deceptive websites or unknowingly disclose their credit card information, resulting in financial losses \cite{bohme2013economics}.
\end{itemize}

\subsection{Data Breach}
\begin{itemize}
    \item \textbf{Malware:} Malicious software spread through compromised DNS systems can allow unauthorised access to corporate data, leading to data breaches \cite{fowler2016data}.
\end{itemize}

\subsection{System Compromise}
\begin{itemize}
    \item \textbf{Malware Infection:} Systems infected with malware due to DNS abuse can be exploited for further attacks, including the creation of botnets or the distribution of ransomware, resulting in system compromise \cite{saxe2018malware}.
\end{itemize}
\captionsetup{font= footnotesize}
\begin{figure}[H]
\centering
\includegraphics[width=\textwidth]{background/DNSabuseHarm.png}
\caption{How DNS Abuse Harms Users.}
\label{fig:figureFour}
\end{figure}


\section{Future Dangers of DNS Abuse}

As technology develops, so do bad actor strategies and tools, creating a dynamic environment for DNS abuse that could present new risks in the future. The sophistication of attacks has increased, which is a major issue. Bad actors are always creating increasingly sophisticated methods to take advantage of DNS, such as creating more convincing phishing schemes and using advanced virus distribution networks \cite{icann2022dnsabusetrends}.

\subsection{Increased Sophistication}
\begin{itemize}
    \item \textbf{Evolving Techniques:} Bad actors are constantly developing more sophisticated techniques to exploit DNS, such as advanced phishing schemes and malware distribution \cite{wrightson2014advanced}.
\end{itemize}

\subsection{IoT Vulnerabilities}
\begin{itemize}
    \item \textbf{Expanding Vulnerabilities:} The widespread adoption of Internet of Things (IoT) devices, which often lack robust security measures, presents a growing target for DNS-based attacks \cite{mahmoud2015internet}.
\end{itemize}

\subsection{Infrastructure Attacks}
\begin{itemize}
    \item \textbf{DNS as a Prime Target:} Attacks on DNS infrastructure can disrupt internet services on a large scale, including DDoS attacks targeting DNS providers or exploiting weaknesses in DNS protocols \cite{dooley2017dns}.
\end{itemize}

\subsection{Deepfakes \& AI}
\begin{itemize}
    \item \textbf{AI-Enhanced Phishing:} The use of AI technologies, such as deepfakes, has made phishing attacks more convincing and deceptive, manipulating audio and video content to impersonate trusted entities \cite{schick2020deep}.
\end{itemize}

\subsection{Cloud Computing Vulnerabilities}
\begin{itemize}
    \item \textbf{Targeting Cloud Services:} As organisations increasingly rely on cloud-based services, bad actors are exploiting DNS vulnerabilities to attack these platforms, potentially leading to data breaches and service disruptions \cite{mather2009cloud}.
\end{itemize}

\subsection{Mobile Device Exploitation}
\begin{itemize}
    \item \textbf{Mobile DNS Attacks:} The rising usage of mobile devices has led bad actors to target smartphones and tablets through DNS-based attacks, which can lead to data theft and the spread of malware \cite{au2016mobile}.
\end{itemize}

\subsection{Cryptocurrency \& Blockchain Exploitation}
\begin{itemize}
    \item \textbf{Crypto-Related DNS Attacks:} Attackers could exploit DNS vulnerabilities to redirect users to fake cryptocurrency exchanges or blockchain platforms, leading to financial fraud and theft of digital assets \cite{bashir2019advanced}.
\end{itemize}

\subsection{Political and Information Warfare}
\begin{itemize}
    \item \textbf{DNS in Cyber Warfare:} The manipulation of domain name systems can be used to spread misinformation or disrupt services during significant political events, serving as a tool for political and information warfare \cite{chapple2021cyberwarfare}.
\end{itemize}

\subsection{Exploiting Emerging Technologies}
\begin{itemize}
    \item \textbf{Abuse in New Tech Domains:} As new technologies such as 5G, AI, and quantum computing advance, tactics involving DNS abuse are likely to evolve, potentially leading to more sophisticated attacks \cite{brunner2021cybersecurity}.
\end{itemize}

\subsection{Supply Chain Attacks}
\begin{itemize}
    \item \textbf{DNS in Supply Chain Compromise:} DNS manipulation can also be employed as part of supply chain attacks, targeting software updates or cloud-based services to compromise organisations \cite{boyson2014cyber}.
\end{itemize}



\captionsetup{font= footnotesize} 
\begin{figure}  [H]
    \centering
    \includegraphics[width=\textwidth]{background/Future Dangers of DNS Abuse.png}
    \caption{Future Dangers of DNS Abuse.}
    \label{fig:LOLOLOL}
\end{figure}

By understanding these future dangers and emerging trends, stakeholders can better prepare and adapt their strategies to anticipate and counteract the evolving nature of DNS abuse.


\section{Foundational Mitigation Strategies \& Best Practices }

To address the broad nature of threats, mitigating DNS abuse requires an integrated strategy that integrates multiple strategies and best practices. The establishment of reporting and monitoring procedures is one fundamental tactic. Automated systems have the ability to track domain name registration patterns that may indicate DNS abuse, and protocols to report questionable actions can help ensure prompt intervention \cite{icannndnssec}. To confirm security and ensure that systems have not been compromised, regular audits of DNS configurations and domain registrations are also necessary \cite{lucas2021tls} .

\begin{enumerate}
    \item \textbf{Monitoring \& Reporting}
    \begin{itemize}
        \item Implementation: Use automated systems to monitor domain name registration for patterns that may indicate DNS abuse \cite{icannndnssec}. Establish procedures for reporting activities to authorities or cybersecurity organisations \cite{lucas2021tls}.
    \end{itemize}
    \item \textbf{Security Awareness Training}
    \begin{itemize}
        \item Implementation: Develop training programmes for users and IT staff with a focus on recognising phishing attempts, practising browsing habits, and understanding DNS security.
    \end{itemize}
    \item \textbf{DNS Security Extensions (DNSSEC)}
    \begin{itemize}
        \item Implementation: Deploy DNSSEC to ensure the integrity of the DNS data. This involves signing DNS records to protect against modification and DNS spoofing.
    \end{itemize}
    \item \textbf{Multi-Factor Authentication (MFA)}
    \begin{itemize}
        \item Implementation: Enforce multifactor authentication (MFA) for domain registrars and interfaces used to manage DNS \cite{icannndnssec}. This adds a layer of security beyond passwords, helping to prevent unauthorised domain transfers or alterations \cite{moghaddam2014ecco}.
    \end{itemize}
    \item \textbf{Blacklisting \& Takedown Services}
    \begin{itemize}
        \item Implementation: Collaborate with cybersecurity firms to identify and blacklist domains engaged in malicious activities. Establish response teams dedicated to removing domains involved in DNS abuse.
    \end{itemize}
    \item \textbf{Collaboration}
    \begin{itemize}
        \item Implementation: Foster collaboration among Internet service providers (ISPs), domain registrars, governments, and cybersecurity organisations. Share intelligence and best practices to collectively improve defence against DNS abuse \cite{skopik2017collaborative}.
    \end{itemize}
    \item \textbf{Regular Audits}
    \begin{itemize}
        \item Implementation: Conduct security audits of domain registrations and DNS configurations to verify their security and ensure that they have not been compromised \cite{coronado2014auditing}.
    \end{itemize}
    \item \textbf{Machine Learning}
    \begin{itemize}
        \item Implementation: Using AI and machine learning algorithms to analyse patterns in DNS traffic and proactively predict instances of DNS abuse \cite{icannndnssec}. This proactive approach enables the identification of threats before they materialise \cite{tsukerman2019machine}.
    \end{itemize}
    \item \textbf{Geo-Blocking \& IP Filtering}
    \begin{itemize}
        \item Implementation: Deploy geo-blocking and IP filtering techniques to limit access to DNS services from regions that have a history of DNS abuse. This can reduce the risk that attackers will use these services to carry out malicious activities or distribute malware \cite{meeseedited}.
    \end{itemize}
    \item \textbf{Enhanced Domain Validation Procedures}
    \begin{itemize}
        \item Implementation: Enhance the domain registration process by implementing validation procedures. This may involve verifying the identity of individuals or organisations that register domains, especially domains that resemble brands or fall into sensitive categories. By taking these measures, we can strengthen security and mitigate the risks associated with fraudulent domain registrations.
    \end{itemize}
\end{enumerate}


\captionsetup{font= footnotesize}
\begin{figure} [H]
    \centering
   \includegraphics[width=1.1\textwidth]{background/diagram (7).png}
    \caption{Mitigation Strategie.}
    \label{sadasdasdada}
\end{figure}

Each of these strategies plays a role in creating a comprehensive defence against DNS abuse. By integrating these tactics, organisations can establish robust, proactive measures to detect, prevent, and mitigate the ever-evolving threats posed by DNS abuse.

\section{Summary \& Synthesis}

After exploring the different forms of DNS abuse , How DNS abuse harms user , Future Dangers of DNS abuse and Mitigation Strategies and Best Practices. I have designed a table that has DNS abuses and the best possible mitigation strategies to help them against them, taking into account the transparency story behind it , user harm and reasoning. 


{

\footnotesize

\begin{longtable}{|p{2.5cm}|p{2.5cm}|p{4cm}|p{3cm}|p{4cm}|} 

\hline
\cellcolor{gray!50}\textbf{DNS Abuse } & 
\cellcolor{gray!50}\textbf{User Harm} & 
\cellcolor{gray!50}\textbf{Mitigation Strategy} & 
\cellcolor{gray!50}\textbf{Reasoning} & 
\cellcolor{gray!50}\textbf{Transparency Aspect} \\ \hline
\endfirsthead

\multicolumn{5}{c}%
{
\hline \cellcolor{gray!50}\textbf{DNS Abuse} & 
\cellcolor{gray!50}\textbf{User Harm} & 
\cellcolor{gray!50}\textbf{Mitigation Strategy} & 
\cellcolor{gray!50}\textbf{Reasoning} & \cellcolor{gray!50}
\textbf{Transparency Aspect} \\ \hline
\endhead

\hline \multicolumn{5}{|r|}{{\cellcolor{gray!50} Continued on next page}} \\ \hline
\endfoot

\hline
\endlastfoot
Phishing & \mbox{Identity Theft}, Financial Loss &  \mbox{Security Awareness} \mbox{Training, Enhanced Domain} Validation Procedures & \mbox{Training helps users} \mbox{recognize phishing} \mbox{attempts. Validation} prevents registration of mimic domains. & \mbox{Increases awareness and} \mbox{scrutiny during domain} registration. \\ \hline

\mbox{Confusable} Domains \mbox{(Typosquatting)} & Unauthorised Account Access & \mbox{Enhanced Domain} \mbox{Validation Procedures}, Regular Audits & \mbox{Prevents Registration} of Similar Domains. \mbox{Audits ensure} \mbox{compliance.} & \mbox{transparent domain} \mbox{registration process.} \\ \hline

\mbox{Domain} \mbox{Hijacking} & \mbox{System} \mbox{Compromise}, \mbox{Data Breach} & \mbox{Multi-Factor Authentication} (MFA), Regular Audits & \mbox{MFA secures domain} management. \mbox{Audits verify security} measures. & \mbox{Accountability in domain} management. \\ \hline

Botnets & \mbox{Malware} \mbox{Distribution} & Collaboration,Machine Learning & \mbox{Intelligence Sharing} \mbox{identifies botnet} \mbox{activities. AI predicts} \mbox{the formation of} \mbox{botnets}. & \mbox{Shared responsibility and} proactive detection. \\ \hline

\mbox{Fast Flux} \mbox{Hosting} & \mbox{System Infections} & Blacklisting and Takedown Services, Geo-Blocking & \mbox{Rapid response to} \mbox{malicious domains.} Restrict access from risky regions. & Responsive and transparent threat management. \\ \hline

\mbox{Domain} \mbox{Generation} Algorithms (DGA) & \mbox{Malware} \mbox{Distribution} & \mbox{Machine Learning, DNS} \mbox{Security Extensions} (DNSSEC) & AI detects abnormal \mbox{patterns. DNSSEC} \mbox{prevents spoofing.} & Integrity and trust in DNS data. \\ \hline

\mbox{IoT} \mbox{Vulnerabilities} & \mbox{Unauthorised} \mbox{Access, Data} \mbox{Breach} & \mbox{Security Awareness} \mbox{Training, Collaboration} & \mbox{Educates on security} \mbox{practices.} \mbox{Collaboration on best} \mbox{practices.} & \mbox{Open exchange of} \mbox{knowledge and efforts.} \\ \hline

Infrastructure Attacks & \mbox{DDoS Attacks}, \mbox{System Downtime} & DNSSEC, Collaboration & Protects DNS data integrity. Sharing of threat intelligence. & \mbox{Collective action}  \mbox{strengthens the DNS} infrastructure.  \\ \hline

Deepfakes and AI & \mbox{Identity Theft}, \mbox{Misinformation} & \mbox{Security Awareness} \mbox{Training, Monitoring} & \mbox{Recognising Phishing.} \mbox{Monitor} \mbox{AI threats.} & \mbox{Vigilance and prompt} \mbox{threat reporting.} \\  \hline

\mbox{Cloud} \mbox{Computing} Vulnerabilities & \mbox{Data Breach}, \mbox{Unauthorised} Access & Regular Audits, Enhanced Validation & \mbox{Secure DNS settings} \mbox{in cloud services.} \mbox{Prevents exploitation.} & \mbox{Framework for secure} \mbox{domain use in cloud.} \hline

\mbox{Mobile Device} Exploitation & Unauthorised Access, Financial Loss & \mbox{MFA, Security Awareness} Training & \mbox{Secures account} \mbox{access.} \mbox{ Raises awareness} of threats. & Mobile security awareness and protection. \\ \hline

\mbox{Political and} Information Warfare & Misinformation, Political \mbox{Manipulation} & Monitoring, Collaboration & \mbox{Monitoring abuse in} \mbox{campaigns. Unified} \mbox{response to } \mbox{misinformation.} & Transparency in monitoring and collective action. \\ \hline

\mbox{Exploiting} Emerging \mbox{Technologies} & system \mbox{Vulnerabilities} &\mbox{ Machine Learning,} \mbox{Collaboration} & \mbox{Analytics to predict} \mbox{DNS abuse. Share} \mbox{knowledge about} \mbox{threats.} & \mbox{Innovation in defense}  \mbox{strategies and sharing.} \\ \hline

\mbox{Supply Chain} \mbox{Attacks} & \mbox{System} \mbox{Compromise,} Data Breach & Regular Audits, Blacklisting & \mbox{Audits for DNS} \mbox{integrity. Rapid} \mbox{response to threats.} & \mbox{Transparency in supply} \mbox{chain security.} \\ \hline

\caption{Mitigation strategies against DNS abuse and its impact on users.} 

\end{longtable}

}


Finally, this chapter has examined all aspects of DNS abuse,  the various forms, the serious harm it does, as well as potential future threats. To create efficient regulations and countermeasures, it is essential to understand the extent and consequences of DNS abuse. Significant progress towards resolving these issues has been made by organisations like the DNS Abuse Institute and ICANN. However, as new technologies are incorporated into the equation and the threat environment changes in sophistication, it becomes increasingly important to adopt alert, flexible and cooperative strategies. The mitigation techniques and best practices discussed in this chapter provide a roadmap for mitigating DNS abuse. Every tactic contributes to a defence mechanism, from advanced technology solutions and improved methods for validation to monitoring and reporting. It is impossible to overestimate the value of cooperation, regular checks, and the application of cutting-edge technologies to anticipate and mitigate DNS abuse. After analysing the data, it is evident that a team effort is needed to comprehend, track, and mitigate DNS abuse. A complex strategy that integrates multiple techniques and encourages collaboration across industries is required instead of a single insufficient strategy. Our approaches to preserving the integrity and security of the DNS and, consequently, the larger Internet infrastructure must adapt, as does the digital environment.

By understanding the connections between different aspects of DNS abuse and reinforcing the collective effort required for effective mitigation, stakeholders can be better prepared to face the challenges ahead. This chapter sets the stage for further research and action, with the aim of contributing to a safer and more secure digital world.



\chapter{State of the Art}

This chapter explores the strategies used to mitigate DNS abuse and new developments in this field. Explore and evaluate the effectiveness and transparency of multiple mitigation techniques, including DNS filtering and threat intelligence, in which experts organise and analyse information about cyber attacks. Additionally, the use of domain-generating techniques and DoT and DoH are two novel forms of DNS abuse that are highlighted in this section. In addition, the role of AI and machine learning in identifying and mitigating DNS abuse is covered. The final half of the section includes a discussion on potential future research areas and technologies to improve DNS abuse mitigation. Case studies offer practical information on DNS abuse occurrences. 


\section{Current Strategies and Their Effectiveness in Relation to DNS Abuse}


DNS abuse presents a significant challenge for Internet entities involved in domain name management. Various approaches are employed to mitigate such abuse, including DNS filtering, which regulates access to specific websites and prevents you from accessing malicious sites that can administer phishing and ransomware. Additionally, threat intelligence methodologies use data analysis to identify potential risks, as exemplified by \cite{schmid2021thirty}. Anomaly detection plays a role in identifying suspicious DNS activities indicative of malicious intent using Packet Analysis to analyse individual packets for DNS allowing for real-time detection and statistical analysis, which involves performing statistical analysis on a large dataset of DNS traffic. However, these methods can face operational challenges, such as errors and the need for fast access to critical threat data. 

\subsection{Transparency in DNS Abuse Mitigation \& DNS Relevance}

\begin{enumerate}
    \item A Case Study of Cloudflare's Transparency Approach



Cloudflare is committed to maintaining transparency \cite{cloudflare_transparency_2022}, which is the keystone of their relationship with customers, guiding each of these approaches to reports of abuse of the DNS and requests that may come from law enforcement. All of these reduce their actions and policies in moulding a trustworthy environment in light of addressing Internet safety and privacy concerns. Their approach to handling DNS abuse reports and law enforcement requests is anchored on three core principles:


\begin{enumerate}
    \item Require due process: Cloudflare will comply with due process as required by law, remaining neutral and not exceeding legal requirements.
    
    \item Respect privacy: Cloudflare respects your privacy and will never sell or otherwise share any personal or private information with any third party without your explicit permission. This is applicable to each and every request.
    
    \item Provide Notice: Cloudflare will notify customers if legal requests are made for their information, unless prohibited by law.

\end{enumerate}

The Cloudflare Transparency Report gives deep statistics and trends based on DNS abuse reports on Cloudflare's response: 

\begin{enumerate}
    \item Abuse Reports: Cloudflare actively responds to abuse reports, including phishing, malware, and copyright infringement.

    \item Actions taken: Cloudflare may terminate services for domains involved in abuses such as phishing or malicious activities.
    
    \item Termination of services: 206 accounts and 530 domains are suspended for the latter half of 2022 for hosting domains associated with CSAM that were not compliant with reported issues.
    
    \item UDRP Requests: Cloudflare responded to 21 UDRP (Uniform Domain Name Dispute Resolution Policy) requests from an ICANN-approved dispute board in the second half of 2022.
    
\end{enumerate}

In addition , Cloudflare's careful description of compliance and due process with respect to handling law enforcement requests comes from their latest Transparency Report. Below is a summary of the major areas covered.

\begin{enumerate}
    \item Legal Sufficiency Review: Cloudflare reviews the legal validity of requests before taking any action according to the law and respects the privacy of the users. Cloudflare responds only to valid law enforcement requests, including court orders and subpoenas.

 \item Respect to International Privacy Laws: Cloudflare checks the disparity of the request, rejecting to honour it if it is at odds with the rights enshrined by the privacy laws of the two countries.


 \item Emergency disclosure requests: Cloudflare only considers making emergency disclosures to law enforcement where there is an imminent danger of harm and clearly requires a stated intention for the follow-up of legal processes.

 \item  National Security Requests and Non-Disclosure Obligations: National security orders are incompatible with the company's aims of transparency; thus, Cloudflare is challenging them.

 \item  International Requests for Data: Cloudflare reviews such requests from the foreign government in accordance with the United States legal process or on a case-by-case basis consistent with international norms and policies.

 \item  Challenging Overly Broad or Inappropriate Requests: Cloudflare may object to and help our customers fight overly broad or inappropriate requests from law enforcement as a matter of transparency and user rights. USA.
\end{enumerate}

Public reporting by Cloudflare and working closely with law enforcement, as well as other partners, form important elements in its strategy of mitigating DNS abuse such as: 

\begin{enumerate}

\item Reporting to the Public \& Transparency: Cloudflare opens its data to the public to showcase the types and the volume of abuse reports for trust-building and to exemplify the notion of efforts against antiabuse.

\item Law Enforcement Cooperation: Cloudflare partners with law enforcement regarding the privacy of citizens in such a way that it makes sure the proper legal justification of the actions in relation to DNS abuse.


\item Mitigation actions: Cloudflare makes a proactive effort to prevent DNS abuse by stopping access to any offending domains that are used in phishing, malware distribution, and other damaging activities.abuse.

\item Challenges to Mitigating DNS abuse: Cloudflare admits to striking a balance between free speech, legal matters, and the need for stakeholders' joint efforts in the quest to fight DNS abuse.

\item Efficiency of Efforts to Mitigate DNS Abuse: Although successful mitigation involves tackling the root causes and diversity within the market, Cloudflare's transparency reports help in understanding abuse mitigation efficiency.

\end{enumerate}

Some of the challenges the Transparency Report showcases that Cloudflare is grappling with the complexity of DNS abuse, how to achieve a balance between transparency and privacy, and being legally and regulatory compliant, besides also having technical limitations in the mitigation of misuse. This allows ideas to be used to improve processes.

\begin{enumerate}
    \item Enhanced Cooperation with Stakeholders: Cloudflare proposes that such a relationship with law enforcement, service providers, and international organisations will be an industry best practice to standard procedures to increase the efficacy of the internet by mitigating DNS abuse.
    
    \item Improve Abuse Detection Systems: Cloudflare said it would invest its capital in funding more development in advanced technologies, machine learning, and further enhancement of abuse detection systems to allow for faster identification and mitigation of abusive content.
    
    \item Transparency Reporting Enhanced: Cloudflare looks forward to providing even more granular transparency reports that would give deep insight into the types of abuse on DNS and the effectiveness of its mitigation for improved abuse handling practice.

    \item Better User Education \& Awareness: The focus is to develop more educational materials and programmes to inform the user, bringing to their notice the risks of cybersecurity and DNS abuse for a safer Internet environment.


    \item Advocate Policy and Legal Reforms: Cloudflare works to propose and advocate policy and legal reforms, which could be embraced in mitigating any potential collision points between privacy laws and law enforcement requests, hence balancing user privacy with the fight against DNS abuse.
    
    \item Create a Multi-stakeholder Feedback Mechanism: A mechanism can be put in place that ensures feedback not only from the users but also from the civil society and other stakeholders; that would bring out to what extent Cloudflare was able to be successful in their efforts towards transparency and reduction of abuse. Such recommendations can be useful for guiding any further policy-making or organisational policy enhancements.
    
    \item Continue to Challenge Over-broad Requests: The company does not even go on to commit to resisting overbroad or inappropriate requests for data; the user's rights and due processes are put above all others, which would not help harden their service as a trusted industry leader.

    
    
\end{enumerate}

In conclusion, the company emphasises its commitment to protecting legal processes and user privacy while navigating government and law enforcement requests. A critical aspect of these reports is Cloudflare's approach to DNS requests, particularly regarding content blocking through its 1.1.1.1 Public DNS Resolver.This was the key answer: Cloudflare, in no uncertain terms, "received legal requests to block content at our DNS servers" and stated its policy to first "exhaust legal remedies" that they could enforce. This is an indication of how very carefully Cloudflare has to adhere to the demands of the law, yet protect the openness of the Internet, bringing out just how major DNS is in all matters that pertain to the accessibility of content on the Internet and governance of the Internet.

\item Google Transparency Reports 

This shows the weight attached to the Domain Name System (DNS) when enforcing the requests from the global governments, more so in between them and the internet governance, in relation to the content removal from Google services. Data from Russia, with tens of thousands of redaction requests, might signal broader actions that include DNS-level interventions. This highlights the kind of role DNS plays in controlling access to the Internet or blocking content, which is usually put under legal and regulatory pressures from major tech companies, including Google.

Any question related to these requests, although not directly related to the manipulation of DNS, implies the possibility of any technical adjustment to be carried out in order to fulfil the criteria directly affecting DNS resolutions. This indirect reference considers DNS to be one of the critical infrastructures in the debate on Internet governance, censorship, and access to information. What it does is show the Google Transparency Report, which indicates the fact that DNS is an important architecture of the Internet and is also a trouble spot for exercising control over digital content and information flow \cite{Google2023}.

\item Amazon Transparency Reports 

Necessarily, such a role of DNS in servicing governments or other legal data demands does not trace directly to specific acts of manipulation in the DNS or intervention at the domain-level. The report explains about Amazon's observance of due process laws in handling requests for data such as subpoenas and search warrants, with a lot of emphasis on customer privacy and protection of data which can be mounted against the state or any other third party institution or person. It goes without saying that handling the domain or the services to do with this website means that a possibility of such a move as DNS changes can be in the offing. However, they do not give clear examples where DNS interventions have been taken, but describe the circumstances related to legal compliance and internet governance without direct reference to DNS \cite{Amazon2023}.

\item DNS- SB Transparency Reports

xTom reported nil compliance, for the most part, within the international statistics of content data requests, requests for information on subscribers, requests to have content taken down, requests to have content blocked, and domain name dispute resolutions in 2020. These zero compliances are placed to highlight the fact that the organisation, in reality, sets the protection level of user data and content integrity too high, which is part of a general position on how DNS and domains are managed for the protection of users and the achievement of operable thresholds \cite{DNSSB2023}.

\item The CyberGhost Transparency Report

An obvious upward trend of the recursion without DMCA complaints, along with flagging malicious activities, flashes up in each year, record by record, before a sudden spike around 2023. Given the growing level of claims and requests, CyberGhost still regards the No Logs policy as a strong sweat so they keep a keen eye and hence stays guardedly strong on the user's privacy and any request relating to DNS. The report is categorical with such an idea that even in the case of mitigating malicious activity, they do not involve logging of DNS queries or respective user activity; therein, the integrity of user data and an assurance towards compliance in privacy. DNS somehow plays a function in this case: It becomes evident that the design of the CyberGhost infrastructure is supposed to be resistant to infiltrators and, hence, capable of withstanding invasions and pressures in no less than those that would compromise an individual's anonymity and right to freely receive information via the Internet \cite{CyberGhostVPN2023}.

\item The Meta-Transparency Reports

At the same level of social media, the enforcement of intellectual property rights, including Facebook and Instagram, shall entail the enforcement of a comprehensive strategy targeting copyright, counterfeit, and trademark infringements, with an important focus on the Domain Name System (DNS) as the centre stage for such activities. The DNS serves both as a foundation for the distribution of information on-line and as a checkpoint in the enforcement process. For example, content removals from Facebook and Instagram amount to 447,123 and 297,356, respectively, in the first half of 2022. This shows a scenario in which interventions range from more than platform moderation to include DNS-level actions of deindexing websites or altering DNS records to block access to infringing content.

The sustained rate of content removals since the latter halves of 2020 and 2021 indicates a reliance on DNS mechanisms. This may explain the huge year-over-year drop in Facebook's copyright and counterfeit content takedown requests from 2020-2021. It would seem that Meta (the parent company of Facebook) may not work with DNS providers to have the offending domains taken down, but instead remove the infringing content. This underscores how important DNS is in the enforcement of intellectual property rights, the checks on the spread of the counterfeit, fake, and grey markets, and in protecting the rights of the owner of intellectual property and trademarks  \cite{Facebook2023}.

\item  T-Mobile Transparency Report

It outlines how the company complies with directions of the law in the management of requests for information from consumers, thus highlighting staying within customers' privacy and legal compliance. Details the approach and policies of the company in response to lawful requests on records of customers within T-Mobile, Metro by T-Mobile, and Sprint, now collectively T-Mobile USA, Inc. (TMUS). At the same time, it provides information about what TMUS does to protect consumers from unauthorised data access, including first-party requests made by the company itself, such as subpoenas, court orders, and warrants, with all processes required following the same. When sharing details surrounding the number and types of requests received in 2022, the report marks a heavy emphasis on TMUS’s efforts to take care of customer privacy and complying with applicable legal obligations. In the case of T-Mobile, it handled 301,388 subpoenas, mostly related to orders to disclose information about the subscriber, such as names and addresses, and 94,599 different types of warrants or search warrants, which can be after historical location data or the content of messages \cite{TMobile2022TransparencyReport}.

\item IBM 1H 2021 Law Enforcement Requests Transparency Report
 
 IBM focuses on data ethics and transparency, just as it has done throughout the years to build trust among clients. The emphasis is on who owns the data and promotes client data policies, belonging to the government, and being fair and not discriminatory. The IBM report aims to make it clear where the company stands on the issue of client data that go through government surveillance. Therefore, it advocated that governments make their request for information directly to the client and ensure that the engagements between them are strictly regulated by legal protocols, including Mutual Legal Assistance Treaties (MLATs). IBM received 27 law enforcement requests in the first half of 2021, most of them related to the provision of basic subscriber contact information. It underpins how rarely and seriously IBM views requests for customer data. This reflects how IBM is committed to client privacy and data protection by ensuring strict controls in relation to data access, including those prompted by legality and governance \cite{IBMTransparencyReport2023}.

 \item Trade Me Transparency Report

The Trade Me report underscores its promise to be candid and open with detailed accounts of dealing with the agencies of the New Zealand Government. This is its 11th annual disclosure. The report demonstrates an effort of Trade Me to strike a balance between legal obligation and privacy of its members by making proactive disclosures of member information to government bodies, establishing transparency, and upholding their values. This stands as a testimony to the very careful process of Trade Me's Trust and Safety team, ensuring the legality and necessity of information released in the aim of maintaining community safety within legal bounds. This practice develops confidence in members in each other and makes the online environment feel safer. Importantly, it provides a specific number of government data requests and disclosures for the year ended June 2023, revealing that Trade Me is keeping the discussion about government access to data open. For example, it refers to a 36\% reduction in the release of voluntary information to the New Zealand Police, which describes the caution of the company in terms of voluntary information release \cite{TradeMeTransparencyReport2023}.

\item Xiaomi Transparency Report: Government Requests for User Information (2022)

It indicates how Xiaomi processes user data requests from the government and testifies to this company's determination towards transparency and legality. Strives to follow technical and organisational practices set as standards within the industry in the world and full respect for the laws and regulations. This general review portrays Xiaomi as a transparent organisation in the way it handles various requests from the government, from the device level to financial and account-based data, underlining the trust that Xiaomi has built with consumers regarding their privacy and data protection. In 2022, there were 51 device-based requests, among the many applications received by the Indian government. Among the device inquiries, 49,683 devices were answered, with 32 in compliance. The Xiaomi compliance rate in India reached an impressive 62. 75\%. It is indicative of the fact that the company is usually under huge government inquiries from regions where it has big stakes and shows the nature of the requests that this company has always faced \cite{XiaomiTransparencyReport}.
 
\item eBay Global Transparency Report 

 The report is a demonstration of eBay's commitment to making the marketplace safe and reliable for the global community of buyers and sellers transacting on its platform. Defined with great focus, eBay lists everything they are doing to protect their marketplace from counterfeit goods, fraud, and any other abuse. With advanced AI technologies and image detection, eBay will be able to identify and remove listings of goods that could pose risks to safety or health, with close follow-up efforts to improve cooperation with rights owners and law enforcement. They are included in measures within the scope of eBay investments in technology and partnerships towards the retention of platform integrity. Reflecting the policies and their impact on the initiatives of the company for more than two decades, the report has highlighted that eBay believes in creating an open and honest marketplace that can help individuals generate economic opportunities from across the world. eBay AI tools had proactively stopped 295 million listings of prohibited items during 2022, a clear indication that its technology is very key to stopping the sale of controlled substances and other damaging items. On the other hand, the Authenticity Guarantee programme further underlines the quality consciousness of eBay and builds trust by allowing verification services for luxury offerings, which include watches, handbags, jewellery, sneakers, and cards \cite{eBayGlobalTransparencyReport2022}.

 \item Cisco Transparency Report: Government Data Demands (First Half of 2023)

 The report underscores Cisco's stated commitment to providing transparency about government requests for customer data around the world. This is an important lens into the world of data privacy and government surveillance, outlining the amounts and kinds of request for content and non-content data being made. The publication of compliance, rejections, and all cases in which data was not found shows the effort to balance legal obligations with the privacy of the customer. In particular, it is in compliance with national security demands, in accordance with the United States Freedom Act of 2015. According to the company's reports, during the specified period, Cisco received 16 demands from the US government agencies regarding non-content data and disclosed it in 7 cases. For this period, the company recorded a compliance rate of 44\% non-content data requests (NCDR). It is from such data that a comprehensive illustration of how Cisco handles government requests for data can be shown to have the regard given to government interests or balance with privacy rights. It also points to international involvement, where of the 27 requests from Germany, 25 were disclosures, therefore showing the extent and scope of the interest governments have in data \cite{CiscoTransparencyReport}.

 \item Apple Transparency Report : Government \& Private Party Requests 

It details the process by which Apple's legal team handles all legal requests from global government agencies and US private parties, categorising them by Devices, Financial Identifiers, and Accounts. This highlights the process that Apple undertakes with all the devotion to the protection of user privacy and information safety, at the same time dealing with the requests within legal standards. This commitment to transparency is aimed at building trust and informing opinions about Apple's operations. The report is key for any reader who is interested in understanding at a more detailed level the intersections of technology, privacy, and law enforcement in the digital age. The information describes the types and volumes of requests in which, for example, Apple reports having received 5,660 device requests in the US and reports that have furnished information for 82\% of these requests, mostly associated with investigations of lost or stolen devices or fraud. The U.S. posted a total of 7,944 account requests, with a disclosure rate of 47\%. This clearly proves that Apple has been pretty guarded in its responses to requests for user data. \cite{AppleTransparencyReportGB}.


\end{enumerate}

\subsection{Effectiveness of Current DNS Abuse Mitigation Strategies}

Different methods are used to mitigate DNS abuse, including the implementation of blocking tools, awareness of potential threats, and identification of anomalous behaviour. DNS filtering entails the regulation of website access based on predetermined rules, which can have varied outcomes depending on the context in which it can happen in different environments such as register and registry in which it implements mechanisms to compare DNS names to the block list and given set of rules then takes the necessary action such as homograph attacks in which DNS filtering mechanism play a role in mitigating them by comparing domain names against block lists and predefined rule to identify potentially malicious homographs as stated earlier. Threat intelligence plays a role in identifying potential dangers and detecting unusual activities within the DNS, as noted \cite{rizvi2022application}, such as allowing proactive identification and assessment of potential threats and malicious activities, including detecting patterns indicative of phishing, domain hijacking, malware distribution, and other forms of DNS abuse. Evaluating the effectiveness of these methods requires careful consideration of their performance in real-world scenarios. For example, while DNS filtering can effectively block malicious content, it may inadvertently permit harmful elements to bypass the filtering process, potentially impacting the user experience. Similarly, the effectiveness of threat intelligence relies on the timeliness and accuracy of the data used. However, identifying anomalous behaviour poses challenges, as distinguishing between malicious actions and legitimate activities performed in innovative ways can be challenging.


\section{Emerging Trends in DNS Abuse}

Trends in DNS abuse had declined among some categories, such as botnets, malware, phishing, and spam. Much of this decline could be attributed to the multipronged approaches that ICANN itself launched around data analysis, community tools, and enforcement of registry and registrar obligations \cite{icann_dns_security_threat}. Although continuing to be slow, adopting organisations did so under the compulsion of situations that left them no choice but to use technology or by those for whom TLS adoption was a matter of technological innovation, choice, or desire for the embrace of technologies simpler and more robust from misdirection. One of the major issues has continued to be privacy, due to the fact that DNS queries have been accidentally found to give away user behaviours. One such move to enhance user privacy is the Query Name Minimisation. The main concern has been how to remain vigilant against DNS abuses while improving privacy without altering service efficiency.

\subsection{Evolving New Forms of DNS Abuse}

The field of cybersecurity is rapidly advancing, bringing forth new challenges as it evolves, and constantly moving the goalposts for defence mechanisms. The introduction of DNS over TLS (DoT) and DNS over HTTPS (DoH) is like a double-edged sword. Although these encryption protocols were designed to enhance privacy and security by encrypting DNS queries, they unintentionally provide attackers with means to disguise malicious traffic. This expands the attack surface, affecting everything from individual devices to corporate networks. For example, attackers could take advantage of DoT and DoH in enterprise settings to avoid outdated security controls and establish hidden communication channels. Furthermore, Domain Generation Algorithms (DGAs) play an important role in cyber threats by automatically generating a large number of random domain names, making it extremely difficult to identify and shut down malicious sites. \cite{kaur2023artificial}. This tactic, integral to botnet command and control (C2) operations, significantly complicates cybersecurity defence efforts to predict and mitigate threats.

The adoption of DoT and DoH offers several benefits, such as enhanced privacy by preventing the surveillance of DNS queries and improved security through the encryption of DNS traffic, which weaken hackers' attempts to intercept or manipulate data. However, these protocols also allow attackers to hide their malicious activities, which poses challenges for traditional DNS security systems in detecting and filtering harmful content. Furthermore, these protocols could accidentally bypass content filtering policies, leading to potential security breaches within organisations.
Conversely, DGAs provide attackers with a method to evade detection and maintain C2 communications, as the dynamically generated domains are difficult to predict and preemptively block. This results in an overwhelming number of domain names for security mechanisms to monitor, complicating the threat intelligence process and necessitating continuous vigilance and blacklist updates. The widespread adoption of these technologies underscores the need for cybersecurity professionals to adopt a proactive and informed stance, understand their potential for exploitation, and develop comprehensive strategies. These strategies must strike a balance between the benefits of encryption and domain generation and the imperative to prevent DNS abuse, ensuring the integrity and security of the online environment.


\subsection{Predictive Measures \& Their Transparency}

Efforts to mitigate DNS abuse are set toward immediately slowing such activities by utilising complex systems and advanced machine learning algorithms to detect patterns indicative of DNS abuse. Articulating and sharing insights about the decision-making processes in predictive modelling is considered significant, as well as the efforts by registrars and registries, acting together, in the context of DNS Abuse Transparency are comprehensive. These entities will invoke a wide range of mitigation measures to minimise damage and losses related to DNS, which will ensure the development of a more secure and trusted Internet environment. Some key mitigation strategies are account-based remediation in the way that maliciously generated accounts are locked out and further validated, in addition to monitoring third-party feeds and reports from cybersecurity organisations, law enforcement, and the public to discover and address abuse early. Moreover, this mitigation involves malware analysis, which comes from attacks on the communication infrastructure and the corresponding IP addresses, through suppression or sinkholes in the context of botnets and the use of domain generation algorithms (DGA) that direct botnet traffic \cite{ M3AAWG2024}. Most specifically, sinkholing is an authoritative measure that directs traffic from abusive domains to harmless servers and allows studies to be conducted on the sources of traffic and the extent of compromise. Compliance with legal and contractual requirements further underscores the actions of registrars and registries against DNS abuse, ensuring that their actions in mitigation are within the context of the ICANN agreements and local laws. 

The evident evaluation of real-time black hole lists (RBLs), in addition to the responsible role of trusted notifiers, further increases the effectiveness and accuracy of mitigating actions, to filter and validate reports on abuse, so that proper responses may be made. This multipronged approach on the part of the registrars and the registries towards the mitigation of DNS abuse does not only emphasise the proactive and reactive measures, but also the possibilities of increased transparency as far as reporting and publicising the actions in place against DNS abuse are concerned. Such transparency is key to building trust, open to accountability, and creating an environment conducive to stakeholders' collaboration for the more effective fight against abuse in the DNS ecosystem. This transparency helps to understand the rationale behind the predictions, map the data used for model training, and clarify the methods that guide decision-making, as highlighted in \cite{hussain2022software}. Striking a balance between the complexity of predictive models and their interpretability is a significant challenge. Therefore, it is essential to approach this challenge with caution, ensuring that the models are not only effective in identifying DNS abuse but also accessible for thorough examination and accountability.


\captionsetup{font= footnotesize}
\begin{figure}[H]
\centering
    \includegraphics[width=1.0\linewidth]{background/DNSECO.png}
    \caption{DNS Ecosystem Contractually Related to ICANN (image
courtesy of Verisign and originally published in SSAC 115 adapted from \cite{SSAC2023SAC115})}
    \label{fig:fig14}
\end{figure}
\clearpage

\section{Technological Advancements}

The mitigation of DNS abuse is increasingly influenced by the integration of artificial intelligence (AI) and machine learning technologies \cite{goethals2021enabling}. At the helm of this evolution are innovative tools such as the iQ Domain Risk Score, which employs machine learning and string analytics to proactively detect potential domain abuses now of registration \cite{dnsabuseAI2023}. This tool aims to act as a mitigation measure by analysing domains against criteria indicative of malicious intent, thereby attempting to stop abuse before it even starts. Additionally, the field is witnessing a transformative shift in analysing abuse report evidence through the adoption of Large Language Models (LLMs), such as generative pre-trained transformers (GPTs). These models are highly adept at parsing and understanding complex data patterns that could be missed by human investigators, enhancing the efficiency and automation of DNS abuse mitigation efforts, and forming a more dynamic defence against cyber threats. However, this progress also highlights an emerging challenge: the potential for malicious entities to exploit AI technologies themselves \cite{halvorsenAI2023}.  Consequently, the intersection of AI and machine learning with DNS abuse mitigation not only heralds significant advancements in cybersecurity strategies, but also emphasises the need for vigilance to prevent these technologies from being used for harmful purposes. This pivotal moment in the fight against DNS abuse underscores the need for ongoing innovation and adaptation to effectively secure digital ecosystems.

\subsection{Role of AI \& ML}

The introduction of AI and machine learning technologies into DNS abuse mitigation marks the beginning of an innovative era focused on proactive detection and neutralisation of cyber threats \cite{tariq2023critical}.  This approach facilitates the rapid analysis of large datasets to uncover patterns indicative of malicious intent in DNS queries. For example, machine learning techniques have been highly effective in analysing DNS queries to classify domain names, significantly improving the detection of domains linked to malware \cite{LiMaliciousDomainDetection2020}. Furthermore, the application of neural network models, such as the Extreme Learning Machine (ELM), has achieved accuracy rates above 95\% in the identification of malicious domains, demonstrating the predictive power of AI in combating cyber threats \cite{ZouDNSGraphMining2015}. Additionally, the technique of DNS graph mining has illuminated AI's potential within cybersecurity frameworks, with methodologies like belief propagation algorithms achieving high precision in identifying infected hosts and malicious domains. These examples underscore the vital role of AI and machine learning in supporting DNS abuse, paving new avenues for early detection and swift mitigation of potential abuses. However, the complexity of AI models and the demand for transparency in their decision-making processes present ongoing challenges. Integrating AI into DNS abuse mitigation strategies improves security measures, but also requires careful attention to ethical considerations and the establishment of governance frameworks \cite{AntonakakisMalwareDomainsUpperDNS2011}. AI and machine learning can help improve DNS abuse mitigation, but experts must be clear about the problem.  It is important to understand how AI models make certain decisions. This helps build trust and ensures that people are responsible for them. There are difficulties in making things clear, such as needing to write down what data was used for training, telling others about the things that affect choices, and explaining how models change to face new risks. It is still difficult to find the right balance between the complexity needed for good threat detection and the openness needed for blame.

In summary,AI and ML are valuable in protecting against rapidly evolving cyber threats and a wide range of devices, including those in the IoT. However, their predictive accuracy can be limited by the quality and quantity of data used for training. Sophisticated attacks designed to evade detection algorithms present a notable challenge, underscoring the importance of continuous learning and adaptation in AI/ML models to maintain their effectiveness.


\section{Case Studies and Real-World Applications}

In recent years, technology has become so widespread that we have witnessed an unmatched number and complexity of cyber threats. A significant vulnerability that can be exploited is the DNS domain name system, a critical part of the internet infrastructure that translates human-readable names into IP addresses \cite{kumari2021sac115}. 

\begin{enumerate} 

\item\textbf{ Case Study 1: OilRig DNS Tunneling Attack }

The case of OilRig reflects the use of custom DNS tunnelling protocols for command and control (C2) operations, thus making it dual-use in nature, both in normal operation and on a fallback communication channel \cite{paloaltonetworks2021dnsattacks}.The xHunt campaign \cite{unit42_xhunt_2021} followed a similar trend of including Snugy backdoor implants in targets of Middle Eastern government organisations and keeping track of them using DNS tunnelling for communication with its C2.  These are examples that underscore the strategic use by adversaries of DNS tunneling techniques for stealthiness and resilience within the context of their operations \cite{unit42_2021}.

\captionsetup{font= footnotesize}
\begin{figure}[H]
    \centering
    \includegraphics[width=\textwidth]{background/DNSTuu.png}
    \caption{DNS tunneling communication between the attacker's command and control (C2) infrastructure and the victim's network.}
    \label{fig:figTen}
\end{figure}



\item\textbf{ Case Study 2: SUNBURST Use of DGAs}

SUNBURST backdoor associated with the breach of the SolarWinds supply chain represents a case in which the use of DGAs is critical, if not only, to conceal communications and system details \cite{paloaltonetworks2021dnsattacks}. The SUNBURST backdoor applies the deep use of DNS manipulation for evasion purposes and subsequent attack stages by encoding basic system identifiers and the usage of DGAs for C2 check-ins \cite{unit42_solarstorm_2021}.

\captionsetup{font= footnotesize}
\begin{figure}[H] 
    \centering
    \includegraphics[width=0.8\linewidth]{background/SUNDNS.png}
    \caption{SUNBURST backdoor's utilization of DGAs and its associated components.}
    \label{fig:figEleven}
\end{figure}

\item\textbf{ Case Study 3: Fast Flux Techniques}
The presence of several C2 domains related to the Smoke Loader malware family using Fast Flux techniques only further underscores the difficulties associated with the tracking and eradication of DNS-enabled threats. \cite{paloaltonetworks2021dnsattacks}.The major takeaway in the rapid rotation of IP addresses of this method points to the dynamism of strategies used in malicious communications, thus improving the means of defence by cybersecurity \cite{unit42_fastflux_2021}.

\captionsetup{font= footnotesize}
\begin{figure}[H]
    \centering
    \includegraphics[width=0.8\linewidth]{background/FastFluDNS.png}
    \caption{The usage of Fast Flux techniques by the Smoke Loader malware family for dynamic C2 domain communications.}
    \label{fig:figTweleve}
\end{figure}


\item\textbf{Case Study 4:  Malicious Newly Registered Domains (NRDs)}

Malicious NRDs crafted opportunistically in the context of the pandemic expose how threat actors exploit current events to engineer targeted attacks. \cite{paloaltonetworks2021dnsattacks} From domains that mirror the information resources of COVID-19 to those that feign government relief programmes, the evolution of such attacks reflects a calculated approach to exploiting public interest and vulnerabilities  \cite{unit42_covid19_phishing_2021} .

\captionsetup{font= footnotesize}
\begin{figure}[H]
    \centering
    \includegraphics[width=0.8\linewidth]{background/PandemicTime.png}
    \caption{The usage of Fast Flux techniques by the Smoke Loader malware family for dynamic C2 domain communications.}
    \label{fig:figThirteen}
\end{figure}

\end{enumerate}

In the coronavirus pandemic, too, phishing attacks changed to initially targeting PPE and testing kits, then turning to government stimulus programmes and subsequently enlisting vaccine distribution. Several of them, in fact, employed sophisticated tools, such as MFA pretending as the US Federal Trade Commission and brands such as Pfizer and BioNTech, to steal credentials. where it emphasised that there was a 530\% surge in vaccine-related phishing attempts and a 189\% increase in attacks on pharmacies and hospitals from December last year to February this year. Advice was given to individuals and organisations that includes being cautious in email and website transactions, advancing security awareness training, and adopting multifactor authentication.

Since January 2020, a total of 69,950 COVID-19 related phishing URLs have been received, of which 33,447 are specifically dedicated to COVID-19. The data have been normalised in such a way that the peak of each topic is at 100\%. The results showed much steadier phishing when it came to topics such as pharmaceuticals and virtual meeting platforms (e.g., Zoom) with vaccines and testing showing sharper rises and falls in the attention of scammers.

\captionsetup{font= footnotesize}
\begin{figure}[H]
    \centering
    \includegraphics[width=0.8\linewidth]{background/CovidPhising.png}
    \caption{Development trends in the majority of COVID-19-related phishing content hosting sites during the period from January 2020 to February 2021. Adapted from \cite{Unit42AtricleCovidPhishing2021}.}
    \label{fig:figFourteen}
\end{figure}

It is evident that a large portion of COVID-19 themed phishing pages targeted leading brands for phishing business credentials, such as Microsoft login, Webmail, and Outlook login. For example, about 23\% of these phishing URLs were posed as Microsoft login pages. This threat has particularly highlighted the shift towards remote work in the pandemic, and hence magnified the relevance of these attacks as one of the foremost methods that bad actors are taking on.


\captionsetup{font= footnotesize}
\begin{figure}[H]
    \centering
    \includegraphics[width=0.8\linewidth]{background/TOPCOVIDURLS.png}
    \caption{Top spoofed websites in COVID-themed phishing attacks (global), where the percentage in each column is the percentage of phishing volume per site and category. Adapted from \cite{Unit42AtricleCovidPhishing2021}.}
    \label{fig:figFiveteen}
\end{figure}

Thus, this clearly indicates a situation whereby the attackers set up websites frequently for COVID-19 themed phishing attacks. Many of these phishing pages are found on sites created less than 32 days, meaning that these sites are launched with specific purposes in view of these imminent attacks. The strategy allows attackers to customise their messages and URLs to the current pandemic trends, indicating the dynamism behind such cyber threats.

\captionsetup{font= footnotesize}
\begin{figure}[H]
    \centering
    \includegraphics[width=0.8\linewidth]{background/AgeCovid.png}
    \caption{Statistic of lifespan distribution of COVID-19-related phishing content hosting sites when the sites are reported. Adapted from \cite{Unit42AtricleCovidPhishing2021}.}
    \label{fig:figSixteen}
\end{figure}


\section{Challenges \& Future Directions}

Mitigating DNS abuse demands an immediate stop to the rapid evolution of cyber threats, underscoring the critical need for rapid global cooperation and the implementation of advanced technology. The key challenge is to achieve a fine balance between reducing false positives and accurately identifying genuine threats, while simultaneously advancing beyond the limitations of outdated technologies \cite{pour2023comprehensive}. The future of this domain largely depends on researchers' ability to enhance technological solutions, particularly focussing on the improvement of AI algorithms for deeper analysis of DNS traffic patterns. This opens a promising pathway for the creation and application of locally developed tools, providing innovative strategies to strengthen DNS defences. The ability to navigate the complex landscape of DNS abuse will require stakeholders to be agile in responding to emerging threats and developing novel solutions. The collective push towards the evolution of technology and methodologies will play a pivotal role in shaping effective DNS abuse management strategies in the years ahead.


\subsection{Identification of Current Challenges}

Mitigating DNS abuse involves developing strategies that should not only be proactive, but kept constantly up to date to handle the changing environment of cyber threats. The fluid nature of these threats means updating current protocols as well as developing new defence methods. With bad actors constantly reviewing their methods to take advantage of the vulnerability of DNS, it has become imperative that the cybersecurity industry continuously updates its defence mechanisms \cite{bhattacharya2023dns}. Being a global phenomenon, the Internet and hence DNS abuse being transnational in character, there is no other alternative than international cooperation. The effectiveness of DNS abuse management would be based on collaborative work across national borders, where experts in different geographical areas come together to share their knowledge and resources \cite{altulaihan2022cybersecurity}. The legal and regulatory framework varies in the various jurisdictions, making it difficult to reach a consensus on the regulations, standards, and enforcement actions. Another big challenge is that, to mitigate DNS abuse, the requirement is necessary to eliminate both false positives and negatives. Balance must be established in such a way that rather strict measures may reduce user experience, while, at the same time, being liberal might bring less detection of malicious activities. The cybersecurity community must continue to advance its detection and response capabilities, due to the increasing levels of sophistication used by DNS abusers. This will keep the security and integrity of the DNS system in good shape, thus protecting this vital part of the Internet infrastructure.

\subsection{Discussion on Future Research Directions and Technologies}

At the ICANN77 meeting, developments on mitigating DNS abuse were presented. These included the draughting of changes mandating that registrars and registries respond to abuse notifications, which contributed to a decline in global abuse levels after Freenom's legal response. Although the ccNSO Domain Abuse Steering Committee argued for a proactive mitigation strategy, the gNSO's analysis found minimal abuse rates in EU ccTLDs, which it attributed to market maturity and non-profit models \cite{VanRoste2023}. To improve domain security and enable international cooperation against changing cyberthreats, future initiatives will focus on creating cutting-edge tools and using technologies such as artificial intelligence and machine learning. This means that we need to look at more complex AI and machine learning tools that can understand the details of web traffic, which will make the results more accurate and stop the sending of wrong signals \cite{ISG2023}.

\section{Summary of Findings}

Research on the transparency of DNS abuse mitigation emphasises how threats are always changing and how mitigation techniques must evolve as well. It highlights how important community involvement and transparency are to fostering trust. Although technological advancements, especially in AI and machine learning, are essential for threat detection, their implementation must be carefully considered. Practical examples provide insight into the efficacy of various strategies. Maintaining a balance between new mitigation measures and effective teamwork and communication is a constant issue. To effectively address DNS abuse, future efforts should focus on using technology, international cooperation, and standardised information exchange.























\chapter{Research Methodology}



A structured questionnaire was sent by email to various stakeholders in the DNS ecosystem. This method was chosen because of its convenience, compliance with participants' busy schedules, and permission for detailed responses at the respondents' will. The approach provided a means of soliciting a wide range of expert observations on DNS abuse in terms of definition, the most prevalent types, mitigation challenges, and the theme of transparency. 



\section{Questionnaire Design and Distribution} 

The questionnaire had to take into account all of these issues in a multidimensional approach, giving great emphasis, but not limited, to address the nature of DNS abuse mitigation transparency.

The questions were specifically designed to extract in-depth information about:


 \begin{enumerate}
 \item The definition of DNS abuse. 
  \item The types of DNS abuse stakeholders most commonly encounter, aiming to identify prevalent patterns and specific concerns within the ecosystem.
  
  \item The challenges and limitations faced in mitigating DNS abuse, seeking to understand the barriers to effective action.
  
  \item The mitigation strategies used, gathering information on the practical steps taken and their perceived effectiveness.
  
  \item The practice of publishing reports or data as a form of transparency, exploring the current state of openness in the field.
  
  \item The role of transparency in aiding or impeding DNS abuse mitigation efforts, probing the potential impacts of increased visibility.
  
  \item  The effects of transparency on the relationships between various DNS stakeholders, considering the broader implications for cooperation and trust.
\end{enumerate}


\section{Stakeholder Responses} 

The insights from the completed questionnaire of the different stakeholders reflect several key themes and insights critical to understanding DNS abuse, as well as the mitigation of this abuse. These include various perspectives on what exactly the definition of DNS abuse is, the types of abuse most observed, and the difficulties experienced by stakeholders in efforts to mitigate its abuse. Additional discussions are related to mitigation methods, which provide a full view of current practices and potential areas for improvement with respect to the DNS ecosystem.

Key themes and insights :

\begin{itemize}
  \item \textbf{Varied Definitions of DNS Abuse:} Although stakeholders largely accepted the definition that had been adopted by the ICANN Contract Parties, they also noted its shortcomings, especially in being too categorical and thus may leave out evolving types of abuse. It was considered that a more flexible way forward would be a robust framework to define the abuses to be mitigated at the domain name level.
  
  \item \textbf{Common Types of DNS Abuse:} They pointed out that phishing was the most common attack type, followed by malware, botnets, and spam. It was also pointed out that one of the most common problems was related to the challenge related to proving the number of spam-related domains.
  
  \item \textbf{Challenges in Mitigation:} Perhaps the most significant was the economic structure of the domain registration industry, its ability to mitigate malicious registrations without fundamentally altering it. Stakeholders clearly state that a significant difference between large registrars, generally considered good actors on the Internet, and smaller registrars with a higher level of DNS abuse underscores the different aspects of this problem within different industry segments.
  
  \item \textbf{Mitigation Strategies:} The responses included different strategies, such as blocking orders from some regions or using software to monitor abusive activities. Recommendations were made on the role of education and outreach, including relevant projects such as NetBeacon and Compass to report abuse and information on DNS abuse.
  
  \item \textbf{Role of Transparency:} Opinions on transparency were mixed since part of the respondents consider this positively because it is a tool that provides evidence to the industry in its fight against abuse, part of them consider it negatively as sensitive mitigation ways could be revealed. The impact of transparency was also elaborated on developments in relationships between all stakeholders, and there is in general agreement that transparency will increase understanding and teamwork through better communication on measures set against abuse.
  
\end{itemize}

The responses of the stakeholders significantly enriched the research by providing a detailed look at the practical challenges and strategies to mitigate DNS abuse. The responses not only offered valuable real-world perspectives, but also highlighted the importance of adaptive definitions, comprehensive mitigation strategies, and thoughtful consideration of transparency's role in the ecosystem. This analysis bridges theoretical knowledge with the experiences of those actively involved in mitigating DNS abuse.

\section{Types of DNS Abuse Encountered} 

The stakeholder responses provided details of the most prevalent forms of DNS abuse that were being encountered within their specific ecosystem. These insights reveal a view of the various types of abuse, each of which poses unique challenges that require tailored mitigation strategies.

\begin{enumerate}
    \item Phishing: Stakeholders identified it as the most prevalent form of DNS abuse and the most visible. In fact, the total number of phishing incidents observed through tools such as NetBeacon and tracked by Compass is a stark and singular metric of just how big and urgent the problem has become in the wider DNS domain.
    \item Malware and Botnets: These also included malware and botnets, i.e. multifaceted DNS abuses. Such abuses not only compromise the integrity of systems but also present a security hazard to users and infrastructures in general.
    \item Spam: It is now recognised as widespread, and stakeholders have pointed out the challenges of quantifying and appropriately addressing the relevant spam-related domains. Therefore, it makes spam elusive for existing mitigation efforts that raise the bar with respect to the pursuit of next-generation detection and response mechanisms.
    \item Compromised CMS : Encounters with compromised content management systems (CMSs) have been referred to as common encounters. Consequently, such attacks are possible in cases of some other existing vulnerabilities in web platforms. This kind of abuse reinforces the need for strong web security control practices and the need for vigilance among platform operators.
    \item "Water Torture" Attacks: Known as random subdomain attacks, they represent a more technical and sophisticated form of DNS abuse. These attacks not only disrupt normal DNS operations but also require advanced countermeasures to effectively mitigate their impact.
\end{enumerate}

The varied nature of DNS abuse that stakeholders encounter underscores the fact that community efforts must continue to build on ongoing collaboration, innovation, and education to address these challenges effectively. This is derived from the experiences of stakeholders and forms a basis of paramount importance on which effective strategies and policies will be formulated in the mitigation of DNS abuse.

\section{Challenges in Mitigation and Mitigation Strategies} 

The responses of stakeholders demonstrated details of the multifaceted challenges in mitigating DNS abuse, coupled with the various strategies used to address these issues.
\begin{enumerate}
    \item Economic and Technical Hurdles: A significant barrier identified was the economic structure of the DNS industry, characterised by low margins and high volumes, often limiting the resources available for robust mitigation efforts against DNS abuse. Stakeholders highlighted that about 80\% of malicious domain registrations could be traced back to a mix of large, well-known registrars and smaller entities with disproportionately high levels of abuse. This economic reality complicates the implementation of effective mitigation strategies, underscoring the need for innovative solutions that are both cost-effective and scalable.
    
    \item Regulatory Gaps: The regulatory environment was also cited as a challenge, including poor, weak, or absent policies and enforcement mechanisms that could not effectively handle DNS abuse effectively. Stakeholders pointed out the necessity for clearer regulations and standards that can guide the industry's anti-abuse efforts more effectively.
    \item Mitigation Strategies: Stakeholders have responded to this with a variety of mitigation strategies. They placed an emphasis on components of education, collaboration, and outreach to raise awareness and develop a social response to DNS abuse. Technological solutions such as abuse reporting intermediaries (NetBeacon) and measurement projects (Compass) that measure the Internet are vital in finding, reporting, and understanding abuse cases. Designed to improve reporting and mitigation, these tools can also capture essential data with a character that helps inform policy and regulatory responses.
    
    
\end{enumerate}


\section{Transparency in DNS Abuse Mitigation} 

The responses of stakeholders underscore the subtle perspective on transparency within the DNS abuse mitigation framework, highlighting both its potential benefits and challenges.

\begin{enumerate}
    \item Benefits of Transparency: Increased transparency is widely recognised as a way to demonstrate commitment in the industry to defend against DNS abuse. It will encourage normalisation of mitigation efforts throughout the ecosystem, which means that proactive activity becomes more commonly adopted and attributed to a culture of responsibility and accountability. Transparency in reporting abuse metrics and mitigation outcomes can also enhance trust between users, regulators, and within the industry itself, promoting a unified approach to addressing DNS abuse. In addition, transparency is seen as a contributing element in improving understanding and cooperation among various entities involved in the DNS, including operators, registrars, registries, and regulators. By sharing information on abuse trends and mitigation strategies, stakeholders can better appreciate each other's challenges and contributions, leading to more effective collaborative efforts.

    \item Challenges and Concerns: Stakeholders raised several concerns about the degree and manner of transparency. One point of concern is that some sensitive mitigation strategies could be exposed that, in turn, could serve as a support for malicious actors, allowing them to discover ways to detect and mitigate abuse. This fine balance between providing useful information and protecting operational integrity is a significant challenge for many in the industry. Furthermore, there is apprehension that increased transparency might lead to regulatory or legal consequences, especially if disclosures are mandated in a manner that does not consider the practical aspects of abuse mitigation. The stakeholders also mentioned operational challenges, such as the capacity to complete transparency reporting, given the current reliance on less formal mechanisms for reporting abuse and monitoring mitigation.
    
    \item Strategic Approach to Transparency: Stakeholders advocate for a strategic approach to transparency that supports the goals of mitigation of DNS abuse without compromising the effectiveness of these efforts. This includes targeted transparency that focuses on aggregate data and trends rather than detailed disclosures of specific mitigation actions or techniques. Additionally, fostering an environment where sharing information does not lead to punitive outcomes, but rather supports collaborative improvement, is considered essential. Although the value of transparency for the mitigation of DNS abuse is considered high, stakeholders cautiously advocate that every step be done with care with regard to what, how, and to whom it shall be disclosed. A balanced approach that improves the collective ability to address DNS abuse while safeguarding the methods used is crucial for the ongoing evolution of transparency practices in the industry.

\end{enumerate}

\section{Impact on Relationships within the DNS Ecosystem} 

 Stakeholders pointed to a clearer potential impact on meaningful relationship building within their particular DNS ecosystems: greater transparency and mitigation. Better transparency is seen by creating a better understanding among different parties, for instance, among registries, registrars, and regulators about challenges and works against abuse, hence their collaboration and trust that improves combined efforts against abuse. However, this provision raises concerns that such transparency could get to the point of obstructing informal cooperation in general or actually reveal sensitive techniques from an operational standpoint detrimental to entities working together. Balance is a key element to ensure that these issues are addressed and that partners work harmoniously with each other within the DNS community.

\section{Analysis and Data } 

The detached examination of stakeholders' emailed responses in regards to DNS abuse and its various connotations is reported herein. In relation to those themes, the following record the main important points.


{
\begin{table}[H]
\centering
\footnotesize 
\begin{tabular}{|p{3cm}|p{9cm}|}
\hline
\cellcolor{gray!50}\textbf{Definition Supported} & 
\cellcolor{gray!50}\textbf{Comments and Suggestions} \\
\hline
\mbox {ICANN Contracted} Parties' Definition & Endorses the ICANN definition for its clarity and actionability. However, it suggests that it may be too narrow and advocates a more flexible framework to encompass evolving threats. Points to a self-authored sophisticated way of defining harms at the domain name layer, promoting adaptability. \\
\hline
\mbox {Critique of} ICANNwiki Definition & Finds the ICANNwiki reference lacking, preferring the SSAC 115 report definition for its broader applicability and recent adoption in RAA amendments. \\
\hline
Mixed Views & \mbox {While there's alignment with the existing categorical} \mbox {definitions for practical reasons, there is a shared belief in the} necessity for definitions that evolve with emerging DNS \mbox {threats. The discussion indicates a desire for a balance} between categorical clarity and adaptability to new forms of abuse. \\
\hline
\end{tabular}
\caption{Varied Definitions and Understandings of DNS Abuse}
\label{table:dns_abuse_definitions}
\end{table}

}

{
\begin{table}[H]
\centering
\footnotesize 
\begin{tabular}{|l|l|p{5cm}|}
\hline
\cellcolor{gray!50}\textbf{Type of DNS Abuse} & 
\cellcolor{gray!50}\textbf{Frequency Mentioned} & 
\cellcolor{gray!50}\textbf{Stakeholder Comments} \\
\hline
Phishing & Most Common & \mbox {Identified as the primary concern} \mbox {across responses, significant} impact observed. \\
\hline
Compromised CMS and Confusable Domains & Frequently Mentioned & \mbox {Highlighted as a prevalent issue} \mbox {alongside phishing and other} platform abuses. \\
\hline

\end{tabular}
\caption{Types of DNS Abuse Encountered}
\label{table:types_of_dns_abuse}
\end{table}
}

{

\begin{table}[H]
\centering
\footnotesize 
\begin{tabular}{|l|p{4cm}|p{4cm}|}
\hline
\cellcolor{gray!50}\textbf{Challenge Type} & 
\cellcolor{gray!50}\textbf{Stakeholder Insights} & 
\cellcolor{gray!50}\textbf{Suggested Solutions} \\
\hline
Economic & \mbox {High volume, low margin} \mbox {business model impedes} anti-abuse efforts. & \mbox {Calls for industry-wide} collaboration and support. \\
\hline
Regulatory Gaps & \mbox {Lack of clear regulations} \mbox {complicates mitigation} efforts. & Advocates for establishing and following industry-wide best practices. \\
\hline
\end{tabular}
\caption{Challenges in Mitigating DNS Abuse}
\label{table:challenges_in_mitigation}
\end{table}


}

{

\begin{table}[H]
\centering
\footnotesize 
\begin{tabular}{|l|p{4cm}|p{4cm}|}
\hline
\cellcolor{gray!50}\textbf{Strategy} & 
\cellcolor{gray!50}\textbf{Description} & 
\cellcolor{gray!50}\textbf{Stakeholder Feedback} \\
\hline
Blocking Orders & \mbox {From certain regions to} mitigate abuse. & \mbox {Implemented alongside} \mbox {other criteria to make} \mbox {services less appealing to} abusers. \\
\hline
Education \& Collaboration & \mbox { Outreach to improve} awareness and cooperation. & \mbox {Viewed as essential, with} \mbox {a need for more systematic} implementation. \\
\hline
\end{tabular}
\caption{Mitigation Strategies Employed}
\label{table:mitigation_strategies}
\end{table}

}

{
\begin{table}[H]
\centering
\footnotesize 
\begin{tabular}{|l|p{4cm}|p{4cm}|}
\hline
\cellcolor{gray!50}\textbf{Aspect of Transparency} & 
\cellcolor{gray!50}\textbf{Benefits} 
& \cellcolor{gray!50}\textbf{Concerns} \\
\hline
Reporting Abuse Metrics & \mbox {Enhances trust and} \mbox {accountability in the} ecosystem. & Risk of exposing sensitive mitigation techniques if not managed carefully. \\
\hline
\end{tabular}
\caption{Transparency in DNS Abuse Mitigation}
\label{table:transparency_in_mitigation}
\end{table}


}

{

\begin{table}[H]
\centering
\footnotesize 
\begin{tabular}{|l|p{4cm}|p{4cm}|}
\hline
\cellcolor{gray!50}\textbf{Relationship Aspect} & \cellcolor{gray!50}\textbf{Positive Impacts} & \cellcolor{gray!50}\textbf{Potential Challenges} \\
\hline
Between Entities & Improved understanding and collaboration from shared data. & Concerns about competitive sensitivity and operational integrity could limit openness. \\
\hline
\end{tabular}
\caption{Impact on Relationships within the DNS Ecosystem}
\label{table:impact_on_relationships}
\end{table}
}

In analysing the data from the stakeholder responses, a thorough examination of DNS abuse has been carried out. The stakeholders, deeply embedded in the DNS ecosystem, provide valuable information on the definitions and manifestations of DNS abuse. They highlight phishing as the most common and worrying type, with compromised CMS and confusable domains also noted for their prevalence. Challenges in mitigation are primarily related to economic factors and regulatory gaps, where the structure of the industry alters mitigation abuse actions and the lack of clear regulations muddy the waters. Mitigation strategies like targeted blocking and collaborative education are in play, though their implementation faces hurdles due to the industry's focus on throughput and the capacities of various entities. 
The role of transparency is acknowledged as double-edged; although it could foster accountability and trust, there is a risk that bad actors exploit sensitive techniques. Stakeholder experiences and strategies contribute to a deeper understanding of DNS abuse, suggesting the need for a multifaceted approach that involves adaptation, collaboration, and a careful balance of transparency.


\chapter{Draft}

\section{Confusable Domains}
\subsection{Identification and Examples of Targeted Domains}

The choice of such domains to target and outsource depends on many factors, each with its implications on the business strategy, marketing, and observance by law. The selection of these domains hence matters a lot in creating potential conflict especially those related to existing trademarks. Understanding these selection criteria is very important to try to negotiate the hurdles of the digital market and to protect rights through intellectual property. 

To navigate these complexities effectively, it is essential to consider several key factors. 

\begin{itemize}
  \item \textbf{Commercial Appeal:} High commercial appeal domains are lucrative targets due to the extremely high possibility of attracting a large traffic flow, with potential revenue generation. Such names are easy to remember, short in length, and directly linked to products or services under some category that is searched most frequently.\cite{Li2002ConflictDomainTrademark}
  
  \item \textbf{Keyword Relevance:} Targeted domains have a certain relevance that holds the keyword itself. These domains are ranked higher in search engine outputs and attract organic traffic, making them a useful tool for businesses aiming to align with the primary keywords used by their target customers.
  
  \item \textbf{Similarity to Well-known Trademarks:} This refers to the practice of registering domains that are similar or confusingly like existing trademarks—known as cybersquatting. This can lead to confrontations with the rightful trademark holders. Trademark law aims to prevent consumer confusion and protect the goodwill associated with the trademark, particularly in disputes over domain infringement.
\end{itemize}

\subsection{Real-life examples}

\begin{itemize}
    \item \textbf{Cybersquatting :} is securing domain names that are the same as or in the likeness of trademarks or brand names, with the intent to sell them at grossly marked-up prices back to the true owner. Among the most notorious examples are several court battles based on the purchase of "nissan.com" and "nissan.net" by a person named Nissan, who owned Nissan Computer Corp many years before Nissan Motors tried to acquire it. This case highlighted tension between the protection of trademarks and individual's interests in using their names for domain name purposes as well as it underscores the challenges of domain name registration and trademark protection in the digital age where domain names closely resemble established trademarks, potentially leading to consumer confusion and dilution of trademark’s value. \cite{Rosznyai2005}
    
    \item \textbf{Typosquatting- URL hijacking :} it deals with the registration of misspelled variants of well-known domain names for the mere purpose of capturing traffic from users who tend to make mistakes in typing a URL. They could register "goggle.com" instead of "google.com" which was used to direct users to a site that bombarded their browsers with pop-ups and ads , leading to malware infections as that site was designed to capitalize on accidental misspellings or phishing  attempts that tricked users into visiting. \cite{SplunkTyposquatting}

     \item \textbf{Reverse Domain Name Hijacking  :} is the act of trademark owners trying to take a domain away from its rightful holder based on the claim of trademark rights, considering that he holds a bona fide registration over the said domain. It may otherwise be described as using legal or dispute resolution mechanisms to try to force people from their domains. \cite{Sun2006DomainTrademarkConflict}

    An RDNH was claimed in a UDRP action against "groovle.com," in which the domain was purported to be too close to Google's trademark. However, since the domain was used for another search engine, it was deemed legitimately used and not to have infringed on Google's trademark or registered in bad faith. \cite{Singh2011ReverseDomainHijacking}

    \item \textbf{Homograph attacks:} are those cases in which people have registered domains using characters that kind of look similar to those used on legitimate sites for instance, using a homographic character to make 'Google.news' appear as 'google.news'. This technique has been followed by fraudsters who are now imitating various Google domains for phishing purposes. The group said it reported similar flaws with applications such as Signal and Telegram, where homograph attacks deceived the user to visit harmful sites by using Unicode in domain names to look like famous brands, including phishing attempts on MyEtherWallet and GitHub users. \cite{Leyden2021}
\end{itemize}

\subsection{ Real-life Mitigations}

The following scenarios are examples of real-life confusable domain mitigations :

\begin{itemize}
    \item \textbf{Cloudflare's Zero Trust Services Approach :} Protection from this problem of newly created phishing websites is given by Cloudflare itself with its protection in the form of Zero Trust services, finding these websites, and blocking confusable domains. Cloudfare zero-trust rules can be enforced using Cloudfare Gateway in a way that they deny access to these illegitimate domains. In such a way, corporate networks are supposed to be secured from phishing attempts that take advantage of human trust in well-known brands. \cite{Cloudflare2023}

     \item \textbf{Swift-URL-SpoofCheck Initiative: } On this open source initiative hosted on GitHub, a domain renderer has been released for the WebURL format defined according to the algorithmic rules established by IDN spoof checking in Chromium has been released. This tool intends to provide the user with an additional form of safety from domains that look confusingly like known-good domains by integrating most but not all the rules that Chromium uses in the label verification process. It includes functionality to identify potential spoofs and visualise how what appear to be valid domains could be deceptive. \cite{KarwaSwiftURLSpoofCheck}
     
     \item \textbf{IDN Handling of Google Chrome: } Google Chrome enforces an IDN (Internationalized Domain Names) policy to determine which form the Unicode or punycode form a domain label should be displayed in. The domain label is tested whether it has mixed script, invisible characters, or visually confusable characters, and whether it is actually validly converted into Unicode. For instance, domains containing characters of different scripts, or those that are clearly identified as mixed script confusables, will be displayed in punycode, warning the users of potential deceptions. Chrome further offers comprehensive warnings to secure URLs that appear to be an imitation of already known web pages. \cite{ChromiumIDN}
     
\end{itemize}

In addition to what I mentioned above, let us look at the most popular mitigations used world-wide :

\begin{enumerate}
  \item Deployment of DNS Security Extensions (DNSSEC): DNSSEC introduces an additional security layer to the DNS query and response mechanism, helping protect against DNS spoofing attacks that often accompany confusable domain strategies. 
  \item Typo-squatting Detection Tool: Tools such as DNStwist and URLCrazy are used to offer organizations similar domain names so that they can either secure these domain names in advance or file litigation for the same.
  \item Anti-Phishing Working Group (APWG): It is a pool for stakeholders to share intelligence, trends and best practices regarding phishing and similar threats associated with confusable domains in which mitigation is carried out in collaboration action between cybersecurity entities and domain registrars, as it allows sharing of threat intelligence with respect or cancelling out the holding of malicious domains.
\end{enumerate}

\subsection{Collaboration Among Registrars, Registries, and DNS Collaborators}

This collaboration should be achieved with DNS registry, registry and collaborators. In that way, they can boost common resources and intelligence that can guide in making the internet more secure and resilient. This strictly falls within the remit of registries and registrars acting in collaboration to put in place such stringent registration policy with procedures for verification, checking against mimicking existing trademarks or even popular domain names.

In this way, the collaboration can even manifest itself via the sharing of sensitive data with regards to domain abuse threats and trends. Databases and threat intelligence platforms are shared amongst stakeholders, allowing them to anticipate and avert most such perils well before they impact netizens. This collective effort will enable the formulation of standards by which to coordinate responses to confusable domain incident reports. Mitigating confusable domains demands that registrars, registries, and DNS collaborators work in a common effort. This is due to the increasing level of threats and the shared responsibility of all the actors involved in the DNS ecosystem. \cite{Catania2022} To put this into perspective, here are some examples: 


\begin{enumerate}
  \item Recent changes in the contract from ICANN's contracted parties have imposed on registrars and registries new specifications to define DNS abuse, together with clear requirements for the actions to be taken by such parties immediately actionable evidence of abuse is received. This is a major step towards establishing more clarity about the roles that may be played by these different stakeholders in addressing the matter of DNS abuse and ensuring there is a common approach to redress. \cite{Weinstein2023}
  \item Approved new obligations of ICANN's contracted parties have been by the community itself further to mitigate DNS abuse, thereby demonstrating the will of the community to come together to address the issues of DNS abuse. \cite{ICANN2023}
  \item Efforts like NetBeacon, with the support of the DNS Abuse Institute, are being rolled out to reduce friction in reporting and mitigating DNS abuse. This service solves the current complexities and quality standards associated with the reporting of DNS abuse as it makes the work easier for the registrars, ultimately narrowing down their scope to the relevant and evidenced report as well as it underlines the need for cooperation among registrars, registries, and other DNS stakeholders. This is what is capable of saving the Internet and safeguarding at the same time the credibility and confidence of DNS. \cite{NetBeacon}
  
\end{enumerate}

\subsection{Techniques for Mitigating Confusable Domains}

Mitigating confusable domains requires sophisticated techniques tailored to address the unique challenges presented by both non-Internationalized Domain Names (non-IDNs) and Internationalized Domain Names (IDNs). This differentiation is crucial due to the distinct nature of threats they pose and the technical feasibility of the mitigation strategies applicable to each. Below is a detailed examination of mitigation techniques, along with discussions on the operational feasibility and potential collaboration frameworks involved.

Non-IDNs Mitigation Techniques : For non-IDNs, strategies focus on identifying and preventing domain squatting and typo-squatting, where attackers register domains that are typographical errors or close variants of legitimate domains to deceive users.

\begin{enumerate}
  \item Registry-Level Measures: Domain registries can implement checks to prevent the registration of domains that are too like existing trademarks or brand names, using algorithms to detect variations and misspellings closely resembling protected names. \cite{WTR2020} 
  \item Trademark Protection Programs: Services like the Trademark Clearinghouse (TMCH) offer mechanisms for trademark holders to protect their rights by receiving notifications when someone attempts to register a domain matching their trademark. \cite{ICANNTMCH}
  \item Automated Monitoring and Reporting: Automated systems can continuously monitor domain registrations for names that closely resemble known trademarks or brand names, enabling rapid detection and legal action against infringers. \cite{TMCH2023}
\end{enumerate}

IDNs Mitigation Techniques : The challenge with IDNs lies in the potential for homograph attacks, where attackers use characters from different scripts that appear visually like characters in the Latin script to create deceptive domains.

\begin{enumerate}
  \item Punycode Awareness and Monitoring: Web browsers and security tools convert IDNs to punycode, a representation that encodes the Unicode characters in ASCII. Awareness of punycode and monitoring for suspicious registrations can help identify potential homograph domains. \cite{SOCRadar2023}
  \item Browser-Level Defenses: Modern web browsers have implemented defences against IDN homograph attacks by displaying the punycode version of the domain or alerting users when a domain name contains characters from multiple scripts. \cite{Malwarebytes2017}
  \item Collaborative Blacklisting and Sharing of Threat Intelligence: Organisations can collaborate to share intelligence about known malicious IDNs, contributing to comprehensive blacklists that can be used by registrars, DNS providers, and end-users to block access to malicious sites. \cite{CyberThreatAlliance2023}
  
\end{enumerate}

\subsection{Technical and Operational Feasibility}
The technical feasibility of these techniques varies. Registry-level measures and trademark protection programmes are highly effective, but require cooperation and standardisation across different legal jurisdictions. Automated monitoring is technically feasible and can be implemented at scale but requires resources for continuous operation and legal follow-up. Browser-level defences are among the most directly impactful, protecting users at the point of access, yet they depend on browser vendors' willingness to implement and maintain these features, and collaboration frameworks play a crucial role in mitigating confusable domains. Initiatives such as the Trademark Clearinghouse (TMCH) facilitate cooperation between trademark holders and domain registries. Meanwhile, organisations such as the Anti-Phishing Working Group (APWG) and the Internet Corporation for Assigned Names and Numbers (ICANN) work toward broader solutions that encompass both non-IDNs and IDNs.

\subsection{ Transparency in Mitigation Efforts}

Transparency in the mitigation of confusable domains plays a pivotal role in the broader strategy to secure the Internet against phishing attacks, trademark infringement, and other malicious activities. This concept entails the practices adopted by domain registries and registrars in identifying potentially malicious domains that mimic or closely resemble legitimate ones, and the extent to which these entities disclose identified confusable domains to the public. One of the primary methods to improve transparency involves the publication of lists of confusable names by registries and registrars. These lists typically include domains flagged for their similarity to existing domain names, potentially infringing trademarks, or those that could be used for malicious purposes. The publication aims to alert the internet community, including businesses and end-users, about possible threats, thereby fostering a proactive approach to domain name security. Here is how transparency can be applied to each of the mitigation techniques described:

\begin{itemize}
  \item \textbf{Cloudflare's Zero Trust Services Approach: } Cloudflare's process for identifying and blocking confusable domains should be transparent to its users. This includes detailing the criteria for flagging domains as phishing sites and the mechanisms in place for users to appeal or request a review of blocked domains. By openly sharing the methodology behind their zero-trust rules and how they are applied through the Cloudflare Gateway, trust in Cloudflare's protective measures is bolstered among corporate networks.
  
  \item \textbf{Swift-URL-SpoofCheck Initiative:} Transparency in this open-source project is inherent through its availability on GitHub, allowing users and developers to scrutinize, contribute to, and understand the tool's underlying logic. The initiative should continue to provide comprehensive documentation on how it integrates Chromium's IDN spoof-checking rules and the rationale behind excluding certain steps. This openness encourages community engagement and continuous improvement of the tool.
  
  \item \textbf{IDN Handling of Google Chrome:}Google's approach to displaying domain names in Unicode or punycode based on their potential for deception benefits from transparency about its IDN policy. Detailed explanations of the checks performed (e.g., mixed script detection, invisible characters) and how decisions are made improve user understanding and awareness of potential threats. Furthermore, publishing information on how users can report misclassified domains or suggest improvements to the IDN policy can further empower users and foster a safer Internet environment.
  
  \item \textbf{Deployment of DNS Security Extensions (DNSSEC): }For DNSSEC to effectively improve trust in the DNS, the processes for securing DNS records and verifying DNS responses should be transparent to DNS administrators and end users. Providing educational resources and clear guidelines on how to implement DNSSEC can help demystify this complex security layer and encourage its widespread adoption.
  
  \item \textbf{Typo-squatting Detection Tools: }The effectiveness of tools like DNStwist and URLCrazy in helping organizations identify potential confusable domains relies on transparency about how these tools generate similar domain names and the criteria used for detection. Openly sharing updates, methodologies, and case studies can help organisations better understand how to use these tools proactively.
  
  \item \textbf{Collaborative Efforts and Intelligence Sharing: }The partnership between cybersecurity entities and domain registrars, as well as initiatives such as the Anti-Phishing Working Group (APWG), should prioritise transparency in their operations. This includes the sharing of methodologies for threat detection, the criteria for taking action against malicious domains, and the processes for stakeholders to contribute or access shared intelligence. Transparency in these collaborative efforts ensures that actions taken against confusable domains are fair, understood by all parties involved, and supported by a broad community of internet security stakeholders.
  
 \item \textbf{Transparency for non-IDN registries : } 

 \begin{enumerate}
  \item Registry-Level Measures: Transparency in level-registry measures becomes a necessity if trust has to be kept between registrants and domain trademark owners. They are published criteria and algorithms used to find variations and misspellings of the names submitted for protection. Making these publicly available can then ensure fairness and feedback in detecting mechanisms is therefore paved for improving them.
  \item Trademark Protection Programs: Services such as those from a Trademark Clearinghouse (TMCH) should operate with full transparency regarding the conductance of verification, matching, and notification. This can help trademark holders by demonstrating transparent guideline procedures that show both the rights of trademark holders and what needs to be done to effectively protect their brands.
  \item Automated Monitoring and Reporting: The automated monitoring systems must be built with such predefined criteria, algorithms, and thresholds that potentially support the involvement of stakeholders. It also makes sure that the brand owners are aware to what extent his trademarks are protected, and thus allows for some parameters within such services making the monitoring more successful.
\end{enumerate}
 
 \item \textbf{Transparency for IDN registries  : } 

  \begin{enumerate}
  \item Monitoring and Identifying Measures for Suspicious Punycode Registrations: All domain registrars and trademark owners, together with security professionals, must adhere to measures on suspicious punycode registrations. Publicising the details of activities carried out to monitor them propagates homographic threats through collective ideas, also in their identification and mitigation.
  \item Browser-Level Defences: A good web browser should play the most critical role in defending against IDN homograph attacks. Browsers must document, provide, and communicate their defence mechanisms in a clear and plain manner to users. For instance, they should indicate when a domain is being displayed in puny code or the scenarios under which their warnings are triggered. This can only give a user assurance if there is transparency as a rationality measure for them in such defence measures to be able to rationalize the triggering of any warning and know what action to take.
  \item Collaborative Blacklisting and Shared Threat Intelligence: Processes for the addition of domains to blacklists and criteria that determine whether a domain is to be considered malicious should be examined. Organisations that put in place an intelligence sharing regime also need to have some rules on data submission and validation and data disposal from blacklists. Transparency in these processes can best ensure that blacklisting is done fairly and accurately and allows the right of appeal, improving trust in collaborative security.
\end{enumerate}
  
  
\end{itemize}

In summary, transparency across all these mitigation techniques not only builds trust among users, developers, and organizations but also enhances the collective ability to respond to and prevent the threats posed by confusable domains.

\subsection{Benefits of Transparency }

The benefits of transparency in the context of confusable domains are multifaceted. Firstly, it promotes accountability among domain registrars and registries, encouraging them to actively participate in the detection and mitigation of confusable domains. Second, transparency acts as a deterrent to malicious actors who might otherwise exploit the anonymity afforded by a lack of public scrutiny. Third, by making such lists public, registries and registrars can empower businesses and trademark owners to take timely action to protect their brands, such as through legal mechanisms or domain purchases. Furthermore, transparency supports community-based mitigation efforts, where cybersecurity researchers and the wider community contribute to identifying and neutralizing threats. This collaborative approach leverages the collective expertise of the cybersecurity community, enhancing the overall effectiveness of mitigation strategies.

\subsection{Drawbacks and Security Concerns} 

However, the publication of confusable domain lists is not without its drawbacks and security concerns. One major concern is that making such lists public could inadvertently provide a roadmap for malicious actors, highlighting potential targets for exploitation. This could lead to a situation where attackers use the information to refine their strategies, for example, by registering domains not yet identified or listed, thereby staying one step ahead of mitigation efforts. Another concern revolves around the risk of false positives, where legitimate domains are mistakenly flagged as confusable. This could harm businesses and individuals whose domain names are wrongfully listed, potentially leading to unwarranted scrutiny, legal challenges, and reputational damage. Moreover, the debate between transparency and security also touches on the effectiveness of disclosure in preventing attacks. While transparency aims to preemptively combat threats, there is an argument that the sheer volume of domain registrations and the dynamic nature of domain abuse may limit the practical utility of such lists to end-users and businesses.

\subsection{ Analysis : Feasibility and Practical Challenges}

 \begin{enumerate}
  \item Automated Monitoring and Reporting: Feasible; Technology exists to automate monitoring, even though the refinement of algorithms to decrease false positives and negatives from human review can probably not be undertaken with existing resources.
  \item  Monitoring Punycode Registrations: Feasible; This option is feasible and will require mainly the use of existing technology and cooperation that could be initiated with little difficulty between relevant stakeholders.
  \item Blacklisting and Threat Intelligence Sharing: Moderately Feasible; Since agreement could be reached on shared platforms and protocols, but they imply strong cooperation and trust among such diverse entities, which is unlikely to be developed fast.
  \item Registry-Level Measures: Not Feasible; this would require very heavy coordination and agreement on standards across diverse jurisdictions and registries, very complex in nature and long-drawn.
  \item Trademark Protection Programs: Moderately Feasible; They are well-functioning processes under such adequate structures like TMCH and can be learnt while proceeding with experience, but likely to face legal and operational issues.
  \item  Browser-Level Defences: Not Feasible; While this is technically feasible, it seems rather infeasible soon that user practices will become uniform across all web browsers and that all users will be well trained in various security practices.
  \item Cloudflare's Zero Trust Services Approach: Feasible; since well-architected infrastructure and broad adoption have made Cloudflare zero trust rules simple and effective to deploy, with a balance of security and operational efficiency without seismic root and branch changes.
  \item IDN Handling of Google Chrome and Browser-Level Defences: Feasible; Given that Chrome today has an enormous user base and that the groundwork for stopping homograph attacks already exists, it stands to reason that a solution is reasonably possible, meaning not too difficult, within a set timeline, and within the lifespan of any other typical software product.
  \item Less feasible options that would introduce some serious friction or are the use of DNSSEC (Domain Name System Security Extension), which is a solution that has a serious implementation downside due to its complexity and the need for stakeholders' absolute timing and synchronisation.
  
\end{enumerate}





\chapter{Evaluation \& Discussion}

In this chapter, the focus will be on evaluating and discussing the project and, most importantly, two forms of DNS abuse, which are confusable domains and phishing due to their popularity among bad actors by testing and validating them.  The chapter will dive into real-life examples to illustrate the severity of these threats and examine existing mitigations and techniques used to mitigate them, to test how well the project met the objectives. In addition, a proposal will be made to improve the transparency around these mitigation strategies to foster accountability and trust. Through the analysis, the feasibility of implementing such transparency measures will be assessed by performing an analysis using data and evidence. Finally, the limitations of the work will be addressed. 


\section{Confusable Domains}
\subsection{Identification \& Examples of Targeted Domains}

The choice of such domains to target and outsource depends on many factors, each with its implications on business strategy, marketing, and law enforcement. The selection of these domains hence matters a lot in creating potential conflict, especially those related to existing trademarks. Understanding these selection criteria is very important in trying to negotiate the hurdles of the digital market and in protecting rights through intellectual property.

\begin{itemize}
  \item \textbf{Commercial Appeal:} The domains with commercial appeal attract traffic and can generate income. They are short and memorable and relate to popular products or services, sometimes leading to ownership arguments \cite{Li2002ConflictDomainTrademark}.
  
  \item \textbf{Keyword Relevance:} Similarly, a domain specific to keywords ranks well in the search and indirectly lures organic traffic, which will help enormously the business in the search for common search queries and, in this process, generates more clicks.
  
  \item \textbf{Similarity to Well-known Trademarks:} Domain names, when similar to those of famous trademarks, can cause legal trouble for the trademark owner. laws prevent confusion and protect the brand reputation in disputes.
\end{itemize}

\subsection{Real-life examples}

\begin{itemize}
    \item \textbf{Cybersquatting :} A famous case was of Amul in 2019-2020. The renowned dairy giant had an impersonation case of similar domain registration, which was used to lead the public toward phishing, such as for fake distributorship or job opportunities. The Indian dairy brand above had this issue repeated three times in the last three years, namely from 2018 to 2020, and the company had to publish public notice while sending legal bodies, which also exhibited to readers the extent to which one can go for the brand defence issue.\cite{MehtaCybersquatting}.
    
    \item \textbf{Typosquatting :} One of the US healthcare providers, Elara Caring, gives an illustrative example of a cyberattack it encountered in December 2020: as a result, the following breaches in healthcare cybersecurity were defined: unauthorised access to email accounts of staff members. The breach, which lasted for a week, underscores the need for an improved incident response \cite{PandaSecurityPhishing}.
    
     \item \textbf{Reverse Domain Name Hijacking  :} is the act of trademark owners trying to take a domain away from its rightful holder based on the claim of trademark rights, considering that he holds a bona fide registration over the said domain. It may also be described as the use of legal or dispute resolution mechanisms to try to force people from their domains \cite{Sun2006DomainTrademarkConflict}.  An RDNH was claimed in a UDRP action against "groovle.com," in which the domain was purported to be too close to Google's trademark. However, since the domain was used for another search engine, it was deemed legitimately used and did not violate Google's trademark or be registered in bad faith \cite{Singh2011ReverseDomainHijacking}.
\end{itemize}


\subsection {Homograph attacks} 


The risk of continued homograph attacks is in using characters that look similar and thus mimic trusted domain names, such as using a lowercase "l" (el) to look like an uppercase "I" (eye) in the name "paypal.com" vs. "paypaI.com". The introduction of International Domain Names has only expanded the scope of such attacks, yet their prevalence is less. Still, the growing trend of more and more phishing attacks and how easy it is to fool users by taking them to fake sites require eternal vigilance. As a result of a new study, "Cutting through the confusion" \cite{holgers2006homograph}, it is going to reveal the scale and potential threat to homograph attacks, such as visual similarity between characters from different scripts, like Cyrillic or Greek, which is displayed in punycode in browsers, when attackers register domains that are visually similar to legitimate ones. This can be summarised in the table below, showing possible and actual registrations of such deceptive domains; hence, bringing out the difference between potential and actual use of the deceptive domain. For example, the domain "yahoo.com" had more than 5000 potential homograph variants, with only two actually registered, and "google.com" had a thousand possibilities, with four actual registrations.This is indicative of the importance of developing early preventive development and raising awareness of the risk from homograph attacks, as it sets into a very necessary chain of understanding mechanics and scope of homographs prevalence. Figure \ref{fig:figureAlot1} clearly shows, by means of a graphic illustration, the scope and scale of homograph attacks, which point to the potential risks that these attacks could pose to online security and the awareness and mitigation strategies that need to be implemented to protect Internet users from such deceptive practices. 

\captionsetup{font= footnotesize}
\begin{figure}[H]
    \centering
    \includegraphics[width=1\linewidth]{evaluation/confusable.png}
    
    \caption{confusables registered for popular domains, adapted from \cite{HomographAttacks}.}
    \label{fig:figureAlot1}
\end{figure}

\subsection{ Real-life Mitigations}

The following scenarios are examples of real-life confusable domain mitigations :

\begin{itemize}
    \item \textbf{Cloudflare's Zero Trust Services Approach :} The Cloudflare Zero Trust Services stops the domain used to mimic the real domain by equipping businesses with anti-phishing protection through Cloudflare Gateway. This protects corporate networks from phishing, using the trust element of well-known brands \cite{Cloudflare2023}. This is initiated in the system when making the very first query to any domain through a DNS resolver of 1.1.1.1, which in turn initiates a fuzzy matching protocol for analysis and comparison with a database of potential phishing domains. This should issue alerts on domains that are similar to those of legitimate brands, hence easily detecting them promptly and for quick response. Cloudflare enables monitoring 24/7 in real time with historical analysis, offering security teams the alert of domain matching certain patterns, hence being suspected, for fast review and action if the domain has shown up within 30 days. Cloudflare further supports with criteria to specify the corresponding investigation in .json, based on given domains or patterns, and may be subject to security risks.

     \item \textbf{IDN Handling of Google Chrome : }Google Chrome enforces an IDN (Internationalised Domain Names) policy to determine in which unicode or punycode form a domain label should be displayed. The domain label is tested to determine whether it has mixed script, invisible characters, or visually confusable characters, and whether it is actually validly converted to Unicode. For instance, domains containing characters of different scripts, or those that are clearly identified as mixed-script confusables, will be displayed in punycode, warning the users of potential deceptions. Chrome also offers comprehensive warnings for secure URLs that appear to be an imitation of already known Web pages \cite{ChromiumIDN}.
     
\end{itemize}

\subsection{Techniques for Mitigating Confusable Domains}

Mitigating confusable domains requires sophisticated techniques tailored to address the unique challenges presented by both non-Internationalised Domain Names (non-IDNs) and Internationalised Domain Names (IDNs). The threat of these two groups is vastly different and the technical possibility of most mitigation strategies also varies greatly. The subsequent section of the paper describes the mitigation methods in detail, addressing their operational feasibility and potential collaboration initiatives.

Non-IDNs Mitigation Techniques: These techniques aim to detect and mitigate domain squatting and typosquatting, where the attacker registers a typographical error or variant of a legitimate domain so that users get confused.

\begin{enumerate}
  \item Registry-Level Measures: Domain registries can implement checks to prevent the registration of domains similar to existing trademarks or brand names, using algorithms to detect variations and misspellings closely similar to protected names \cite{WTR2020}.
  \item Trademark Protection Programmes: Trademark Clearinghouse (TMCH) offers mechanisms for trademark owners to protect their rights by receiving notifications when someone attempts to register a domain that matches their trademark \cite{ICANNTMCH}.
  \item Automated Monitoring and Reporting: Automated systems can continuously monitor domain registrations for names that closely resemble known trademarks or brand names, allowing rapid detection and legal action against infringers \cite{TMCH2023}.
\end{enumerate}

IDNs Mitigation Techniques: The problem with IDNs is the potential for homograph attacks, where attackers can use characters from different scripts that visually appear to look like characters in Latin script.

\begin{enumerate}
  \item Punycode Awareness and Monitoring: Web browsers and security tools convert IDNs to punycode, a representation that encodes Unicode characters in ASCII. Awareness of punycode and monitoring for suspicious registrations can help identify potential homograph domains \cite{SOCRadar2023}.
  \item Browser-Level Defences: Modern web browsers have implemented defences against IDN homograph attacks by displaying the punycode version of the domain or alerting users when a domain name contains characters from multiple scripts \cite{Malwarebytes2017}.
  \item Collaborative Blacklisting and Sharing of Threat Intelligence: Organisations can collaborate to share information on known malicious IDNs, contributing to comprehensive blacklists that can be used by registrars, DNS providers, and end users to block access to malicious sites \cite{CyberThreatAlliance2023}.
  
\end{enumerate}


\subsection{ Transparency in Mitigation Efforts}

The element of transparency in dealing with confusable domains will be a great support in protecting the Internet from malicious activities such as phishing and trademark infringements. This encompasses a set of practices by domain registries and registrars to identify and publicise those domains that could mislead due to their similarity to legitimate ones. It contains means of transparency, such as publishing lists of those domains to alert the community about possible threats and taking secure measures, if possible.

\begin{itemize}
  \item \textbf{Cloudflare's Zero Trust Services Approach: }Cloudflare's process for identifying and blocking confusable domains should be transparent to its users. This includes detailing the criteria for flagging domains as phishing sites and the mechanisms in place for users to appeal or request a review of blocked domains. By openly sharing the methodology behind their zero-trust rules and how they are applied through the Cloudflare Gateway, trust in Cloudflare's protective measures is bolstered among corporate networks.
  
  
  \item \textbf{IDN Handling of Google Chrome:} Transparency from the side of Google Chrome for the display of domain names helps the user understand the risks to their security. How policy could be properly executed, reported, or suggested for changes by the community of users to increase internet safety will also be explained.

  \item \textbf{Typo-squatting Detection Tools: }The similar methodology should be evident in how similar domain names are detected by tools such as DNStwist or URLCrazy and how it could be detected and shared in a proactive security setup, which could also help others in any organisation.
  
  \item \textbf{Collaborative Efforts and Intelligence Sharing: }The partnership between cybersecurity entities and domain registrars, as well as initiatives such as the Anti-Phishing Working Group (APWG), should prioritise transparency in their operations. This includes the sharing of methodologies for threat detection, the criteria for taking action against malicious domains, and the processes for stakeholders to contribute or access shared intelligence. Transparency in these collaborative efforts ensures that actions taken against confusable domains are fair, understood by all parties involved, and supported by a broad community of internet security stakeholders.
  
 \item \textbf{Transparency for non-IDN registries : } 

 \begin{enumerate}
  \item Registry-Level Measures: Transparency in level-registry measures becomes a necessity if trust has to be kept between registrants and domain trademark owners. They are published criteria and algorithms used to find variations and misspellings of names submitted for protection. Making these publicly available can then ensure fairness, and feedback in detecting mechanisms is therefore paved for improving them.
  \item Trademark Protection Programmes: Communication is openly clear about all operations; this includes verification and notification. The guidelines make it easier to understand the rights and measures to protect your brand.
  \item Automated Monitoring and Reporting: set by the criteria and thresholds for informing the brand owners about the protection level for their trademark, and thereby will enable improved monitoring.
  
\end{enumerate}
 
 \item \textbf{Transparency for IDN registries  : } 

  \begin{enumerate}
  \item Monitoring and Identifying Measures for Suspicious Punycode Registrations: All domain registrars and trademark owners, together with security professionals, must adhere to measures on suspicious punycode registrations. Publicising the details of activities carried out to monitor them propagates homographic threats through collective ideas, also in their identification and mitigation.
  
  \item Browser-Level Defences: Web browsers have an important role in the defence of homograph attacks. They have to explain clearly to the user their defence mechanisms, such as ways of displaying punycode domains and ways of raising warnings so that they can be understood and trusted.
  
  \item Collaborative Blacklisting and Shared Threat Intelligence: Threat intelligence should be collaborative, based on a set of criteria, to blacklist domains. Clear rules on how to submit, validate, and remove data can also enhance the fairness and trust in collaborative security efforts.
\end{enumerate}
  
  
\end{itemize}

In summary, transparency in all these mitigation techniques not only builds trust between users, developers, and organisations, but also enhances the collective ability to respond to and prevent threats posed by confusable domains.


\subsection{ Analysis : Feasibility \& Practical Challenges}

 \begin{enumerate}
  \item Automated Monitoring and Reporting: Feasible; Technology exists to automate monitoring, even though the refinement of algorithms to decrease false positives and negatives from human review can probably not be undertaken with existing resources.
  \item  Punycode Registration Monitoring: Feasible; It will mainly require the use of existing technology and cooperation that could be initiated with little difficulty between stakeholders.
   \item Cloudflare's Zero Trust Services Approach: Feasible; since well-architected infrastructure and broad adoption have made Cloudflare zero-trust rules simple and effective to deploy, with a balance of security and operational efficiency without seismic root and branch changes.
  \item IDN Handling of Google Chrome and Browser-Level Defences: Feasible; Given that Chrome today has an enormous user base and that the groundwork for stopping homograph attacks already exists, it stands to reason that a solution is reasonably possible, meaning not too difficult, within a set timeline, and within the lifespan of any other typical software product.
  \item Blacklisting and Threat Intelligence Sharing: Moderately Feasible; Since agreement could be reached on shared platforms and protocols, but they imply strong cooperation and trust among such diverse entities, which is unlikely to be developed fast.
  \item Trademark Protection Programmes: Moderately Feasible; They are well-functioning processes under such adequate structures like TMCH and can be learnt while proceeding with experience, but likely to face legal and operational issues.
  \item  Browser-Level Defences: Not Feasible; While this is technically feasible, it seems rather infeasible soon that user practices will become uniform across all web browsers and that all users will be well trained in various security practices.
  \item Registry-Level Measures: Not Feasible; this would require very heavy coordination and agreement on standards across diverse jurisdictions and registries.
 

  
\end{enumerate}

\section{Phishing}

\subsection{Real-life examples}

\begin{enumerate}
    \item InterMed and Spectrum Healthcare Partners fell for a major phishing attack on 44,000 patient data. The InterMed breach involved clinical information for 33,000 patients, specifically names, birthdates, insurance, and some social security numbers, from 4 to 10 September. In another case, Central Maine Orthopaedics is reported to have breached 11,308 of their patient record files by unauthorised access to emails that contain personal and clinical details. It is such an incident that really makes it very paramount to strengthen email security and at the same time to provide professional training on data protection \cite{HIPAAJournal2020Phishing} .
    \item Google and Facebook were almost fooled by a group of phishers into a \$100 million sophisticated scam in which they were imitating legitimate invoices from the suppliers. The case is one among many that have caused the vulnerability of technology firms to social engineering and the need for reinforced security, employee training, and verification processes to combat ever-changing cyber threats \cite{CNBC2019Phishing}.
\end{enumerate}

\subsection{Real-life Mitigations} 

\begin{itemize}
    
     \item \textbf{Comprehensive Security Measures :} LaptopMD points out that the risk of ignorant searches requires the formulation of policies that make it difficult to land on some sites. In addition to this, the awareness of phisher techniques and browsing issues by employees will greatly save them from being caught in cases of phishing \cite{DigitalGuardianPhishingPrevention}.
     
     \item \textbf{Technological \& Human Factors :} Combining technology with awareness, SecureHIM advises that both should be combined by any organisation to include spam filters and two-factor authentications with the vigilance of employees to detect and eliminate the risks of phishing \cite{DigitalGuardianPhishingPrevention}.
     
     \item \textbf{Awareness against unsolicited emails :} The Centre for Democracy and Technology outlines the training that should be provided to avoid activities such as phishing, including the need not to respond to unsolicited emails even when suspecting anything fishy \cite{CDTPhishingMitigation}.
     
     
\end{itemize}

\subsection{Techniques for Mitigating Phishing}

Current phishing attack mitigation techniques focus mainly on preventing phishing
emails from reaching users' inboxes and discouraging users from accessing
phishing websites \cite{Suzuki2021Phishing}.

\begin{enumerate}
    \item Email filters: It uses algorithms that filter phishing emails, based on the reputation of the sender, the embedding of the link, and suspicious keywords, so that these emails cannot reach the inbox.
    \item Domain blocking: Take steps to block access from within an organisation's network to known phishing sites so that the organisation's users do not stumble on them accidentally.
    \item User Training: Train users on how to recognise signs from phishing emails and the risk associated with clicking on unknown links or sharing personal and sensitive information.
\end{enumerate}


The idea of a detailed thinking process of the offender, along with the description of the attributes in the environment that allow the attack to occur, is introduced with the Situational Crime Prevention Approach \cite{Suzuki2021Phishing}: This method was developed considering the theory that it is possible to deter potential attackers if the level of effort they make, the risk they take, and the likely rewards they receive are raised, stay the same, and lowered accordingly. It is worth mentioning that the criminal perspective is necessary to understand, and creating a hostile environment for phishing operations by implementing certain strategic preventive measures is crucial. This method includes the following steps:

 

\begin{enumerate}
    \item Increasing the Effort for Attackers: Implement strong authentication mechanisms and encryption to increase the difficulty of accessing or spoofing phishing websites or legitimate email accounts.
    \item Clarifying User Responsibilities: Information about users' role in security, such as awareness of phishing signs and reporting aids.
    \item Enhancing Detection Probability: Using the latest detection technologies and threat intelligence to recognise and eliminate phishing threats on time.
    \item Limiting Phishers' Access: Limiting the breadth of information that is accessible to the public, which might be used to construct compelling phishing emails and fake the identity of someone else or an organisation.
    \item Discouraging Future Attacks: Implement punitive actions such as tracking down familiar attackers and sharing information about the attack with a larger group of people to deter repeat offenders.
\end{enumerate}

This measure is designed not only to stop a phishing attack, but rather to create an environment that would lead to the cost-benefit ratio for phishing not so appealing to the attackers. Comprehensive perspectives on addressing phish through the three methods above singularly go to dramatically lower the vulnerability of organisations and individual persons to such acts.

In addition, Phishlimiter \cite{Chin2018PhishLimiter} , which is a new phishing detection and mitigation approach using Software-Defined Networking where it first proposes a new technique for deep packet inspection (DPI) and then leverages it with software-defined networking (SDN) to identify phishing activities through email and web-based communication. This is how it works:

\begin{enumerate}
    \item Deep Packet Inspection (DPI): Examines the network packet data more than the basic header information. Used to look at the content of packets looking for known signatures and patterns associated with phishing.
    \item Store and Forward (SF) and Forward and Inspect (FI) modes: SF mode temporarily stores packets for a thorough inspection before forwarding, while FI mode prioritises immediate forwarding with a parallel inspection to reduce latency.
    \item Artificial Neural Network (ANN): A machine learning model used to classify network traffic based on characteristics to detect links to potential phishing signatures.
    \item Dynamic adjustment of network flows: In the case of standard recognition, the system can dynamically change the routing to bypass the link or reduce flow to prevent the phishing process.
    \item Minimal disruption to network services: Designed to maintain the mitigation process minimised without targeting performance to ensure that final services would run smoothly even during the measure.
\end{enumerate}

\subsection{Transparency in Mitigation Efforts}

Here is how transparency can be applied to each of the mitigation techniques described:
\begin{itemize}
    \item \textbf{Employee Awareness and Training :} 
    
    Communication: This will consist of clearly informing the employees about the kind of threat and what it could mean for the organisation and their role in these defences.
    
    Accessibility: Make people aware that the repository exists, or make the resources easily available for reference. 

     \item \textbf{Comprehensive Security Measures: }
     
     Policy Publishing: All available policies, especially those related to web browsing, email attachments, and the use of security tools, will be published openly to let employees know about them.
     

    Changes and Updates: Introduce the workforce to changes relating to security measures and how such changes are beneficial and serve as a cover against new hazards.

     \item \textbf{Technological \& Human Factors: } 
     
     Tool Transparency: Clearly state the tool and the reason for its being in place for security (e.g., spam filters, two-factor authentication), and its work on subduing phishing. 
     
    User Control and Visibility: Attempt to give users some form of control or visibility over the security tools through which their work could be affected. Feedback from a blocked phishing attempt could, for example, help to reinforce the training.

     \item \textbf{Awareness against unsolicited emails:  } 
     
     Open Communication on Threats: Constant updates on new phishing techniques and any other notable attacks are discussed among the industry to be updated.
     
    Best practices: Develop best practices for easy identification and to be visible on how to catch and react to phishing attacks, with graphic examples or checklists.

    \item \textbf{Email Filters:} 

    The effectiveness of email filtering technology in mitigating phishing attempts is enhanced by transparency in its operational parameters. This helps to make the user understand, starting from analysing the reputation of the sender to the various steps related to decision making of phishing keywords. Continuous improvement builds trust, and perhaps some community members might even wish to provide feedback on how to improve the performance of the filters or report inaccuracies to the filter system regarding combat against the threats of phishing.

    \item \textbf{Domain Blocking:} 

    Such measures may include transparency in the criteria of the blacklisting and regular updates in relation to the access to the known phishing sites inside the organisation's network. This implies also setting a clear means of reporting unlisted phishing sites and correcting false positives by stakeholders. 

    \item \textbf{Situational Crime Prevention Approach (SCP):}

    The major advantage of the SCP approach is the clear disclosure of both the applied methodology and the results obtained. This is through the explanation of the analysis of the criminal's thought and environmental factors aiding phishing, whereby the stakeholders are enlightened, therefore, they make efforts to reduce it. 

    \item \textbf{Phishlimiter:}

    The phishing detection system, such as the DPI integrated with SDN, has the ability to make its operation transparent, may increase user confidence, and preserve system functionality. It could emphasise the reliability and credibility of such a system if the criteria and algorithms by which the system determines a potential phishing attack are clearly spelt out. 

\end{itemize}

\subsection{ Analysis : Feasibility \& Practical Challenges }

\begin{enumerate}
    
    \item Comprehensive Security Measures :Feasible; The deployment of web filters and secure browsing policies is technically straightforward with existing technology. The main effort lies in the continuous update of policies and employee education.
    \item Technological and Human Factors: Feasible; The integration of spam filters, two-factor authentication, and secure browsing add-ons is readily achievable with current technology. The human element, continuous employee vigilance, enhances the effectiveness of these tools without significant additional costs.
    \item Awareness against unsolicited emails: Feasible; Establishing and communicating best practices for handling suspicious emails involves minimal costs and leverages existing communication channels within organisations.

     \item User Training: Feasible; The training of the user's awareness on phishing is practical and beneficial, as it allows giving room for the user to measure the feedback on the effectiveness of training and to give suggestions for improvements that can enhance programme accessibility and user participation.
     
     \item Situational Crime Prevention Approach (SCP): Feasible; sharing information that identifies how an offender behaves and the environment that helps him/her attack. Although presenting this success story is of great value, great care must be taken in the handling of the detailed analyses of criminal tactics to avoid misuse. Community feedback will allow for further development.
     
     \item Domain Blocking: Moderately Feasible; Updating blacklists and dealing with the false positives, which have to be dealt with. This is a mammoth task, especially for relatively smaller organisations with few resources at their disposal. The process demands balance in responding very accurately within a very short time, which can over-stretch resources.
     
    \item Email Filters: Not Feasible;  Describing the general criteria and algorithms for email filtering is possible, but full disclosure risks security by enabling attackers to circumvent these measures. Partial transparency can be achieved without compromising the integrity of the system.
    
    \item Phishlimiter: Not Feasible; The complexity and proprietary nature of technologies like DPI and SDN make full disclosure of Phishlimiter's operations impractical. Detailed investigation of operations could compromise security. Keeping up with evolving phishing tactics requires continuous updates, which may not always be promptly disclosed to avoid aiding adversaries.
    
\end{enumerate}


\section{Collaboration Among Registrars, Registries, and DNS Collaborators}

This collaboration should be achieved with the DNS registry, the registry, and the collaborators. In that way, they can boost common resources and intelligence that can guide making the Internet more secure and resilient. This strictly falls within the remit of registries and registrars acting in collaboration to put in place such stringent registration policy with procedures for verification, checking against mimicking existing trademarks or even popular domain names. In this way, the collaboration can even manifest itself through the sharing of sensitive data with regard to domain abuse threats and trends. Databases and threat intelligence platforms are shared amongst stakeholders, allowing them to anticipate and avert most such perils well before they impact netizens. This collective effort will enable the formulation of standards by which to coordinate responses to confusable domain incident reports. Mitigating confusable domains and phishing requires that registrars, registries, and DNS collaborators work together in a common effort. This is due to the increasing level of threats and the shared responsibility of all actors involved in the DNS ecosystem \cite{Catania2022}. To put this into perspective, here are some examples: 


\begin{enumerate}
  \item New specifications on defining DNS abuse have been entered into ICANN’s contracts from ICANN’s contracted parties. Furthermore, there are clear requirements that define the actions to be taken by registry and registry after receiving immediate actionable evidence of abuse. This move clarifies the roles that different stakeholders can play in addressing the DNS abuse issue and establishing a common approach to redress \cite{Weinstein2023}.
  \item Some of these new duties have been positively approved by the community. The community supported the new obligations of the ICANN contract parties to further mitigate DNS abuse. The message that this example sends to everyone is that the community is willing to join and participate in DNS abuse and other challenges to address \cite{ICANN2023}.
  \item  Efforts such as NetBeacon, with the support of the DNS Abuse Institute, are being rolled out to reduce friction in reporting and mitigating DNS abuse. This service solves the current complexities and quality standards associated with the reporting of DNS abuse, as it makes the work easier for registrars, ultimately narrowing down their scope to the relevant and evidenced report, and underlines the need for cooperation among registrars, registries, and other DNS stakeholders. This is what is capable of saving the Internet and, at the same time, protecting the credibility and confidence of DNS \cite{NetBeacon}.
  
\end{enumerate}

Real-life examples of entities seeking to block the resolution of DNS names used by bad actors for phishing and other malicious activities, especially in connection with public recursive DNS servers, frequently revolve around matters of control, filtering, or securing internet traffic with various kinds of motivation corresponding to such sectors. Consider the following:

\begin{enumerate}
    \item Governmental Efforts to Block DNS Resolutions: Governments may interfere directly with DNS operations to enforce some censorship or block access to particular types of content. For instance, China uses the Great Firewall for regulation of access to the World Wide Web within their territory, including doing some DNS mismanagement to block unwanted content \cite{XuAlbert2017MediaCensorship}.
    \item Corporate and ISP DNS Filtering: DNS filtering can be deployed by companies and even ISPs in a bid to achieve enhanced online security. For instance, Heimdal Security explicates how DNS filtering works as one of the measures to prevent their access to various harmful or inappropriate websites since it first checks the requests for domains. If some areflagged, access is denied, hence maintaining both security and productivity within one's organisation. This approach is very effective for the prevention of phishing and malware attacks because it stops DNS requests towards malicious sites \cite{
HeimdalDNSSecurity2023}.
    \item Ad Block DNS Services : Cloudflare discusses how DNS filtering can be used to prevent access to malicious sites and also filter what is harmful or unfit for viewing. This is done at the DNS level to prevent these sites from loading on devices. Cloudflare uses its DNS to filter part of a more prominent access control policy, which is an effort to secure company data and govern what employees will see on the network they manage \cite{CloudflareDNSFiltering2023}.   
\end{enumerate}

 On the negative side, attackers are taking advantage of DNS blocking mechanisms to carry out DNS-based attacks. These include using DGAs (Domain Generation Algorithms) for malware communication, using FastFlux techniques for slip-streaming attacks, basically creating malicious newly registered domains (NRDs) that appear benign and legitimate to an outside observer, etc. All this makes it difficult to block bad content at the DNS level, which calls for quite sophisticated countermeasures.


\section{Benefits of Transparency }


Transparency has numerous advantages when it comes to handling confusable domains and mitigating phishing. First, it encourages domain registry owners and registrars to be more accountable to each other by motivating them to take an active role in the identification and removal of confusable domains and phishing websites. Second, openness discourages bad actors who might otherwise take advantage of the anonymity provided by a lack of public monitoring. Third, by making these lists available to the public, registries and registrars enable companies and trademark owners to promptly take precautionary measures to safeguard their brands, including acquiring domain names or pursuing legal action. Transparency also facilitates community-based mitigation initiatives, in which researchers studying cybersecurity and the broader community work together to detect and eliminate dangers. This coordinated effort not only tackles confusable domains, but also considerably impedes phishing attempts by revealing, and thus reducing, the strategies employed by bad actors. The effectiveness of these tactics is significantly increased by using the collective expertise and attention to detail of the cybersecurity community, resulting in a more secure online environment for all parties involved.


\section{Drawbacks and Security Concerns} 

At the same time, the issue of publishing confusable domain lists, while ultimately beneficial for cybersecurity, also implies several limitations and security issues, such as the problem of phishing. First, the reason for concern is that publishing these lists can serve as a manual for bad actors since it potentially discloses potential domains for phishing. For example, if victims are exposed in such a way, bad actors can quickly adjust their approaches, ensuring that their activities stay one step ahead of countermeasures. Second, false positives, or legitimate domains that are mistakenly identified as confusable, also represent a serious challenge. For real businesses and people, a loophole for that is their potential association with phishing since this can entail unnecessary attention, legal action, and reputation damage. Third, the issue of transparency also raises concern about how relevant releasing such data is in terms of attack prevention. While the logic of making the lists available to the public before they can be misused for abuse, such as phishing, is clear, the sheer volume of domain registration and the continuous changes in domain abuse tactics undermine their actual usefulness for end-users and organisations to proactively identify and tackle phishing risks.



\section{Limitations of Research Conducted on DNS Abuse Transparency }

\begin{enumerate}
    \item Variability in reporting standards: The majority of challenges faced in actual practice come with a lack of uniformity in standards from DNS infrastructure providers from the definition of abuse to the thresholds of actions and how those actions are being reported. This inconsistency has made efforts to collect and compare the data of different entities difficult to piece together into one coherent picture for sensible enforcement of DNS abuse mitigation. 

    \item Limited Availability of Data: The general lack of transparency reports that are available to the public. Providers either do not at all release or do, and in doing so, have an omission of information required to be looked into. This means that there are still gaps in understanding the whole spectrum of strategies deployed by the DNS abuse domain due to data unavailability. 
    
    \item Reluctance to Share Sensitive Information: Information provided by respondents about abuses and what they do to mitigate them. In general, concerns about privacy, security, and the potential of revealing vulnerabilities to bad actors contribute to this reluctance, which leads to Limiting the capacity of researchers to perform a comprehensive analysis of DNS abuse mitigation strategies.
    
    \item Dynamic Nature of DNS Abuse: The evolving tactics employed by those who abuse DNS are constantly changing, so results can be out of date very fast. It is more difficult to create best practices that are applicable and efficient over time due to this quick change. Because DNS abuse is dynamic, it requires ongoing research and strategy adaptation to stay ahead of new threats.

    \item Potential Bias in Self-Reported Data: Self-reporting in transparency reports can still bias the results. Organisations tend to emphasise their achievements while downplaying their shortcomings or difficulties. Due to this biased reporting, opinions about how well DNS abuse is being controlled can be distorted, which could cause mitigation efforts to be overestimated.
    
    \item Complexity of Measuring Impact: Due to the specific nature of the Internet ecosystem, it is quite challenging to evaluate the efficacy of DNS abuse mitigation techniques. Furthermore, the evaluation process is quite complex due to the indirect impact of specific actions of DNS abuse prevention efforts on the larger effects.

    \item International and Jurisdictional Challenges: Due to the international scope of the Internet, different legal and regulatory frameworks in different jurisdictions have an impact on DNS abuse and how it is mitigated. These differences highlight the need for cross-border cooperation and harmonisation by adding complications to the implementation and evaluation of transparent practices on an international scale.

    \item Ethical and Privacy Considerations: Ethical issues related to data collection and analysis that may include sensitive or personally identifiable information must be addressed in research in this field. Respecting ethical and privacy standards is very important, but it can also restrict the available research approaches, further limiting the breadth and depth of the study.
    
  
\end{enumerate}

As demonstrated, all these limitations show the complex difficulties in conducting a thorough research on DNS abuse mitigation transparency. Only by operating with the principles of team interaction, originality, and a willingness to develop research tools and approaches in the given area, can the misrepresentation difficulties be resolved.

\section{How well did the project meet the objectives?}

In evaluating the success and impact of the research project on DNS Abuse Transparency, a key question in assessing the achievements and influence of the study on DNS Abuse Transparency is addressed. To what extent did the project fulfil its original objectives? This section seeks to systematically evaluate the project's accomplishments in relation to its objectives, taking into account the intricate domain of DNS abuse and the difficulties associated with improving transparency and management procedures. A thorough review is provided by looking at stakeholder participation, the contribution to understanding DNS abuse, the objective achievement, and the practical consequences of the results. Shortcomings are acknowledged and recommendations for further research and development are made, recognising both the successes and the areas that still require improvement. This reflection not only demonstrates the progress gained but also the continuous path toward a DNS ecosystem that is more open, safe, and resistant to abuse.


\subsection{Objective Fulfilment}

The objective of these projects was to increase the understanding and transparency of stakeholders about reporting DNS abuse. Despite many challenges that the project faced, such as working with different reporting standards or limited access to data, this effort managed to uncover light flaws in the approach to mitigating and reporting DNS abuse. This project also demonstrated the high level of complexity and variety of approaches taken by different institutions in reporting, which accentuated the importance of unified and mandatory reporting requirements.

\subsection{Impact on Understanding DNS Abuse}

Given the difficulties, the research project yielded valuable information on the state of mitigation of DNS abuse. It showed that DNS abuse is unpredictable and, due to rapidly changing tactics of bad actors, mitigation measures should also be updated regularly. The study revealed several gaps in current knowledge and approaches to address, as shown by the lack of transparency reports and the unwillingness of providers to share critical data on their scopes. These findings could lay the ground for further research and policy-making.

\subsection{Stakeholder Engagement}

It was important to include interaction with stakeholders, including registry owners, policy makers, and DNS registry. The project encouraged discussion of the need for increased sector and jurisdiction cooperation and transparency. However, it seems that there is more to be desired in the impact on stakeholder behaviours and policies: for example, more proactive efforts and collaboration in the fight against DNS abuse.

\subsection{Practical Implications}
The outcomes of the project lead to positive results in the impact of transparency in reducing DNS abuse. Implementing recommendations for setting up standardised report creation processes and facilitation of data exchange could lead to more coherent and effective DNS abuse mitigation efforts. These recommendations provide practical next steps for stakeholders to better address the issues raised.

\subsection{Suggestions for Improvement}


Future projects, therefore, can discuss study questions on newly developed strategies, which abuse DNS, and research more aspects of transparency, so that they can obtain deeper insights. Further improvements in how to engage stakeholders include more transparency in forums to work with them toward the establishment of joint research projects, thereby increasing the scope and quality of information. Furthermore, promoting the idea of the project could lead to more noticeable changes in practice and policy.


\subsection{Future Vision}

This research project embarked on an extensive effort to clarify the complexity of the transparency of DNS abuse. Taking into account obstacles such as the dynamic nature of abuse methods and the availability of data, the effort achieved significant achievements in highlighting important areas for development and setting the stage for future breakthroughs. This development would represent a significant step forward in creating greater transparency and consistency in efforts to mitigate DNS abuse in recognition of the work being done and research and cooperation that will continue. Consequently, the way forward demands that all actors work together toward the recommendations for having a more secure, safe online environment.

\chapter{Conclusion}

\section{Brief Review}

This project has looked at the abuse of the DNS, in which a situation leads to malicious actors using domain names for their malicious activities, such as phishing. DNS infrastructure providers, including registries and registrars, have thus focused on playing a role in controlling, if not reducing, this abuse. The research included complaints received from providers and steps taken to abuse by deleting or blocking name registrations. Therefore, the core of the study lies in the concept of transparency and the degree to which providers disclose and document these actions. Obviously, the issuing of comprehensive transparency reports does not go as far as the requirement, and it is a very key aspect towards promoting trust and accountability in the digital world. 


\section{Main Results}

\subsection{Related back to Project Objectives:}

This project led to the serious transparency gaps in the mitigation of DNS abuse adopted by infrastructure providers. Mitigation action against DNS abuse includes technical measures, such as the traceability of responsible individuals and strict reporting and communication of mitigation action. This goes parallel with our first objective: to understand how the area practices on transparency and, when necessary, establish a requirement for standardised measures.

\subsection{Summary of Proposals:}

Throughout the research project, several strategies have been discovered to improve transparency:


\begin{enumerate}

    \item Regular Transparency Reporting: Request the providers of DNS infrastructure to report on a regular basis with respect to actions undertaken in the mitigation of DNS abuse.

    \item Stakeholder Engagement: Increase cooperation and communication toward the regulation of transparency between DNS providers, users, and legislators.
.
    \item Public Accountability Mechanisms: Develop a means for members of the general public to monitor and evaluate DNS abuse mitigation efforts.
    
    \item Innovation in Defence Strategies and Sharing: Highlight the need to develop strategies to prepare for the fight against DNS abuse and share these strategies among participants to further promote innovation in their approaches to the fight against different types of abuses.
    
    \item Transparency in Monitoring and Collective Action: Open observation of DNS activity and the collaborative work of all parties involved in the DNS ecosystem to power a single strategy to stop abuse.
    
\end{enumerate}

These strategies seek to add to the basic measures outlined in this research project, considering the individual efforts of the providers of DNS infrastructure, but also the collective efforts and shared responsibilities throughout the DNS ecosystem. With these measures, definitely the internet community can help improve the level of trust and transparency, thus ensuring that a much more effective DNS system is in place securing against abuse.

\section{Future Work}

\subsection{Further Research Directions:} Future research needs to work towards integrating AI and machine learning in predictive DNS abuse detection, as well as the extent to which international regulatory frameworks manage to establish transparency requirements. This could be further researched into how the difference in degrees of openness and transparency changes user trust and behaviour, in addition to transparency, and the perception of infrastructure DNS providers. Furthermore, studies could assess how well different transparency techniques mitigate DNS abuse in the real world.

\subsection{Practical Next Steps for Developing Transparency Best Practices: }

\begin{enumerate}
    \item Framework Development: The development of a common transparency framework that can be applied by DNS infrastructure providers across the globe and work with the leading industry partners in this respect.

    \item Technology Solutions: Investigate technical possibilities for the automation of DNS abuse data collection and sharing in a more transparent manner.

    \item Policy Recommendations: Transparency in such activities should be a requirement for draughting policies and should encourage legislative support for DNS abuse mitigation initiatives.
    
    \item Stakeholder Collaboration: The regulatory bodies, which maintain the DNS infrastructure, and the cybersecurity communities come together to fight the battle and find an amicable solution to these challenges.
    
    \item Transparency Standardisation: Uniformity in the standardisation of reporting across the industry would help maintain uniformity in the level of disclosure of mitigation efforts for DNS abuse.
    
    \item Real-Time Monitoring: This ensures abuse is detected in time and threats are responded to in time, and real-time dashboards are used.
    
    \item  Public Awareness: User education on DNS security to improve public awareness and protect them from possible abuses.
\end{enumerate}


\subsection{Enhanced Transparency Practices for DNS Abuse Mitigation: }

Building on these initial steps, registries and registrars are urged to implement improved transparency measures such as the following to strengthen the DNS ecosystem's resistance to abuse:

\begin{enumerate}
    \item Public Reporting: Create a detailed and consistent transparency report in which we issue the number of abuse reports received, the steps to be taken, and the result of those steps. In addition, this transparency increases the confidence of users and makes the organisation responsible for efficient mitigation of abuse.

    \item Stakeholder Engagement: Provide forums or advisory committees to discuss and evaluate mitigation solutions for DNS abuse that involve a wide range of stakeholders, such as government representatives, cybersecurity professionals, and members of civil society. This guarantees that decision-making procedures take into account a wide range of points of view.

    \item Best Practice Sharing: Encourage a transparency environment by sharing best practices, resources, and innovations to mitigate DNS abuse with colleagues in the DNS ecosystem. 

    \item User Education: Recreate and distribute instructional materials to help domain owners and end users identify and mitigate DNSabuse. Empowering people with knowledge can drastically reduce the effectiveness of phishing and other abusive techniques.

    \item Automated Abuse Detection: Make use of AI and machine learning technology to automatically identify possible DNS abuse behaviours. Exchange anonymous indicators of compromise (IoCs) with reliable partners to increase the resilience of the ecosystem as a whole.

\end{enumerate}


\subsection{Future Directions in DNS Abuse Mitigation: }

Future studies and practical initiatives should focus on the following areas to better address the dynamic nature of DNS abuse and proactively counter new threats:

\begin{enumerate}
    \item Emerging Technologies: Research DNS, the possibility of abuse, and develop a targeted mitigation solution within AI-generated content and growth in IoT devices.

    \item AI and Machine Learning for Proactive Defence: Using data analysis to find possible abuse vectors and advancing AI and machine learning models to anticipate and handle DNS abuse before it happens.

    \item Enhanced IoT Security: Establishing security standards with the support of partnerships and laws for manufacturers of IoT devices to reduce the tendency to exploit the device in DNS abuse.

    \item Global Policy and Regulation Dialogue: Participate in discussions that deal with policies, coordinating mitigation for DNS abuse, and laws that can enhance security, privacy, and openness.

    \item Transparency Evolution: Advancing transparency standards in line with technology, emphasising real-time data sharing, blockchain-based log reporting, and user-friendly interfaces to prevent unauthorised access to data.

    
\end{enumerate}


\subsection{Contributions to Future Transparency Practices:}

This research further contributes to the ongoing development of best-practice for transparency in reporting on DNS abuse mitigation. This emphasises the importance of transparent and consistent reporting and interaction with stakeholders so that policy makers can rely on reporting for informed policy formulation and to be able to contribute to safer cyberspace. Therefore, this has been one of the main findings of this study in the remaining challenges of the field. The ways out include informed and active user bases, reduction of DNS abuse, and even more reliable Internet ecosystems.


\section{Reflection}
\subsection{Personal Learning: } 

This project really opened my eyes and taught me a lot about the complexity of DNS abuse and how this could pose a challenge in the development of systems due to the potential lack of transparency. DNS abuse can be viewed as dynamic, in the sense that it continues to evolve to show new tactics used by the bad actors, so that the mitigation strategies also need to be adaptive. And what I have learnt is that transparency in work is not only sharing information, but, in fact, it helps build trust within the community, increases efficiency of efforts aimed at abuse mitigation, and has an overall positive influence on governance of the Internet and security.

\subsection{Evaluation of Research Process:  }

In other words, this research project exposed serious challenges to the study of transparency in mitigating DNS abuse, from secrecy in sharing sensitive data, dreaded by privacy and security considerations, to the threat of data self-reporting biases. What it made clear was the fact that the process made it clear that a delicate balance needs to be achieved between the need for security and releasing just enough information to be, basically, transparent. However, these approaches allowed in-depth research to be carried out, and at the same time, they criticised the general area of what needs to be improved, including methods to identify more accurate ways by which transparency practices influence the reduction of DNS abuse.



\subsection{Perspective on Research Findings and Contributions: }

This research project provides a broader look at current practices and how effective they are within the ongoing discussions on how to mitigate and possibly even yield more effective transparency of DNS abuse. This research suggests that there should be an organised approach to openness from gap analysis and practical strategy recommendations. This calls for designing best-practice guidelines that strengthen cooperation among all the parties in the DNS ecosystem. Although very good progress is being made, my work highlights the continued need for attention and effort in this area, suggesting that the road toward a more transparent, safe, and abuse-resistant DNS landscape is very far from complete.


This project has broadly enriched my knowledge in relation to mitigation of DNS abuse and transparency pursued, while provoking very important information that should take this field into account in future research in the area of cybersecurity and Internet governance.





\chapter{Findings}
hello how are you 


% note that your supervisor may have a strong opinion on the style of referencing you use. Some background is available at https://www.overleaf.com/learn/latex/Bibtex_bibliography_styles
\bibliographystyle{IEEEtran} %Changed to IEEETran by HS
%\bibliographystyle{unsrt}
\nocite{*}
\bibliographystyle{plainnat}
\bibliography{bibs/sample}
\appendix
\renewcommand{\thechapter}{A\arabic{chapter}}
\chapter{Appendix}

\section{Detailed Transparency Report and DNS Abuse Mitigation by Cloudflare}
\label{app:cloudfare}


\begin{itemize}
    \item \textbf{Abuse Reports and Actions Taken}
    \begin{enumerate}
        \item Handling Abuse Reports: Among many other DNS abuses, phising, malware, and copyright infringements are most common at 
        \item Termination of Services: 
        \begin{itemize}
            \item Suspended Accounts and Dom: In the last half of 2022, Cloudflare claims to be committed to suspending 206 accounts and 530 domains that have proof to host content for Child Sexual Abuse Material (CSAM).
        \end{itemize}
        \item  Uniform Domain Name Dispute Resolution Policy (UDRP) Requests: Approximately 21 UDRP requests were handled in the latter half of 2022, which shows that Cloudflare is quite serious about resolving domain disputes amicably.
        \end{enumerate}
    \item \textbf{Law Enforcement and Legal Compliance}
    \begin{enumerate}
        \item Legal Sufficiency Review: Cloudflare will respond only to requests of this kind that meet a legal requirement or exemplary cases. In the sphere of law enforcement, there are court orders and subpoenas.
        \item International Privacy Laws: The company will not allow a state to demand a data reach if the legatees of this state contradict such an approach to privacy dictated by the outside state to Cloudflare. This policy highlights the adherence of Cloudflare to previous and subsequent points of the legal framework.
        \item Emergency Disclosure Requests: In such cases, the company agrees to some level of disclosure when there is a formal requirement for legal follow-up.
        \item National Security Requests: The company claims that it only served transparently and open agencies and other organisations. That is why it appeals against national security orders that do not adhere to the purpose of transparent informatics company performance.
        \item International Data Requests: Respond to foreign government requests on US legal standard cases or case evaluations.
    \end{enumerate}
    \item \textbf{Mitigation of DNS Abuse}
    \begin{enumerate}
        \item Public Reporting and Transparency: Cloudflare publicly reports and discloses these triggers of abuse, their kinds and quantity, to be able to maintain transparency in the relation of trust that allows ant-abuse to exist. 
        \item Law Enforcement Cooperation: Continue your partnerships with law enforcement, ensuring that everything you do is justified from a legal perspective, particularly with respect to DNS abuse.
        \item Challenges to Mitigating DNS Abuse: It is difficult to find a proper balance between the role of each side in defending legal interests and allowing collaborative measures to be accountable for DNS commitment. 
        \item Efficiency of efforts: Even with those complications, Cloudflare efficiently mitigates abuse by facilitating root cause solutions and market factor multitude.
    \end{enumerate}
    \item \textbf{Proposals for Future Enhancements}
    \begin{enumerate}
    \item Stakeholder Cooperation: Coordinated law enforcement in service delivery and other roles. Formal collaboration with international organisations and agencies. 
    \item Advances in Abuse Detection: The organisation has also developed plans to invest in advanced technologies and machine learning to improve abuse detection and response times. 
    \item Transparency Reporting: The organisation has also assured the community of its commitment to further increase the frequency and level of detail in transparency reports this year, describing more clearly the nature and mitigation of DNS abuse.
    \item User Education and Awareness: The organisation has also committed to developing and, more importantly, distributing educational materials aimed at increasing user awareness of the risks related to cybersecurity and DNS abuse. 
    \item Policy and Legal Reforms: Because there is likely to be a conflict between privacy laws and external privacy laws, the solution also suggested two law enforcement demands, proposing that participants engage in advocating changes in the resolution of the arising conflict. 
    \item Multi-stakeholder Feedback Mechanism: developed and proposed to the Executive Board for adoption and implementation, which outlines feedback mechanisms that shall include input from users, civil societies, and other stakeholders. This feedback shall form the organisational foundation for improvement and policy formulation.
    \end{enumerate}
        
\end{itemize}





\section{Presentation Slides}

\begin{figure}[H]
  \centering
  \begin{subfigure}[b]{0.55\linewidth}
    \includegraphics[width=\linewidth]{appendix/PRE1.png}
    \label{fig:left}
  \end{subfigure}
  \hfill % adds horizontal space between the figures
  \begin{subfigure}[b]{0.55\linewidth}
    \includegraphics[width=\linewidth]{appendix/PRE2.png}
    \label{fig:right}
  \end{subfigure}
  \label{fig:images}
\end{figure}

\begin{figure}[H]
  \centering
  \begin{subfigure}[b]{0.55\textwidth}
    \includegraphics[width=\textwidth]{appendix/pre3.png}
    \label{fig:left}
  \end{subfigure}
  \hfill % adds horizontal space between the figures
  \begin{subfigure}[b]{0.55\textwidth}
    \includegraphics[width=\textwidth]{appendix/pre4.png}
    \label{fig:right}
  \end{subfigure}
  \label{fig:images}
\end{figure}

\begin{figure}[H]
  \centering
  \begin{subfigure}[b]{0.55\textwidth}
    \includegraphics[width=\textwidth]{appendix/pre5.png}
    \label{fig:left}
  \end{subfigure}
  \hfill % adds horizontal space between the figures
  \begin{subfigure}[b]{0.55\textwidth}
    \includegraphics[width=\textwidth]{appendix/pre6.png}
    \label{fig:right}
  \end{subfigure}
  \label{fig:images}
\end{figure}

\begin{figure}[H]
  \centering
  \begin{subfigure}[b]{0.55\textwidth}
    \includegraphics[width=\textwidth]{appendix/pre7.png}
    \label{fig:left}
  \end{subfigure}
  \hfill % adds horizontal space between the figures
  \begin{subfigure}[b]{0.55\textwidth}
    \includegraphics[width=\textwidth]{appendix/pre8.png}
    \label{fig:right}
  \end{subfigure}
  \label{fig:images}
\end{figure}

\begin{figure}[H]
  \centering
  \begin{subfigure}[b]{0.55\textwidth}
    \includegraphics[width=\textwidth]{appendix/pre9.png}
    \label{fig:left}
  \end{subfigure}
  \hfill % adds horizontal space between the figures
  \begin{subfigure}[b]{0.55\textwidth}
    \includegraphics[width=\textwidth]{appendix/pre10.png}
    \label{fig:right}
  \end{subfigure}
  \label{fig:images}
\end{figure}

\begin{figure}[h]
    \centering
    \includegraphics[width=0.55\linewidth]{appendix/pre11.png}
    \label{fig:lol}
\end{figure}


\end{document}