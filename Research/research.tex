\chapter{Research Methodology}



A structured questionnaire was sent by email to various stakeholders in the DNS ecosystem. This method was chosen because of its convenience, compliance with participants' busy schedules, and permission for detailed responses at the respondents' will. The approach provided a means of collecting a wide range of observations on DNS abuse in terms of definition, the most prevalent types, mitigation challenges, and the theme of transparency. 



\section{Questionnaire Design and Distribution} 

The questionnaire had to take into account some of the issues in a multidimensional approach, giving great emphasis, but not limited, to address the nature of DNS abuse mitigation transparency.

The questions were specifically designed to extract in-depth information about:


 \begin{enumerate}
 \item The definition of DNS abuse. 
 
  \item The types of DNS abuse stakeholders are most frequently encountered, with the aim of identifying patterns and specific concerns within the ecosystem.
  
  \item The challenges and limitations faced in mitigating DNS abuse, seeking to understand the barriers to effective action.
  
  \item The mitigation strategies used, gathering information on the practical steps taken and their perceived effectiveness.
  
  \item The practice of publishing reports or data as a form of transparency, exploring the current state of openness in the field.
  
  \item The role of transparency in aiding or impeding DNS abuse mitigation efforts, probing the potential impacts of increased visibility.
  
  \item  The effects of transparency on the relationships between various DNS stakeholders, considering the broader implications for cooperation and trust.
\end{enumerate}


\section{Stakeholder Responses} 

The insights from the completed questionnaire of the different stakeholders reflect several key themes and insights into understanding DNS abuse, as well as the mitigation of this abuse which provide a full view of current practices and potential areas for improvement with respect to the DNS ecosystem. The key themes and insights include:



\begin{itemize}
  \item \textbf{Varied Definitions of DNS Abuse:} Although stakeholders largely accepted the definition that had been adopted by the ICANN Contract Parties, they also noted its shortcomings, especially in being too categorical and thus may leave out evolving types of abuse. It was considered that a more flexible way forward would be a robust framework to define the abuses to be mitigated at the domain name level.
  
  \item \textbf{Common Types of DNS Abuse:} They pointed out that phishing was the most common attack type, followed by malware, botnets, and spam. It was also pointed out that one of the most common problems was related to the challenge related to proving the number of spam-related domains.
  
  \item \textbf{Challenges in Mitigation:} Perhaps the most significant was the economic structure of the domain registration industry and its ability to mitigate malicious registrations without fundamentally altering it. Stakeholders clearly state that a significant difference between large registrars, generally considered good actors on the Internet, and smaller registrars with a higher level of DNS abuse underscores the different aspects of this problem within different industry segments.
  
  \item \textbf{Mitigation Strategies:} Responses include different strategies, from blocking orders from some regions to the use of software to monitor abusive activities. Recommendations were made in the area of education and outreach projects, such as NetBeacon and Compass, to report abuse and information on DNS abuse.
  
  \item \textbf{Role of Transparency:} Opinion on the transparency process was kind of mixed, since part of the respondents consider this positively because it is a tool that provides evidence to the industry in their fight against abuse and part of them consider negatively the way sensitive mitigation ways could be revealed. The consequence of transparency was elaborated on the development of relationships between all stakeholders, and there is, in general, a common understanding that transparency would increase understanding and teamwork through better communication on set measures against abuse.
  
\end{itemize}

Thus, the responses of stakeholders were quite valuable in providing a snapshot of practical challenges and possible mitigation strategies for DNS abuse. Consequently, it was a very valuable presentation of a real-world view of the need for the world to have adaptive definitions and comprehensive mitigation strategies, along with transparency in the ecosystem. This further extrapolates the theoretical knowledge with the help of the experiences of people involved on an active basis in the mitigation of DNS abuse.

\section{Types of DNS Abuse Encountered} 

The stakeholder responses described the most common form of DNS abuse assumed within their area of interest. From these, views emerge for a variety of threats, where each of them brings unique challenges that favour specific mitigation measures.

\begin{enumerate}
    \item Phishing: Stakeholders identified it as the most prevalent form of DNS abuse and the most visible. In fact, the total number of phishing incidents observed through tools such as NetBeacon and tracked by Compass is a stark and singular metric of just how big and urgent the problem has become in the wider DNS domain.
    \item Malware and Botnets: These also included malware and botnets, i.e. multifaceted DNS abuses. Such abuses not only compromise the integrity of systems, but also present a security hazard to users and infrastructures, in general.
    \item Spam: It is now recognised as widespread, and stakeholders have pointed out the challenges of quantifying and appropriately addressing the relevant spam-related domains. Therefore, it makes spam elusive for existing mitigation efforts that raise the bar in the pursuit of next-generation detection and response mechanisms.
    \item Compromised CMS: Common encounters with the content management system (CMS) compromised. Consequently, attacks remain possible where some other existing vulnerabilities on web platforms are concerned. This kind of abuse is the reason why strong web security control practices should be in place and why platform operators must be alert.
    \item "Water Torture" Attacks: Known as random subdomain attacks, they are even more technical and sophisticated ones. Such attacks not only compromise the regular functioning of DNS but also take some of the most sophisticated countermeasures to mitigate its impact on the affected parties.
\end{enumerate}

The continued variation in DNS abuse experienced by stakeholders indicates that community efforts in terms of collaborations, innovations, and education will build on the experiences derived from stakeholders and inform the most relevant foundation on which effective strategies and policies will be developed to mitigate DNS abuse effectively.

\section{Challenges in Mitigation and Mitigation Strategies} 

The responses of the stakeholders demonstrated details of the multifaceted challenges in mitigating DNS abuse, coupled with the various strategies used to address these issues.
\begin{enumerate}
    \item Economic and Technical Hurdles: A clear barrier related to the economic nature of the DNS industry was identified, which is characterised by low margins and high volumes, leaving little resources to help mitigate DNS abuse on a large scale. The responsible registrars stressed that 80\% of the malicious domain name registrations were related to several large and reputable registrars and some small registrars, which is an example of how, from an economic point of view, the cost of security is not proportional to the amount of dangerous information that is being exploited. This points to a clear economic barrier in the DNS ecosystem and calls for new ways of efficient and economical mitigation.
    \item Regulatory Gaps: The regulatory environment was mentioned as a challenge. Included poor, weak, or lack of policies and enforcement mechanisms that did not allow us to combat DNS abuse properly. Therefore, all respondents also emphasised the need for more clear regulations and standards that could better guide industry attempts to mitigate abuse.
    \item Mitigation Strategies: They have focused on components of education, collaboration, and outreach to create knowledge and a social response. Software tools to support these efforts, abuse reporting intermediaries and measurement projects that measure the Internet, are crucial in finding, reporting, and understanding abuse cases. These tools are also designed to help report and mitigate risk, but can collect critical data with a policy and the character of a regulatory response.
    
    
\end{enumerate}


\section{Transparency in DNS Abuse Mitigation} 

The responses of stakeholders underscore the subtle perspective on transparency within the DNS abuse mitigation framework, highlighting both its potential benefits and challenges.

\begin{enumerate}
    \item Benefits of transparency: Transparency is increasingly recognised as a method for sector participants to show a firm-wide commitment to combating DNS abuse. By allowing mitigation efforts to be driven down the stack and done more rapidly and often overtly, the action becomes associated with a culture of accountability and responsibility. However, when it comes to issuing data on the prevalence of abuse and the effects of mitigation measures, transparency can strengthen confidence among users, regulators, and sector participants. Furthermore, it is acknowledged as a contributing element to a comprehensive understanding of and collaboration among the various actors in the DNS landscape. Operators, registrars, registries, and regulators may have a better understanding of the difficulties and achievements of many other performers in the section if they exchange data on abusive activities and mitigation measures.

    \item Challenges and Concerns: Stakeholders raised several concerns about the degree and manner of transparency. One fear is that some sensitive mitigation strategies are being disclosed and could serve as a means for the bad guys to find ways to find out about abuse and its mitigation. Many within the industry find here a fine balance between useful information on the one hand and, at the same time, protecting operational integrity on the other. Additionally, there is apprehension that increased transparency could lead to regulatory or legal consequences, especially if disclosures are mandated in a manner that does not consider the practical aspects of abuse mitigation. Stakeholders also pointed out such operational challenges as the ability to complete transparency reporting, given the current reliance on less formal mechanisms to report abuse and monitoring mitigation.
    
    \item Strategic Approach to Transparency: The stakeholders are for the vision that a strategic approach to transparency in support of the goals and aims for mitigation of DNS abuse should, in no way, be undermined from being effective. This would include targeted transparency, with the point being more aggregated data and trends than detailed disclosure of specific mitigation actions or techniques. But this environment should be nourished where sharing information does not result in punitive consequences, but, on the other side, supports collaborative improvement, which is considered essential. The value of transparency for the mitigation of DNS abuse is considered of high value, but stakeholders advise to be careful with each step in what, how and to whom it will be disclosed. This shall be a balanced approach that increases the collective ability to address DNS abuse, rather than safeguarding methods used. In short, it is of the utmost importance for the continual evolution of transparency practices in the industry.


\end{enumerate}

\section{Impact on Relationships within the DNS Ecosystem} 

 Stakeholders pointed to a clearer potential impact on meaningful relationship building within their particular DNS ecosystems: greater transparency and mitigation. Better transparency is seen by creating a better understanding among different parties, for instance, among registries, registrars, and regulators about challenges and works against abuse, hence their collaboration and trust that improves combined efforts against abuse. However, this provision raises concerns that such transparency could get to the point of obstructing informal cooperation in general or actually reveal sensitive techniques from an operational standpoint detrimental to entities working together. Balance is a key element to ensure that these issues are addressed and that partners work harmoniously with each other within the DNS community.

\section{Analysis and Data } 

The following is a summary of the emailed responses of the stakeholders in relation to DNS abuse mitigation. In relation to those themes, the following record the main important points.


{
\begin{table}[H]
\centering
\footnotesize 
\begin{tabular}{|p{3cm}|p{9cm}|}
\hline
\cellcolor{gray!50}\textbf{Definition Supported} & 
\cellcolor{gray!50}\textbf{Comments and Suggestions} \\
\hline
\mbox {ICANN Contracted} Parties' Definition & Endorses the ICANN definition for its clarity and actionability. However, it suggests that it may be too narrow and advocates a more flexible framework to encompass evolving threats. Points to a self-authored sophisticated way of defining harms at the domain name layer, promoting adaptability. \\
\hline
\mbox {Critique of} ICANNwiki Definition & Finds the ICANNwiki reference lacking, preferring the SSAC 115 report definition for its broader applicability and recent adoption in RAA amendments. \\
\hline
Mixed Views & \mbox {While there is alignment with the existing categorical} \mbox {definitions for practical reasons, there is a shared belief in the} necessity for definitions that evolve with emerging DNS \mbox {threats. The discussion indicates a desire for a balance} between categorical clarity and adaptability to new forms of abuse. \\
\hline
\end{tabular}
\caption{Varied Definitions and Understandings of DNS Abuse}
\label{table:dns_abuse_definitions}
\end{table}

}

{
\begin{table}[H]
\centering
\footnotesize 
\begin{tabular}{|l|l|p{5cm}|}
\hline
\cellcolor{gray!50}\textbf{Type of DNS Abuse} & 
\cellcolor{gray!50}\textbf{Frequency Mentioned} & 
\cellcolor{gray!50}\textbf{Stakeholder Comments} \\
\hline
Phishing & Most Common & \mbox {Identified as the primary concern} \mbox {across responses, significant} impact observed. \\
\hline
Compromised CMS and Confusable Domains & Frequently Mentioned & \mbox {Highlighted as a prevalent issue} \mbox {alongside phishing and other} platform abuses. \\
\hline

\end{tabular}
\caption{Types of DNS Abuse Encountered}
\label{table:types_of_dns_abuse}
\end{table}
}

{

\begin{table}[H]
\centering
\footnotesize 
\begin{tabular}{|l|p{4cm}|p{4cm}|}
\hline
\cellcolor{gray!50}\textbf{Challenge Type} & 
\cellcolor{gray!50}\textbf{Stakeholder Insights} & 
\cellcolor{gray!50}\textbf{Suggested Solutions} \\
\hline
Economic & \mbox {High volume, low margin} \mbox {business model impedes} anti-abuse efforts. & \mbox {Calls for industry-wide} collaboration and support. \\
\hline
Regulatory Gaps & \mbox {Lack of clear regulations} \mbox {complicates mitigation} efforts. & Advocates for establishing and following industry-wide best practices. \\
\hline
\end{tabular}
\caption{Challenges in Mitigating DNS Abuse}
\label{table:challenges_in_mitigation}
\end{table}


}

{

\begin{table}[H]
\centering
\footnotesize 
\begin{tabular}{|l|p{4cm}|p{4cm}|}
\hline
\cellcolor{gray!50}\textbf{Strategy} & 
\cellcolor{gray!50}\textbf{Description} & 
\cellcolor{gray!50}\textbf{Stakeholder Feedback} \\
\hline
Blocking Orders & \mbox {From certain regions to} mitigate abuse. & \mbox {Implemented alongside} \mbox {other criteria to make} \mbox {services less appealing to} abusers. \\
\hline
Education \& Collaboration & \mbox { Outreach to improve} awareness and cooperation. & \mbox {Viewed as essential, with} \mbox {a need for more systematic} implementation. \\
\hline
\end{tabular}
\caption{Mitigation Strategies Employed}
\label{table:mitigation_strategies}
\end{table}

}

{
\begin{table}[H]
\centering
\footnotesize 
\begin{tabular}{|l|p{4cm}|p{4cm}|}
\hline
\cellcolor{gray!50}\textbf{Aspect of Transparency} & 
\cellcolor{gray!50}\textbf{Benefits} 
& \cellcolor{gray!50}\textbf{Concerns} \\
\hline
Reporting Abuse Metrics & \mbox {Enhances trust and} \mbox {accountability in the} ecosystem. & Risk of exposing sensitive mitigation techniques if not managed carefully. \\
\hline
\end{tabular}
\caption{Transparency in DNS Abuse Mitigation}
\label{table:transparency_in_mitigation}
\end{table}


}

When analysing the data from the stakeholder responses, a thorough examination of DNS abuse was carried out. The stakeholders provided valuable information on the definitions of DNS abuse. The commonly mentioned type was phishing due to its spectrum in all aspects, compromised CMS, and confusables were also identified by everyone as prevalent. The most challenging aspects of mitigation included economic factors that influence achievable levels of DNS abuse in the branch and regulatory gaps because the market structure changes the evaluation of mitigation abuse measures and the lack of set regulations blurs perceptions. The mitigation strategies and plans implemented are based on targeted blocking and collaborative education; their efficiency is reduced by the industry's focus on performance and the paths of different entities. Transparency is described as a double-sided weapon; although it has accountability and trust-building potential, it can harm the industry when people use their malicious understandings. Stakeholder experiences and actions contribute to the understanding of DNS abuse, maximising the need for a multidimensional approach based on adaptation, cooperation, and the equivalence of transparency.

