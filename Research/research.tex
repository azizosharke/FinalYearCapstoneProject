\chapter{Research Methodology}



A structured questionnaire was sent by email to various stakeholders in the DNS ecosystem. This method was chosen because of its convenience, compliance with participants' busy schedules, and permission for detailed responses at the respondents' will. The approach provided a means of soliciting a wide range of expert observations on DNS abuse in terms of definition, the most prevalent types, mitigation challenges, and the theme of transparency. Emailing was chosen to reach the various players so that a much greater participation level would be reached, suitable to the schedules of the large number of participants, yet permitting ample space for an in-depth approach to the subject. Such an approach has made it achievable to strike a balance for both accessibility and convenience for participants while meeting the need for comprehensive data collection.



\section{Questionnaire Design and Distribution} 

The questionnaire had to take into account all of these issues in a multidimensional approach, giving great emphasis, but not limited, to the following definitions, types encountered in practice, challenges of dealing with mitigation, and considerations regarding transparency. Even at the very outset, the respondent could be invited to agree or state their view of a well-known definition of DNS abuse, thus showing that this matter is diverse in application and interpretation.

The questions were carefully crafted to elicit detailed insights on:

 \begin{enumerate}
 \item The definition of DNS abuse. 
  \item The types of DNS abuse stakeholders most commonly encounter, aiming to identify prevalent patterns and specific concerns within the ecosystem.
  
  \item The challenges and limitations faced in mitigating DNS abuse, seeking to understand the barriers to effective action.
  
  \item The mitigation strategies employed, gathering information on the practical steps taken and their perceived effectiveness.
  
  \item The practice of publishing reports or data as a form of transparency, exploring the current state of openness in the field.
  
  \item The role of transparency in aiding or impeding DNS abuse mitigation efforts, probing the potential impacts of increased visibility.
  
  \item  The effects of transparency on the relationships between various DNS stakeholders, considering the broader implications for cooperation and trust.
\end{enumerate}


\section{Stakeholder Responses} 

Insights from the completed questionnaire of the different stakeholders reflect several key themes and insights critical to an understanding of DNS abuse, as well as the mitigation of this abuse. These include varied perspectives on what exactly definition of DNS abuse, the types of abuse mostly observed, and difficulties experienced by stakeholders in efforts to mitigate its abuse. Additional discussions are related to methods of mitigation on how transparency provides a full view of current practices and potential areas for improvement in respect with the DNS ecosystem.

Key Themes and Insights:

\begin{itemize}
  \item \textbf{Varied Definitions of DNS Abuse:} Although stakeholders largely accepted the definition that had been adopted by the ICANN Contract Parties, they also noted its shortcomings, especially in being too categorical, and thus may leave out evolving types of abuse. It was considered that a more flexible way forward would be a robust framework for defining the abuses to be mitigated at the domain name level.
  
  \item \textbf{Common Types of DNS Abuse:} They pointed out that phishing was the most common attack type, followed by malware, botnets, and spam. It was also pointed out that one of the most common problems was linked to the challenge related to proving the quantity of spam-related domains.
  
  \item \textbf{Challenges in Mitigation:} Perhaps the most significant was the economic structure of the domain registration industry, its ability to mitigate malicious registrations without fundamentally altering it. The stakeholders clearly state that a significant difference between large registrars, generally considered good actors of the internet, and smaller registrars with a higher level of DNS abuse underscores the different aspects of this problem within different industry segments.
  
  \item \textbf{Mitigation Strategies:} The responses included different strategies, such as blocking orders from some regions or using software to monitor abusive activities. Recommendations were made regarding the role of education and outreach, including relevant projects such as NetBeacon and Compass for abuse reporting and information for DNS abuse.
  
  \item \textbf{Role of Transparency:} Opinions on transparency were mixed since part of the respondents consider this positively because it is a tool that provides evidence to the industry in its fight against abuse, part of them consider it negatively as sensitive mitigation ways could be revealed. The impact of transparency was also elaborated on developments in relationships between all stakeholders, and there is in general agreement that transparency will increase understanding and teamwork through better communication on measures set against abuse.
  
\end{itemize}

The stakeholder responses significantly enriched the research by providing a detailed look into the practical challenges and strategies in DNS abuse mitigation. The responses not only offered valuable real-world perspectives but also highlighted the importance of adaptive definitions, comprehensive mitigation strategies, and thoughtful consideration of transparency's role in the ecosystem. This analysis bridges theoretical knowledge with the experiences of those actively engaged in mitigating DNS abuse.

\section{Types of DNS Abuse Encountered} 

The stakeholder responses provided a detailed on the most prevalent forms of DNS abuse that were being encountered within their specific ecosystem. These insights reveal a view of the various types of abuse, each of which poses unique challenges that require tailored mitigation strategies.

\begin{enumerate}
    \item Phishing: It was identified by stakeholders as the most prevalent form of DNS abuse and the most visible. In fact, the total number of phishing incidents observed through tools like NetBeacon and tracked by Compass is a stark and singular metric of just how big and urgent the problem has become in the wider DNS domain.
    \item Malware and Botnets: These also included malware and botnets, that is, multifaceted DNS abuses. Such abuses compromise not only the integrity of systems but also present a security hazard to users and infrastructures in general.
    \item Spam: It is now recognised as widespread, and stakeholders have pointed out the challenges of quantifying and appropriately addressing the relevant spam-related domains. Therefore, it makes spam elusive for existing mitigation efforts that raise the bar with respect to the pursuit of next-generation detection and response mechanisms.
    \item Compromised CMS : Encounters with compromised content management systems (CMSs) have been referred to as a common encounter. Consequently, such attacks are possible in cases of some other existing vulnerabilities in web platforms. This kind of abuse reinforces the need for strong web security control practices and the need for vigilance among platform operators.
    \item "Water Torture" Attacks : Known as random subdomain attacks, represent a more technical and sophisticated form of DNS abuse. These attacks not only disrupt normal DNS operations but also require advanced countermeasures to effectively mitigate their impact.
\end{enumerate}

The varied nature of DNS abuse that stakeholders encounter underlines the fact that community efforts must continue building on ongoing collaboration, innovation, and education to address these challenges effectively. This is derived from the experiences of stakeholders and forms a basis of paramount importance on which effective strategies and policies will be formulated in the mitigation of DNS abuse.

\section{Challenges in Mitigation and Mitigation Strategies} 

The responses of stakeholders demonstrated details of the multifaceted challenges in mitigating DNS abuse, coupled with the various strategies used to address these issues.
\begin{enumerate}
    \item Economic and Technical Hurdles: A significant barrier identified was the economic structure of the DNS industry, characterised by low margins and high volumes, often limiting the resources available for robust mitigating DNS abuse efforts. Stakeholders highlighted that about 80\% of malicious domain registrations could be traced back to a mix of large, well-known registrars and smaller entities with disproportionately high levels of abuse. This economic reality complicates the implementation of effective mitigation strategies, underscoring the need for innovative solutions that are both cost-effective and scalable.
    
    \item Regulatory Gaps: The regulatory environment was also cited as a challenge, including poor, weak, or absent policies and enforcement mechanisms that could not effectively handle DNS abuse effectively. Stakeholders pointed out the necessity for clearer regulations and standards that can guide the industry's anti-abuse efforts more effectively.
    \item Mitigation Strategies: Stakeholders have responded to this with a variety of mitigation strategies. They placed an emphasis on components of education, collaboration, and outreach to raise awareness and develop a social response to DNS abuse. Technological solutions such as abuse reporting intermediaries (NetBeacon) and measurement projects (Compass) that measure the internet are vital in the finding, reporting, and understanding of the abuse cases. Designed to improve reporting and mitigation, these tools can also capture essential data with a character that helps inform policy and regulatory responses.
    
    
\end{enumerate}


\section{Transparency in DNS Abuse Mitigation} 

The responses of stakeholders underscore the subtle perspective on transparency within the DNS abuse mitigation framework, highlighting both its potential benefits and challenges.

\begin{enumerate}
    \item Benefits of Transparency: Increased transparency is widely recognised as a way to demonstrate commitment in the industry to defend against DNS abuse. It will encourage the normalisation of efforts for the mitigation across the ecosystem, which means that proactive activity becomes more commonly adopted and attributed to a culture of responsibility and accountability. Transparency in reporting abuse metrics and mitigation outcomes can also enhance trust among users, regulators, and within the industry itself, promoting a unified approach to addressing DNS abuse. In addition, transparency is seen as a contributing element in improving understanding and cooperation among various entities involved in the DNS, including operators, registrars, registries, and regulators. By sharing information on abuse trends and mitigation strategies, stakeholders can better appreciate each other's challenges and contributions, leading to more effective collaborative efforts.

    \item Challenges and Concerns: However, at the same time, stakeholders raised several concerns about the degree and manner of transparency. One point of concern is that some sensitive mitigation strategies could be exposed that, in turn, could serve as a support for malicious actors, allowing them to discover ways to detect and mitigate abuse. This fine balance between providing useful insights and protecting operational integrity is a significant challenge for many in the industry. Furthermore, there is apprehension that increased transparency might lead to regulatory or legal repercussions, especially if disclosures are mandated in a manner that does not consider the practical aspects of abuse mitigation. Stakeholders also mentioned operational challenges, such as the capacity for comprehensive transparency reporting, given the current reliance on less formal mechanisms for abuse reporting and mitigation tracking.
    
    \item Strategic Approach to Transparency: Stakeholders advocate for a strategic approach to transparency that supports the goals of DNS abuse mitigation without compromising the effectiveness of these efforts. This includes targeted transparency that focuses on aggregate data and trends rather than detailed disclosures of specific mitigation actions or techniques. Additionally, fostering an environment where sharing information does not lead to punitive outcomes but rather supports collaborative improvement is seen as essential. Although the value in transparency for the mitigation of DNS abuse is considered high, stakeholders cautiously advocate that every step be done carefully with regard to what, how, and to whom it shall be disclosed. A balanced approach that enhances the collective ability to address DNS abuse while safeguarding the methods employed is crucial for the ongoing evolution of transparency practices in the industry.

\end{enumerate}

\section{Impact on Relationships within the DNS Ecosystem} 

 Stakeholders pointed to a clearer potential impact on meaningful relationship building within their particular DNS ecosystems: greater transparency and mitigation. Better transparency is seen by creating a better understanding among different parties, for instance, among registries, registrars, and regulators about challenges and works against abuse, hence their collaboration and trust that improves combined efforts against abuse. However, this provision raises concerns that such transparency could get to the point of obstructing informal cooperation in general or actually reveal sensitive techniques from an operational standpoint detrimental to entities working together. Balance is a key element to ensure that these issues are addressed and that partners work harmoniously with each other within the DNS community.

\section{Analysis and Data } 

The detached examination of stakeholders' emailed responses in regards to DNS abuse and its various connotations is herein reported. In relation to those themes, the following record the main important points.


{
\begin{table}[H]
\centering
\footnotesize 
\begin{tabular}{|p{3cm}|p{9cm}|}
\hline
\cellcolor{gray!50}\textbf{Definition Supported} & 
\cellcolor{gray!50}\textbf{Comments and Suggestions} \\
\hline
\mbox {ICANN Contracted} Parties' Definition & Endorses the ICANN definition for its clarity and action-ability. However, it suggests that it may be too narrow and advocates a more flexible framework to encompass evolving threats. Points to a self-authored sophisticated way of defining harms at the domain name layer, promoting adaptability. \\
\hline
\mbox {Critique of} ICANNwiki Definition & Finds the ICANNwiki reference lacking, preferring the SSAC 115 report definition for its broader applicability and recent adoption in RAA amendments. \\
\hline
Mixed Views & \mbox {While there's alignment with the existing categorical} \mbox {definitions for practical reasons, there is a shared belief in the} necessity for definitions that evolve with emerging DNS \mbox {threats. The discussion indicates a desire for a balance} between categorical clarity and adaptability to new forms of abuse. \\
\hline
\end{tabular}
\caption{Varied Definitions and Understandings of DNS Abuse}
\label{table:dns_abuse_definitions}
\end{table}

}

{
\begin{table}[H]
\centering
\footnotesize 
\begin{tabular}{|l|l|p{5cm}|}
\hline
\cellcolor{gray!50}\textbf{Type of DNS Abuse} & 
\cellcolor{gray!50}\textbf{Frequency Mentioned} & 
\cellcolor{gray!50}\textbf{Stakeholder Comments} \\
\hline
Phishing & Most Common & \mbox {Identified as the primary concern} \mbox {across responses, significant} impact observed. \\
\hline
Compromised CMS & Frequently Mentioned & \mbox {Highlighted as a prevalent issue} \mbox {alongside phishing and other} platform abuses. \\
\hline
\end{tabular}
\caption{Types of DNS Abuse Encountered}
\label{table:types_of_dns_abuse}
\end{table}
}

{

\begin{table}[H]
\centering
\footnotesize 
\begin{tabular}{|l|p{4cm}|p{4cm}|}
\hline
\cellcolor{gray!50}\textbf{Challenge Type} & 
\cellcolor{gray!50}\textbf{Stakeholder Insights} & 
\cellcolor{gray!50}\textbf{Suggested Solutions} \\
\hline
Economic & \mbox {High volume, low margin} \mbox {business model impedes} anti-abuse efforts. & \mbox {Calls for industry-wide} collaboration and support. \\
\hline
Regulatory Gaps & \mbox {Lack of clear regulations} \mbox {complicates mitigation} efforts. & Advocates for establishing and following industry-wide best practices. \\
\hline
\end{tabular}
\caption{Challenges in Mitigating DNS Abuse}
\label{table:challenges_in_mitigation}
\end{table}


}

{

\begin{table}[H]
\centering
\footnotesize 
\begin{tabular}{|l|p{4cm}|p{4cm}|}
\hline
\cellcolor{gray!50}\textbf{Strategy} & 
\cellcolor{gray!50}\textbf{Description} & 
\cellcolor{gray!50}\textbf{Stakeholder Feedback} \\
\hline
Blocking Orders & \mbox {From certain regions to} mitigate abuse. & \mbox {Implemented alongside} \mbox {other criteria to make} \mbox {services less appealing to} abusers. \\
\hline
Education \& Collaboration & \mbox { Outreach to improve} awareness and cooperation. & \mbox {Viewed as essential, with} \mbox {a need for more systematic} implementation. \\
\hline
\end{tabular}
\caption{Mitigation Strategies Employed}
\label{table:mitigation_strategies}
\end{table}

}

{
\begin{table}[H]
\centering
\footnotesize 
\begin{tabular}{|l|p{4cm}|p{4cm}|}
\hline
\cellcolor{gray!50}\textbf{Aspect of Transparency} & 
\cellcolor{gray!50}\textbf{Benefits} 
& \cellcolor{gray!50}\textbf{Concerns} \\
\hline
Reporting Abuse Metrics & \mbox {Enhances trust and} \mbox {accountability in the} ecosystem. & Risk of exposing sensitive mitigation techniques if not managed carefully. \\
\hline
\end{tabular}
\caption{Transparency in DNS Abuse Mitigation}
\label{table:transparency_in_mitigation}
\end{table}


}

{

\begin{table}[H]
\centering
\footnotesize 
\begin{tabular}{|l|p{4cm}|p{4cm}|}
\hline
\cellcolor{gray!50}\textbf{Relationship Aspect} & \cellcolor{gray!50}\textbf{Positive Impacts} & \cellcolor{gray!50}\textbf{Potential Challenges} \\
\hline
Between Entities & Improved understanding and collaboration from shared data. & Concerns about competitive sensitivity and operational integrity could limit openness. \\
\hline
\end{tabular}
\caption{Impact on Relationships within the DNS Ecosystem}
\label{table:impact_on_relationships}
\end{table}
}

In analysing the data from the stakeholder responses, a thorough examination of DNS abuse has been undertaken. The stakeholders, deeply embedded in the DNS ecosystem, provide valuable insights into the definitions and manifestations of DNS abuse. They highlight phishing as the most frequent and worrying type, with compromised CMS also noted for its prevalence. Challenges in mitigation are predominantly tied to economic factors and regulatory gaps, where the industry's structure impairs antiabuse actions and a lack of clear regulations muddies the waters. Mitigation strategies like targeted blocking and collaborative education are in play, though their implementation faces hurdles due to the industry's focus on throughput and the capacities of various entities. The role of transparency is acknowledged as double-edged: Although it could foster accountability and trust, there is a risk that abusers exploit sensitive techniques. Stakeholder experiences and strategies contribute to a deeper understanding of DNS abuse, suggesting the need for a multifaceted approach that involves adaptation, collaboration, and a careful balance of transparency.

