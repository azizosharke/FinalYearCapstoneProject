\chapter{Background}
\label{Chapt2}


This chapter will cover the relevant basic information about this project, focusing on the world of DNS abuse and transparency. This will involve a detailed examination of the domain name system and its role in an online community and the many abuses it suffers, a comprehensive history of the policies and established entities used widely to combat DNS abuse, and, in particular, a detailed analysis of the DNS Abuse Institute and what it has achieved. Observing these various methodologies and appreciating their strengths and weaknesses, the reader will get a comprehensive idea of the current DNS abuse situation and the need for a transparent and proactive approach. This chapter emphasises the importance of the suggested solution in an era where digital authenticity is required, not only by providing information, but also by laying the groundwork for its presentation as a better and essential progression in the battle against DNS abuse.

\section{Understanding DNS \& Its Vulnerabilities}

DNS plays a role in maintaining ongoing online activities; privacy and security problems still arise. The ScienceDirect paper "Domain Name System Security and Privacy: A Contemporary Survey" provides a detailed analysis of these concerns that highlights the fundamental importance of DNS while illuminating the weaknesses that malicious actors may take advantage of \cite * {Sciencedirect2023dns}. The types of security threats vary widely, from DNS infrastructure targeting DDoS attacks to cache poisoning and DNS traffic hijacking. These attacks have the potential to cause damage, including service interruptions, and help with theft and spying. The lack of encryption in standard DNS design makes user query data accessible to abuse and eavesdropping, raising serious privacy concerns. However, weaknesses do not define the end of the story. The same survey also studies new solutions to improve DNS security and privacy. One example of such a new security measure is the deployment of DNSSEC, or DNS Security Extensions, which authenticate the DNS data and ensure their integrity and reliability while resisting some types of attack. Moreover, privacy-enhancing technologies have made it possible for DNS queries to be encrypted to block eavesdropping and information manipulation. They include DNS over HTTPS and TLS. The DNS threats and protection settings change over time in sync with the Internet. Such flaws and current efforts to mitigate them are part of making DNS robust and more reliable. The standard DNS request includes three types of query to speed up the process and reduce the path data length: the first is recursive, where the DNS client takes a direct answer, or an error record not found from the DNS server; the second is iterative, and if the server does not have the answer, it indicates to the client the next server, which can have it, and this happens again and again until the client receives an answer or a dead end; the third is non-recursive, when the DNS server knows the answer because it is directly responsible for this information, or, for example, has it due to the request received earlier. This factor reduces the unnecessary load on the servers involved and ultimately the traffic on the Internet.


\section{Strategies \& Collaborations in Addressing DNS Abuse}

The DNS Abuse Institute, which will focus on DNS abuse to help increase security through the domain name system, will focus on these efforts to address DNS abuse with a comprehensive approach throughout the Internet infrastructure. It helps the Internet community identify, report and mitigate DNS abuse in its mission of making the online environment more secure. Efforts by the institute, such as Compass Dashboards, provide data to registries and registrars that will enable proper decisions in combating DNS abuse. They show the commitment to transparency and education by issuing publications such as the "DNSAI 2022 Annual Report" or "DNSAI Bulletin 2023 04; Account Takeovers," which provide information on DNS abuse and how recommended mitigation practices \cite{dnsabuseinstitute2023}. Another such global strategy against DNS abuse has been contributed by the Internet Corporation for Assigned Names and Numbers (ICANN)\cite{icann2022dnsabuse} in collaboration with the entire DNS community, ICANN supports a synchronised method in the development of policies and standards on how to mitigate DNS abuse while ensuring the openness of the Internet. These participatory pillars hint at concerted efforts through policy development, technological developments, and stakeholder engagement as a central component in this collective approach to combating DNS abuse \cite{dnsai2022report}. 



\section{Different Forms of DNS Abuse}

DNS abuse takes many forms, each with its effects on users and the Internet as a whole. It is essential to understand these various pieces of evidence to create responses and regulations that work. This section will examine the comprehensive analysis of DNS abuse presented, describing the description, mechanism, and impact of each kind \cite{dotmagazine2022dnsabuse}.

\subsection{Phishing}
\begin{itemize}
    \item \textbf{Description:} Phishing is a technique aimed at deceiving individuals by creating website addresses that mimic those of companies, to trick users into revealing sensitive information such as login credentials, credit card numbers, or personal identification information \cite{webinarcare2023dnsstats}.
    \item \textbf{Mechanism:} The deception can be carried out by sending mail or messages about the need to follow the link to a site similar to the real one. \cite{jakobsson2006phishing}.
    \item \textbf{Impact:} Victims may suffer identity theft, financial fraud, and security compromise.
\end{itemize}

\subsection{Confusable Domains (Typosquatting)}
\begin{itemize}
    \item \textbf{Description:} Registering domain names that look visually similar to popular websites, taking advantage of typing errors or character similarities \cite{inta2023dnstypo}.
    \item \textbf{Mechanism:} A user can simply type a typo in the address bar, and the consequences of a visit to this site can be malicious software or an attempt to phish.
    \item \textbf{Impact:} Deception of users and potential harm to brand reputation \cite{edelman2008typosquatting}.
\end{itemize}

\subsection{Domain Hijacking}
\begin{itemize}
    \item \textbf{Description:} Unauthorised acquisition of domain names by exploiting security vulnerabilities in the domain registration system \cite{inta2023dnstypo}.
    \item \textbf{Mechanism:} There are times when the attacker, using social engineering, phishing, or vulnerabilities in the protection of the authorised domain, gains control over it.
    \item \textbf{Impact:} Loss of control of the website, redirection to malicious sites, and potential data breaches.
\end{itemize}

\subsection{Botnets}
\begin{itemize}
    \item \textbf{Description:} It involves controlling a group of computers infected with malware, used to carry out attacks or spread spam and malware \cite{citpyour}.
    \item \textbf{Mechanism:} Once a potential victim’s computer is infected, a particularly large number of compromised computers will form a network under the attacker’s control.
    \item \textbf{Impact:} Results in large-scale DDoS attacks, mass spam campaigns, and widespread malware dissemination.
\end{itemize}

\subsection{Fast Flux Hosting}
\begin{itemize}
    \item \textbf{Description:} A technique used to conceal the location of websites associated with phishing and malware distribution \cite{lin2013genetic}.
    \item \textbf{Mechanism:} Involves a network of compromised hosts that regularly modify DNS records to avoid detection.
    \item \textbf{Impact:} Makes tracking and shutting down malicious sites difficult.
\end{itemize}

\subsection{Domain Generation Algorithms (DGA)}
\begin{itemize}
    \item \textbf{Description:} It generates domain names that act as meeting points for botnets \cite{antonakakis2012throw}.
    \item \textbf{Mechanism:} Malicious software uses algorithms to generate a sequence of domain names for command-and-control servers.
    \item \textbf{Impact:} Adds complexity to efforts to disrupt botnet command and control channels.
\end{itemize}

    
\subsection{Dangling DNS Records}
\begin{itemize}
    \item \textbf{Description:}  DNS entry pointing to a resource (like around an IP address or domain name) that is under the control of the owner of the originating domain. This occurs in a scenario where cloud resources are being decommissioned and the respective DNS records for such resources are not updated \cite{friess2024cloudy}.
    \item \textbf{Mechanism:} These unclaimed DNS entries will then become available for any attacker to set up malicious services on those resources, effectively "hacking" the traffic intended for the services from the original domain.
    \item \textbf{Impact:} Results in some security issues, such as phishing, malware distribution, and data intercepting, which puts end-user information at risk from other cybercrime activities against them and their organisation.
\end{itemize}


\captionsetup{font= footnotesize}
\begin{figure}[H]
\centering
\includegraphics[width=0.7\textwidth]{background/dnsformstypes.png}
\caption{Different Forms of DNS Abuse.}
\label{fig:figureThree}
\end{figure}




\section{How DNS Abuse Harms Users}

The consequences of DNS abuse are severe and cause harm not only to the end-user but also to the organisation, and they are more than basic technological interruptions. Identity theft is one of the most prominent and direct outcomes. For instance, when it comes to phishing, a widely used kind of DNS abuse, people are lured to realistic, but misleading websites; we see how people are led to fake but realistic platforms to get information. Even if it leads to financial theft or unauthorised access to accounts, information obtained through phishing schemes can cause long-term damage to a person's reputation and credit\cite{godaddy2023dnsabuse}.

\subsection{Identity Theft}
\begin{itemize}
    \item \textbf{Phishing:} Phishing attacks often use domain names that imitate legitimate websites, fooling users into providing sensitive information such as usernames, passwords, or financial details, leading to potential identity theft.
\end{itemize}

\subsection{Financial Loss}
\begin{itemize}
    \item \textbf{Deceptive Transactions:} Users may be tricked into making payments to deceptive websites or unknowingly disclose their credit card information, resulting in financial losses \cite{bohme2013economics}.
\end{itemize}

\subsection{Data Breach}
\begin{itemize}
    \item \textbf{Malware:} Malicious software spread through compromised DNS systems can allow unauthorised access to corporate data, leading to data breaches \cite{fowler2016data}.
\end{itemize}

\subsection{System Compromise}
\begin{itemize}
    \item \textbf{Malware Infection:} Systems infected with malware due to DNS abuse can be exploited for further attacks, including the creation of botnets or the distribution of ransomware, resulting in system compromise \cite{saxe2018malware}.
\end{itemize}
\captionsetup{font= footnotesize}
\begin{figure}[H]
\centering
\includegraphics[width=\textwidth]{background/DNSabuseHarm.png}
\caption{How DNS Abuse Harms Users.}
\label{fig:figureFour}
\end{figure}


\section{Future Dangers of DNS Abuse}

Bad actor strategies and tools evolve along with technological refinements, and the progression of DNS abuse trends could entail new threats in the future. A very important aspect is that the complexity of the attack has increased. Increasing their attacks with more sophistication each time, bad actors find increasingly complex ways to exploit DNS, for example, by developing new, more deceptive phishing efforts or using highly complicated virus dissemination networks \cite{icann2022dnsabusetrends}.

\subsection{Increased Sophistication}
\begin{itemize}
    \item \textbf{Evolving Techniques:} Bad actors are constantly developing more sophisticated techniques to exploit DNS, such as advanced phishing schemes and malware distribution \cite{wrightson2014advanced}.
\end{itemize}

\subsection{IoT Vulnerabilities}
\begin{itemize}
    \item \textbf{Expanding Vulnerabilities:} The widespread adoption of Internet of Things (IoT) devices, which often lack robust security measures, presents a growing target for DNS-based attacks \cite{mahmoud2015internet}.
\end{itemize}

\subsection{Infrastructure Attacks}
\begin{itemize}
    \item \textbf{DNS as a Prime Target:} Attacks on DNS infrastructure can disrupt internet services on a large scale, including DDoS attacks targeting DNS providers or exploiting weaknesses in DNS protocols \cite{dooley2017dns}.
\end{itemize}

\subsection{Deepfakes \& AI}
\begin{itemize}
    \item \textbf{AI-Enhanced Phishing:} The use of AI technologies, such as deepfakes, has made phishing attacks more convincing and deceptive, manipulating audio and video content to impersonate trusted entities \cite{schick2020deep}.
\end{itemize}

\subsection{Cloud Computing Vulnerabilities}
\begin{itemize}
    \item \textbf{Targeting Cloud Services:} As organisations increasingly rely on cloud-based services, bad actors are exploiting DNS vulnerabilities to attack these platforms, potentially leading to data breaches and service disruptions \cite{mather2009cloud}.
\end{itemize}

\subsection{Mobile Device Exploitation}
\begin{itemize}
    \item \textbf{Mobile DNS Attacks:} The rising usage of mobile devices has led bad actors to target smartphones and tablets through DNS-based attacks, which can lead to data theft and the spread of malware \cite{au2016mobile}.
\end{itemize}

\subsection{Cryptocurrency \& Blockchain Exploitation}
\begin{itemize}
    \item \textbf{Crypto-Related DNS Attacks:} Attackers could exploit DNS vulnerabilities to redirect users to fake cryptocurrency exchanges or blockchain platforms, leading to financial fraud and theft of digital assets \cite{bashir2019advanced}.
\end{itemize}

\subsection{Political and Information Warfare}
\begin{itemize}
    \item \textbf{DNS in Cyber Warfare:} The manipulation of domain name systems can be used to spread misinformation or disrupt services during significant political events, serving as a tool for political and information warfare \cite{chapple2021cyberwarfare}.
\end{itemize}

\subsection{Exploiting Emerging Technologies}
\begin{itemize}
    \item \textbf{Abuse in New Tech Domains:} As new technologies such as 5G, AI, and quantum computing advance, tactics involving DNS abuse are likely to evolve, potentially leading to more sophisticated attacks \cite{brunner2021cybersecurity}.
\end{itemize}

\subsection{Supply Chain Attacks}
\begin{itemize}
    \item \textbf{DNS in Supply Chain Compromise:} DNS manipulation can also be employed as part of supply chain attacks, targeting software updates or cloud-based services to compromise organisations \cite{boyson2014cyber}.
\end{itemize}



\captionsetup{font= footnotesize} 
\begin{figure}  [H]
    \centering
    \includegraphics[width=0.7\textwidth]{background/Future Dangers of DNS Abuse.png}
    \caption{Future Dangers of DNS Abuse.}
    \label{fig:LOLOLOL}
\end{figure}

By understanding these future dangers and emerging trends, stakeholders can better prepare and adapt their strategies to anticipate and counteract the evolving nature of DNS abuse.


\section{Foundational Mitigation Strategies \& Best Practices }


To address the broad nature of threats, mitigating DNS abuse requires an integrated strategy that integrates multiple strategies and best practices. The establishment of reporting and monitoring procedures is one fundamental tactic. Automated systems have the ability to track domain name registration patterns that may indicate DNS abuse, and protocols to report questionable actions can help ensure prompt intervention \cite{icannndnssec}. To confirm security and ensure that systems have not been compromised, regular audits of DNS configurations and domain registrations are also necessary \cite{lucas2021tls} .

\begin{enumerate}
    \item \textbf{Monitoring \& Reporting}
    \begin{itemize}
        \item Implementation: Use automated systems to monitor domain name registration for patterns that may indicate DNS abuse \cite{icannndnssec}. Establish procedures for reporting activities to authorities or cybersecurity organisations \cite{lucas2021tls}.
    \end{itemize}
    \item \textbf{Security Awareness Training}
    \begin{itemize}
        \item Implementation: Develop training programmes for users and IT staff with a focus on recognising phishing attempts, practising browsing habits, and understanding DNS security.
    \end{itemize}
    \item \textbf{DNS Security Extensions (DNSSEC)}
    \begin{itemize}
        \item Implementation: Deploy DNSSEC to ensure the integrity of the DNS data. This involves signing DNS records to protect against modification and DNS spoofing.
    \end{itemize}
    \item \textbf{Multi-Factor Authentication (MFA)}
    \begin{itemize}
        \item Implementation: Enforce multifactor authentication (MFA) for domain registrars and interfaces used to manage DNS \cite{icannndnssec}. This adds a layer of security beyond passwords, helping to prevent unauthorised domain transfers or alterations \cite{moghaddam2014ecco}.
    \end{itemize}
    \item \textbf{Blacklisting \& Takedown Services}
    \begin{itemize}
        \item Implementation: Collaborate with cybersecurity firms to identify and blacklist domains engaged in malicious activities. Establish response teams dedicated to removing domains involved in DNS abuse.
    \end{itemize}
    \vspace{20px}
    \item \textbf{Collaboration}
    \begin{itemize}
        \item Implementation: Foster collaboration among Internet service providers (ISPs), domain registrars, governments, and cybersecurity organisations. Share intelligence and best practices to collectively improve defence against DNS abuse \cite{skopik2017collaborative}.
    \end{itemize}
    \item \textbf{Regular Audits}
    \begin{itemize}
        \item Implementation: Conduct security audits of domain registrations and DNS configurations to verify their security and ensure that they have not been compromised \cite{coronado2014auditing}.
    \end{itemize}
    \item \textbf{Machine Learning}
    \begin{itemize}
        \item Implementation: Using AI and machine learning algorithms to analyse patterns in DNS traffic and proactively predict instances of DNS abuse \cite{icannndnssec}. This proactive approach enables the identification of threats before they materialise \cite{tsukerman2019machine}.
    \end{itemize}
    \item \textbf{Geo-Blocking \& IP Filtering}
    \begin{itemize}
        \item Implementation: Deploy geo-blocking and IP filtering techniques to limit access to DNS services from regions that have a history of DNS abuse. This can reduce the risk that attackers will use these services to carry out malicious activities or distribute malware \cite{meeseedited}.
    \end{itemize}
    \vspace{25px}
    \item \textbf{Enhanced Domain Validation Procedures}
    \begin{itemize}
        \item Implementation: Enhance the domain registration process by implementing validation procedures. This may involve verifying the identity of individuals or organisations that register domains, especially domains that resemble brands or fall into sensitive categories. By taking these measures, we can strengthen security and mitigate the risks associated with fraudulent domain registrations.
    \end{itemize}
\end{enumerate}


\captionsetup{font= footnotesize}
\begin{figure} [H]
    \centering
   \includegraphics[width=0.8\textwidth]{background/diagram (7).png}
    \caption{Mitigation Strategie.}
    \label{sadasdasdada}
\end{figure}

Each of these strategies plays a role in creating a comprehensive defence against DNS abuse. By integrating these tactics, organisations can establish robust, proactive measures to detect, prevent, and mitigate the ever-evolving threats posed by DNS abuse.

\section{Summary \& Synthesis}

After exploring the different forms of DNS abuse, we look at How DNS abuse harms the user, Future Dangers of DNS abuse, and Mitigation Strategies and Best Practices. I have designed a table that has DNS abuses and the best possible mitigation strategies to help them against them, taking into account the transparency story behind it, user harm, and reasoning. 


{

\footnotesize

\begin{longtable}{|p{2.5cm}|p{2.5cm}|p{4cm}|p{3cm}|p{4cm}|} 

\hline
\cellcolor{gray!50}\textbf{DNS Abuse } & 
\cellcolor{gray!50}\textbf{User Harm} & 
\cellcolor{gray!50}\textbf{Mitigation Strategy} & 
\cellcolor{gray!50}\textbf{Reasoning} & 
\cellcolor{gray!50}\textbf{Transparency Aspect} \\ \hline
\endfirsthead

\multicolumn{5}{c}%
{
\hline \cellcolor{gray!50}\textbf{DNS Abuse} & 
\cellcolor{gray!50}\textbf{User Harm} & 
\cellcolor{gray!50}\textbf{Mitigation Strategy} & 
\cellcolor{gray!50}\textbf{Reasoning} & \cellcolor{gray!50}
\textbf{Transparency Aspect} \\ \hline
\endhead

\hline \multicolumn{5}{|r|}{{\cellcolor{gray!50} Continued on next page}} \\ \hline
\endfoot

\hline
\endlastfoot
Phishing & \mbox{Identity Theft}, Financial Loss &  \mbox{Security Awareness} \mbox{Training, Enhanced Domain} Validation Procedures & \mbox{Training helps users} \mbox{recognize phishing} \mbox{attempts. Validation} prevents the registration of mimic domains. & \mbox{Increases awareness and} \mbox{scrutiny during domain} registration. \\ \hline

\mbox{Confusable} Domains \mbox{(Typosquatting)} & Unauthorised Account Access & \mbox{Enhanced Domain} \mbox{Validation Procedures}, Regular Audits & \mbox{Prevents Registration} of Similar Domains. \mbox{Audits ensure} \mbox{compliance.} & \mbox{transparent domain} \mbox{registration process.} \\ \hline

\mbox{Domain} \mbox{Hijacking} & \mbox{System} \mbox{Compromise}, \mbox{Data Breach} & \mbox{Multi-Factor Authentication} (MFA), Regular Audits & \mbox{MFA secures domain} management. \mbox{Audits verify security} measures. & \mbox{Accountability in domain} management. \\ \hline

Botnets & \mbox{Malware} \mbox{Distribution} & Collaboration,Machine Learning & \mbox{Intelligence Sharing} \mbox{identifies botnet} \mbox{activities. AI predicts} \mbox{the formation of} \mbox{botnets}. & \mbox{Shared responsibility and} proactive detection. \\ \hline

\mbox{Fast Flux} \mbox{Hosting} & \mbox{System Infections} & Blacklisting and Takedown Services, Geo-Blocking & \mbox{Rapid response to} \mbox{malicious domains.} restrict access from risky regions. & Responsive and transparent threat management. \\ \hline

\mbox{Domain} \mbox{Generation} Algorithms (DGA) & \mbox{Malware} \mbox{Distribution} & \mbox{Machine Learning, DNS} \mbox{Security Extensions} (DNSSEC) & AI detects abnormal \mbox{patterns. DNSSEC} \mbox{prevents spoofing.} & Integrity and trust in DNS data. \\ \hline

\mbox{Dangling DNS} Records & Service Disruption & Monitoring and Auditing of DNS Records & \mbox{Regular monitoring} \mbox{allows for the early} \mbox{detection of dangling} \mbox{DNS records,} \mbox{reducing the window } \mbox{of opportunity for} attackers. & Promotes proactive security \mbox{practices and reduces} \mbox{the incidence of service} interruptions \\ \hline

\mbox{IoT} \mbox{Vulnerabilities} & \mbox{Unauthorised} \mbox{Access, Data} \mbox{Breach} & \mbox{Security Awareness} \mbox{Training, Collaboration} & \mbox{Educates on security} \mbox{practices.} \mbox{Collaboration on best} \mbox{practices.} & \mbox{Open exchange of} \mbox{knowledge and efforts.} \\ \hline

Infrastructure Attacks & \mbox{DDoS Attacks}, \mbox{System Downtime} & DNSSEC, Collaboration & Protects DNS Data Integrity. Sharing of threat intelligence. & \mbox{Collective action}  \mbox{strengthens the DNS} infrastructure.  \\ \hline

Deepfakes and AI & \mbox{Identity Theft}, \mbox{Misinformation} & \mbox{Security Awareness} \mbox{Training, Monitoring} & \mbox{Recognising Phishing.} \mbox{Monitor} \mbox{AI threats.} & \mbox{Vigilance and prompt} \mbox{threat reporting.} \\  \hline

\mbox{Cloud} \mbox{Computing} Vulnerabilities & \mbox{Data Breach}, \mbox{Unauthorised} Access & Regular Audits, Enhanced Validation & \mbox{Secure DNS settings} \mbox{in cloud services.} \mbox{Prevents exploitation.} & \mbox{Framework for secure} \mbox{domain use in cloud.} \hline

\mbox{Mobile Device} Exploitation & Unauthorised Access, Financial Loss & \mbox{MFA, Security Awareness} Training & \mbox{Secures account} \mbox{access.} \mbox{ Raises awareness} of threats. & Mobile security awareness and protection. \\ \hline

\mbox{Political and} Information Warfare & Misinformation, Political \mbox{Manipulation} & Monitoring, Collaboration & \mbox{Monitoring abuse in} \mbox{campaigns. Unified} \mbox{response to } \mbox{misinformation.} & Transparency in monitoring and collective action. \\ \hline

\mbox{Exploiting} Emerging \mbox{Technologies} & system \mbox{Vulnerabilities} &\mbox{ Machine Learning,} \mbox{Collaboration} & \mbox{Analytics to predict} \mbox{DNS abuse. Share} \mbox{knowledge about} \mbox{threats.} & \mbox{Innovation in defense}  \mbox{strategies and sharing.} \\ \hline

\mbox{Supply Chain} \mbox{Attacks} & \mbox{System} \mbox{Compromise,} Data Breach & Regular Audits, Blacklisting & \mbox{Audits for DNS} \mbox{integrity. Rapid} \mbox{response to threats.} & \mbox{Transparency in supply} \mbox{chain security.} \\ \hline

\caption{Mitigation strategies against DNS abuse and its impact on users.} 

\end{longtable}

}


Finally, this chapter has examined all aspects of DNS abuse,  the various forms, the serious harm it does, and potential future threats. Understanding these ranges and the effects they can have is important for the development of regulation and measures. Both the DNS Abuse Institute and ICANN have taken great steps in dealing with this issue. With the advancement of technology and the growing threats, it is more of an adaptive and collaborative approach that remains the key. Possible mitigation techniques that have been discussed outline a guide to the possible approach to combating DNS abuse such as advanced technology, enhanced validation, and continuous monitoring. Cooperation with the use of new technologies is indicated, hence, in DNS abuse mitigation, to reach a joint effort in the management of abuse. Thus, a comprehensive strategy would, of course, call for some appropriate tools, but it would also be a combination of approaches and, most importantly, cooperation from the industry. The evolution of the digital landscape requires adaptable approaches to maintain the security of the DNS and Internet infrastructure. 



\chapter{State of the Art}

This chapter outlines the methods for mitigating abuse of DNS, and developments in the field, and compares the efficiency and transparency of the various ways to counteract the threat, including DNS filtering, or threat intelligence in which experts bring together and examine relevant information about cyber attacks. Furthermore, the section mentions the recent methods of DNS abuse, such as actions related to domain-generating techniques, as well as DoT and DoH. AI and machine learning are emphasised to detect and counter DNS abuse: The last peculiarities characterise the latter half of the section. It discusses the various directions that require more expertise and solutions and technologies that can advance the prevention of DNS abuse. Case studies present real examples of DNS abuse actions and events.




\section{Current Strategies and Their Effectiveness to DNS Abuse}


DNS abuse presents a challenge to Internet entities involved in domain name management. Various approaches are employed to mitigate such abuse, including DNS filtering, which regulates access to specific websites and prevents you from accessing malicious sites that can administer phishing and ransomware. In addition, threat intelligence methodologies use data analysis to identify potential risks, as exemplified by \cite{schmid2021thirty}. Anomaly detection plays a role in identifying suspicious DNS activities indicative of malicious intent using Packet Analysis to analyse individual packets for DNS allowing for real-time detection and statistical analysis, which involves performing statistical analysis on a large dataset of DNS traffic. However, these methods can face operational challenges, such as errors and the need for fast access to critical threat data. 

\subsection{Transparency in DNS Abuse Mitigation \& DNS Relevance}

\begin{enumerate}
    \item A Case Study of Cloudflare's Transparency Approach

Cloudflare claims to be committed to maintaining transparency \cite{cloudflare_transparency_2022}, which is the keystone of their relationship with customers, guiding each of these approaches to reports of abuse of the DNS and requests that may come from law enforcement. All of these reduce their actions and policies in shaping a trustworthy environment in light of addressing Internet safety and privacy concerns. Their approach to handling DNS abuse reports and law enforcement requests are anchored on three core principles:


\begin{enumerate}
    \item Due Process: Cloudflare will comply with due process as required by law, remaining neutral and not exceeding legal requirements.
    
    \item Privacy: Cloudflare respects your privacy and will never sell or otherwise share any personal or private information with any third party without your explicit permission. This applies to each request.
    
    \item Notice: Cloudflare will notify customers if legal requests are made for their information unless prohibited by law.

\end{enumerate}

Handling of DNS Abuse and Law Enforcement Requests:

\begin{enumerate}
    \item Cloudflare's response to DNS abuse by phishing and malware is decisive action: service termination for non-compliant domains. In the second semester of 2022, a significant number of accounts and domains were suspended because they hosted harmful content.

    \item The legality of such requests is reviewed with strictness by the company, ensuring that required information is provided to the respective bodies within international privacy laws; if they infringe upon user rights, they are rejected.   
\end{enumerate}

Challenges and Efforts to Mitigate DNS Abuse:

\begin{enumerate}
    \item Cloudflare aims to mitigate DNS abuse, balance free speech with the law, and bring cooperation with all parties through its proactive work.

 \item The company understands the challenge of dealing with DNS abuse, and great effort is made to provide transparency concerning the privacy standards set by the law.
 
\end{enumerate}

Future Directions:

\begin{enumerate}

\item Cloudflare intends to improve partnership participation and abuse detection systems with due transparency in reporting. They have also redoubled their efforts in the field of education to increase cybersecurity awareness among users and lead reform policies and legal concerns in line with the balance between privacy and law enforcement. 

\end{enumerate}

In conclusion, the company emphasises its commitment to protecting legal processes and user privacy while navigating government and law enforcement requests. A significant aspect of these reports is Cloudflare's approach to DNS requests, particularly regarding content blocking through its 1.1.1.1 Public DNS Resolver. This was the key answer: Cloudflare, in no uncertain terms, "received legal requests to block content at our DNS servers" and stated its policy to first "exhaust legal remedies" that they could enforce. This is an indication of how very carefully Cloudflare has to adhere to the demands of the law, yet protect the openness of the Internet, bringing out just how DNS is in all matters that pertain to the accessibility of content on the Internet and governance of the Internet. Detailed statistics, trends, and specific case studies that formed the basis of their latest transparency reports can be found in Appendix \ref{app:cloudfare}.

\item Google Transparency Reports 

This shows the weight attached to the Domain Name System (DNS) when enforcing the requests from the global governments, more so in between them and the internet governance, concerning the content removal from Google services. Data from Russia, with tens of thousands of redaction requests, might signal broader actions that include DNS-level interventions. This highlights the kind of role DNS plays in controlling access to the Internet or blocking content, which is usually put under legal and regulatory pressure from major tech companies, including Google. Any question related to these requests, although not directly related to the manipulation of DNS, implies the possibility of any technical adjustment to be carried out to fulfil the criteria directly affecting DNS resolutions. This indirect reference considers DNS to be one of the critical infrastructures in the debate on Internet governance, censorship, and access to information. What it does is show the Google Transparency Report, which indicates the fact that DNS is an important architecture of the Internet and is also a trouble spot for exercising control over digital content and information flow \cite{Google2023}.

\item Amazon Transparency Reports 

This role of DNS in the service of governments or other legal data demands does not trace directly to specific acts of manipulation in the DNS or intervention at the domain level. The report explains Amazon's observance of due process laws in handling requests for data such as subpoenas and search warrants, with a lot of emphasis on customer privacy and protection of data which can be mounted against the state or any other third party institution or person. Handling the domain or the services to do with this website means that a possibility of such a move as DNS changes can be in the offing. However, they do not give clear examples where DNS interventions have been taken, but describe the circumstances related to legal compliance and internet governance without direct reference to DNS \cite{Amazon2023}.
\vspace{20px}


\item The Meta-Transparency Reports

At the same level of social media, the enforcement of intellectual property rights, including Facebook and Instagram, shall entail the enforcement of a comprehensive strategy targeting copyright, counterfeit, and trademark infringements, with an important focus on the Domain Name System (DNS) as the centre stage for such activities.  DNS serves both as a foundation for the distribution of information online and as a checkpoint in the enforcement process. For example, content removals from Facebook and Instagram amount to 447,123 and 297,356, respectively, in the first half of 2022. This shows a scenario in which interventions range from more than platform moderation to include DNS-level actions of deindexing websites or altering DNS records to block access to infringing content.

The sustained rate of content removals since the latter halves of 2020 and 2021 indicates a reliance on DNS mechanisms. This may explain the huge year-over-year drop in Facebook's copyright and counterfeit content takedown requests from 2020-2021. It would seem that Meta may not work with DNS providers to have the offending domains taken down but instead remove the infringing content. This underscores how important DNS is in the enforcement of intellectual property rights, in the control of counterfeit, fake, and grey markets, and in protecting the rights of the owner of intellectual property and trademarks \cite{Facebook2023}.

\item  T-Mobile Transparency Report

It outlines how the company complies with directions of the law in the management of requests for information from consumers, thus highlighting staying within customers' privacy and legal compliance. Details the approach and policies of the company in response to lawful requests on records of customers within T-Mobile, Metro by T-Mobile, and Sprint, now collectively T-Mobile USA, Inc. (TMUS). At the same time, it provides information about what TMUS does to protect consumers from unauthorised data access, including first-party requests made by the company itself, such as subpoenas, court orders, and warrants, with all processes required following the same. When sharing details on the number and types of request received in 2022, the report puts a heavy emphasis on TMUS's efforts to respect customer privacy and comply with applicable legal obligations. In the case of T-Mobile, it handled 301,388 subpoenas, mostly related to orders to disclose information about the subscriber, such as names and addresses, and 94,599 different types of warrants or search warrants, which can be after historical location data or the content of messages \cite{TMobile2022TransparencyReport}.

\item IBM 1H 2021 Law Enforcement Requests Transparency Report
 
 IBM focuses on data ethics and transparency, just as it has done throughout the years to build trust among clients. The emphasis is on who owns the data and promotes client data policies, belonging to the government, and being fair and not discriminatory. The IBM report aims to make it clear where the company stands on the issue of client data that go through government surveillance. Therefore, it advocated that governments make their request for information directly to the client and ensure that the engagements between them are strictly regulated by legal protocols, including Mutual Legal Assistance Treaties (MLATs). IBM received 27 law enforcement requests in the first half of 2021, most of them related to the provision of basic subscriber contact information. It underpins how rarely and seriously IBM views requests for customer data. This reflects how IBM is committed to client privacy and data protection by ensuring strict controls in relation to data access, including those prompted by legality and governance \cite{IBMTransparencyReport2023}.
 
\item Xiaomi Transparency Report: Government Requests for User Information 

It indicates how Xiaomi processes user data requests from the government and testifies to this company's determination towards transparency and legality. Strives to follow technical and organisational practices set as standards within the industry in the world and full respect for the laws and regulations. This general review portrays Xiaomi as a transparent organisation in the way it handles various requests from the government, from the device level to financial and account-based data, underlining the trust that Xiaomi has built with consumers regarding their privacy and data protection. In 2022, there were 51 device-based requests, among the many applications received by the Indian government. Among the device inquiries, 49,683 devices were answered, with 32 in compliance. The Xiaomi compliance rate in India reached an impressive 62. 75\%. It is indicative of the fact that the company is usually under huge government inquiries from regions where it has big stakes and shows the nature of the requests that this company has always faced \cite{XiaomiTransparencyReport}.
 
\item eBay Global Transparency Report 

 The report is a demonstration of eBay's commitment to making the marketplace safe and reliable for the global community of buyers and sellers transacting on its platform. Defined with great focus, eBay lists everything they are doing to protect their marketplace from counterfeit goods, fraud, and any other abuse. With advanced AI technologies and image detection, eBay will be able to identify and remove listings of goods that could pose risks to safety or health, with close follow-up efforts to improve cooperation with rights owners and law enforcement. They are included in measures within the scope of eBay investments in technology and partnerships towards the retention of platform integrity. Reflecting the policies and their impact on the initiatives of the company for more than two decades, the report has highlighted that eBay believes in creating an open and honest marketplace that can help individuals generate economic opportunities from across the world. eBay AI tools had proactively stopped 295 million listings of prohibited items during 2022, a clear indication that its technology is very key to stopping the sale of controlled substances and other damaging items. On the other hand, the Authenticity Guarantee programme further underlines the quality consciousness of eBay and builds trust by allowing verification services for luxury offerings, which include watches, handbags, jewellery, sneakers, and cards \cite{eBayGlobalTransparencyReport2022}.

 \item Apple Transparency Report : Government \& Private Party Requests 

It details the process by which Apple's legal team handles all legal requests from global government agencies and US private parties, categorising them by Devices, Financial Identifiers, and Accounts. This highlights the process that Apple undertakes with all the devotion to the protection of user privacy and information safety, at the same time dealing with the requests within legal standards. This commitment to transparency is aimed at building trust and informing opinions about Apple's operations. The report is key for any reader who is interested in understanding at a more detailed level the intersections of technology, privacy, and law enforcement in the digital age. The information describes the types and volumes of requests in which, for example, Apple reports having received 5,660 device requests in the US and reports that have furnished information for 82\% of these requests, mostly associated with investigations of lost or stolen devices or fraud. The U.S. posted a total of 7,944 account requests, with a disclosure rate of 47\%. This clearly proves that Apple has been pretty guarded in its responses to requests for user data. \cite{AppleTransparencyReportGB}.


\end{enumerate}

\subsection{Effectiveness of Current DNS Abuse Mitigation Strategies}

There are various ways in which this abuse can be mitigated. Approaches include the deployment of blocking tools, the knowledge of potential attacks, and the detection of suspicious attempts. DNS filtering is a practice in which access to any particular website is controlled based on predefined rules concerning the result you would obtain based on the background context, and it can occur in multiple forums, e.g., register and registry are such forums where a DNS filtering mechanism would compare DNS names to the block lists and the set of rules then takes the necessary action such as. It could be used to prevent homograph attacks volumetric DDoS attacks DNS filtering mechanisms could be used to compare domain names against block lists and the predefined rule to identify possibly dangerous homographs discussed above. Additionally, threat intelligence contributes to the identification of warning signs and detection of abnormal activities in DNS \cite{rizvi2022application}. This can help to identify and assess potential threats and evil activities early. For example, it can detect similarities that might point to a phishing campaign, domain hijacking, malware distribution, or another form of DNS abuse. To determine the relative effectiveness of each of these methods, their applications must be compared with real-world performance. DNS filtering, for example, might be effective in blocking malicious content. However, it may also allow harmful content to penetrate the filtering process and therefore impact the end-user experience. Threat intelligence is as effective as the timeliness and accuracy of the data used. However, identifying anomalous behaviour poses challenges, as distinguishing between malicious actions and legitimate activities performed in innovative ways can be challenging.



\section{Emerging Trends in DNS Abuse}

Trends in DNS abuse had declined among some categories, such as botnets, malware, phishing, and spam. Much of this decline could be attributed to the multipronged approaches that ICANN itself launched around data analysis, community tools, and enforcement of registry and registrar obligations \cite{icann_dns_security_threat}. Although continuing to be slow, adopting organisations did so under the compulsion of situations that left them no choice but to use technology or by those for whom TLS adoption was a matter of technological innovation, choice, or desire for the embrace of technologies simpler and more robust from misdirection. One of the major issues has continued to be privacy, due to the fact that DNS queries have been accidentally found to give away user behaviours. One such move to enhance user privacy is the Query Name Minimisation. The main concern has been how to remain vigilant against DNS abuses while improving privacy without altering service efficiency.

\subsection{Evolving New Forms of DNS Abuse}


The field of cybersecurity is rapidly advancing, bringing forth new challenges as it evolves, and constantly moving the goalposts for defence mechanisms. In such a setting, the rapid growth and implementation of DNS over TLS (DoT) and DNS over HTTPS (DoH) constitute a double-edged sword. While the above encryption protocols were intended to increase privacy and security by encrypting DNS questions, they also incidentally provide threat actors with a means to mask malicious traffic, thereby increasing the threat surface. The above can be pointed out in various facets, from personal devices to organisational networks. For example, malefactors might employ DoH and DoT in the enterprise context to bypass obsolete security safeguards and create concealed communication links. In addition, domain generation algorithms (DGA) are of great importance in cyber threats, generating a massive number of random domain names automatically, making it difficult to locate and deactivate threat-promoting websites \cite{kaur2023artificial}. This method, which is an integral aspect of botnet command and control (C2) operations, complicates the efforts of cyber defence systems to anticipate and identify dangers.

The benefit of enabling DoT and DoH is to improve the level of current privacy by avoiding DNS query surveillance and encrypting DNS traffic, which reduces the likelihood of intercepting or manipulating data by bad actors. However, such protocols do give attackers a means to hide their malicious activities, which in turn poses problems to traditional DNS security systems when trying to detect and deflect harmful content. This could cause such protocols to unintentionally bypass content filtering policies and, therefore, give way to potential security breaches within the organisational environment. On the other hand, DGAs enable bad actors to avoid detection and keep their C2 communication channel open because dynamically produced domains are impossible to forecast and block on a preemptive basis. As a result, numerous domain names will become available to security facilities to monitor, making the intelligence task more difficult and requiring consistent focus and blacklist updates. Given that both methods have achieved substantial use, cybersecurity practitioners are encouraged to take a proactive and educated position, recognise the potential for exploitation of these patterns, and establish comprehensive procedures. Those should take into account the advantages of encryption and domain generation, as well as the requirement to combat DNS-based abuse on all digital fronts.





\subsection{Predictive Measures \& Their Transparency}

Efforts to mitigate DNS abuse are set toward immediately slowing such activities by utilising complex systems and advanced machine learning algorithms to detect patterns indicative of DNS abuse. Articulating and sharing insights about the decision-making processes in predictive modelling is considered significant, as well as the efforts by registrars and registries, acting together, in the context of DNS Abuse Transparency are comprehensive. These entities will invoke a wide range of mitigation measures to minimise damage and losses related to DNS, which will ensure the development of a more secure and trusted Internet environment. Some key mitigation strategies are account-based remediation in the way that maliciously generated accounts are locked out and further validated, in addition to monitoring third-party feeds and reports from cybersecurity organisations, law enforcement, and the public to discover and address abuse early. Moreover, this mitigation involves malware analysis, which comes from attacks on the communication infrastructure and the corresponding IP addresses, through suppression or sinkholes in the context of botnets and the use of domain generation algorithms (DGA) that direct botnet traffic \cite{ M3AAWG2024}. Most specifically, sinkholing is an authoritative measure that directs traffic from abusive domains to harmless servers and allows studies to be conducted on the sources of traffic and the extent of compromise. Compliance with legal and contractual requirements further underscores the actions of registrars and registries against DNS abuse, ensuring that their actions in mitigation are within the context of the ICANN agreements and local laws. 

The evident evaluation of real-time blackhole lists (RBLs), in addition to the responsible role of trusted notifiers, further increases the effectiveness and accuracy of mitigating actions, to filter and validate reports on abuse, so that proper responses may be made. This multipronged approach on the part of the registrars and the registries towards the mitigation of DNS abuse does not only emphasise the proactive and reactive measures, but also the possibilities of increased transparency as far as reporting and publicising the actions in place against DNS abuse are concerned. This type of transparency is key, as it helps build trust, is open to accountability and fosters an environment conducive to the collaboration of stakeholders that will allow a more effective fight against abuse in the DNS ecosystem, as illustrated in the figure \ref{fig:fig14} below. This transparency enables us to understand the predictions of the models, the opportunity to map the data used when training the model, and how to understand the methods that underlie the decision, as highlighted in \cite{hussain2022software}. The problem lies in the fact that the complexity of modern predictive models and their simplicity of interpretation are very sensitive. Therefore, it is essential to approach this challenge with caution, ensuring that the models are not only effective in identifying DNS abuse, but also accessible for thorough examination and accountability. 

\captionsetup{font= footnotesize}
\begin{figure}[H]
\centering
    \includegraphics[width=0.6\linewidth]{background/DNSECO.png}
    \caption{DNS Ecosystem Contractually Related to ICANN (image
courtesy of Verisign and originally published in SSAC 115 adapted from \cite{SSAC2023SAC115})}
    \label{fig:fig14}
\end{figure}


\section{Technological Advancements}

Mitigation of DNS abuse is increasingly influenced by the integration of artificial intelligence (AI) and machine learning technologies \cite{goethals2021enabling}. At the helm of this evolution are innovative tools such as the iQ Domain Risk Score, which uses machine learning and string analytics to actively detect potential domain abuses now of registration \cite{dnsabuseAI2023}. This tool aims to act as an advance in the mitigation mechanism that prevents abuse by analysing the domains against criteria that indicate malicious intent in an attempt to stop the abuse before it takes place. The field sits at a crossroads of transformation in analysing evidence from abuse reports through the adoption of Large Language Models (LLMs), such as Generative Pre-trained Transformers (GPTs). The models are especially suited to parse and comprehend the complicated data relations humans may overlook, boosting the efficacy and automation of DNS abuse remediation and expanding the shield against cyber attacks. However, this progress also highlights an emerging challenge: the potential for malicious entities to exploit AI technologies themselves \cite{halvorsenAI2023}. Therefore, the combination of AI and machine learning with DNS abuse mitigation marks an unprecedented milestone in the history of cybersecurity, while simultaneously serving as a warning that these technologies should not be used for harmful purposes. The present stage in the life cycle of DNS abuse is a time to reinvent and adjust more if digital ecosystems are to be properly protected.

\subsection{Role of AI \& ML}


The introduction of AI and ML technologies into DNS abuse mitigation represents the dawn of an exciting new age of proactive detection and neutralisation of cyberspace threats\cite{tariq2023critical}. The application of this technology makes it possible to quickly analyse large volumes of data and identify patterns that suggest an element of DNS query malevolence. For example, machine learning has been widely used in the analysis of DNS queries to categorise domain names. On the basis of the findings, it has become possible to detect malicious ones with a significantly higher level of precision \cite{LiMaliciousDomainDetection2020}. For example, Li argued that machine learning reduced the number of false positives during domain verification. Meanwhile, neural network models, including the Extreme Learning Machine, exposed a level of precision of 99. 5\% while analysing malicious domain patterns. Thus, it has been shown that it is possible to increase the predictive power of AI in terms of cybersecurity \cite{ZouDNSGraphMining2015}. Furthermore, DNS graph mining has shown the way of AI application possibilities in the domain of cybersecurity, such as the opportunity to exploit belief propagation algorithms allowing for high rates of precision for infected hosts and malicious domain identification. The following cases demonstrate the importance of AI and ML in DNS abuse, creating a new opportunity for rapid identification and early prevention. However, the complexity of AI models with an urgent need for transparency presents a significant problem. Although AI incorporation can contribute to secure measures against DNS abuse, the effort should not exclude ethics and certain governance considerations. Antonakakis et al. \cite{AntonakakisMalwareDomainsUpperDNS2011}. mentions that "including AI in DNS abuse strategies will require consideration of the governance and ethics of stakeholders". AI and machine learning can help improve DNS abuse mitigation, but experts must be clear about the problem.  It is important to understand how AI models make certain decisions. This helps to build trust and ensures that people are responsible for them. There are difficulties in making things clear, such as needing to write down what data was used for training, telling others about the things that affect choices, and explaining how models change to face new risks. It is still difficult to find the right balance between the complexity needed for good threat detection and the transparency needed for blame.

In summary, AI and ML are useful for defending against a variety of devices and rapidly changing cyber threats, such as those in the IoT. The performance of their predictions is sometimes conditioned on the quality of the data and the volume of data used in the development of the system. Continuous learning and adjustment in AI/ML versions are essential due to the emergence of more advanced assaults which avoid detection algorithms.


\section{Case Studies and Real-World Applications}


In recent years, technology has become so widespread that we have witnessed an unmatched number and complexity of cyber threats. A significant vulnerability that can be exploited is the DNS domain name system, a critical part of the internet infrastructure that translates human-readable names into IP addresses \cite{kumari2021sac115}. 

\begin{enumerate} 

\item\textbf{ Case Study 1: OilRig DNS Tunneling Attack }

The case of OilRig reflects the use of custom DNS tunnelling protocols for command and control (C2) operations, thus making it dual-use in nature, both in normal operation and on a fallback communication channel \cite{paloaltonetworks2021dnsattacks}.The xHunt campaign as seen in the figure \ref{fig:figTen} below \cite{unit42_xhunt_2021} followed a similar trend of including Snugy backdoor implants in targets of Middle Eastern government organisations and keeping track of them using DNS tunnelling for communication with its C2.  These are examples that underscore the strategic use by adversaries of DNS tunneling techniques for stealthiness and resilience within the context of their operations \cite{unit42_2021}.

\captionsetup{font= footnotesize}
\begin{figure}[H]
    \centering
    \includegraphics[width=0.6\textwidth]{background/DNSTuu.png}
    \caption{DNS tunneling communication between the attacker's command and control (C2) infrastructure and the victim's network.}
    \label{fig:figTen}
\end{figure}


\vspace{25px}
\item\textbf{ Case Study 2: SUNBURST Use of DGAs}

SUNBURST backdoor associated with the breach of the SolarWinds supply chain represents a case in which the use of DGAs is critical, if not only, to conceal communications and system details \cite{paloaltonetworks2021dnsattacks}. The SUNBURST backdoor, as observed in Figure \ref{fig:figEleven} below, applies the deep use of DNS manipulation for evasion purposes and subsequent attack stages by encoding basic system identifiers and the usage of DGAs for C2 check-ins \cite{unit42_solarstorm_2021}.

\captionsetup{font= footnotesize}
\begin{figure}[H] 
    \centering
    \includegraphics[width=0.6\linewidth]{background/SUNDNS.png}
    \caption{SUNBURST backdoor's utilization of DGAs and its associated components.}
    \label{fig:figEleven}
\end{figure}

\item\textbf{ Case Study 3: Fast Flux Techniques}

The presence of several C2 domains related to the Smoke Loader malware family using Fast Flux techniques only further underscores the difficulties associated with the tracking and eradication of DNS-enabled threats. \cite{paloaltonetworks2021dnsattacks}.The major takeaway in the rapid rotation of IP addresses of this method, as the figure \ref{fig:figTweleve} below, points to the dynamism of strategies used in malicious communications, thus improving the means of defence by cybersecurity \cite{unit42_fastflux_2021}.

\captionsetup{font= footnotesize}
\begin{figure}[H]
    \centering
    \includegraphics[width=0.6\linewidth]{background/FastFluDNS.png}
    \caption{The usage of Fast Flux techniques by the Smoke Loader malware family for dynamic C2 domain communications.}
    \label{fig:figTweleve}
\end{figure}


\item\textbf{Case Study 4:  Malicious Newly Registered Domains (NRDs)}

Malicious NRDs crafted opportunistically in the context of the pandemic expose how threat actors exploit current events to engineer targeted attacks as observed in the figure \ref{fig:figThirteen}. From domains that mirror the information resources of COVID-19 to those that feign government relief programmes \cite{paloaltonetworks2021dnsattacks}, the evolution of such attacks reflects a calculated approach to exploiting public interest and vulnerabilities  \cite{unit42_covid19_phishing_2021} .

\captionsetup{font= footnotesize}
\begin{figure}[H]
    \centering
    \includegraphics[width=0.6\linewidth]{background/PandemicTime.png}
    \caption{The usage of Fast Flux techniques by the Smoke Loader malware family for dynamic C2 domain communications.}
    \label{fig:figThirteen}
\end{figure}

\end{enumerate}

In the coronavirus pandemic, too, phishing attacks changed to initially targeting PPE and testing kits, then turning to government stimulus programmes and subsequently enlisting vaccine distribution. Several of them employed sophisticated tools, such as MFA pretending to the US Federal Trade Commission and brands such as Pfizer and BioNTech, to steal credentials. where it emphasised that there was a 530\% surge in vaccine-related phishing attempts and an 189\% increase in attacks on pharmacies and hospitals from December last year to February this year. Advice was given to individuals and organisations that included being cautious in email and website transactions, advancing security awareness training, and adopting multifactor authentication.

Since January 2020, a total of 69,950 COVID-19-related phishing URLs have been received, of which 33,447 are specifically dedicated to COVID-19, as Figure \ref{fig:figFourteen} shows. Data have been normalised in such a way that the peak of each topic is 100\%. The results showed much steadier phishing when it came to topics such as pharmaceuticals and virtual meeting platforms (e.g., Zoom) with vaccines and testing showing sharper rises and falls in the attention of scammers.

\captionsetup{font= footnotesize}
\begin{figure}[H]
    \centering
    \includegraphics[width=0.5\linewidth]{background/CovidPhising.png}
    \caption{Development trends in the majority of COVID-19-related phishing content hosting sites during the period from January 2020 to February 2021. Adapted from \cite{Unit42AtricleCovidPhishing2021}.}
    \label{fig:figFourteen}
\end{figure}

It is evident that a large portion of COVID-19 themed phishing pages targeted leading brands for phishing business credentials, such as Microsoft login, Webmail, and Outlook login as demonstrated in figure \ref{fig:figFiveteen}. For example, about 23\% of these phishing URLs were posed as Microsoft login pages. This threat has particularly highlighted the shift towards remote work in the pandemic and hence magnified the relevance of these attacks as one of the foremost methods that bad actors are taking on.


\captionsetup{font= footnotesize}
\begin{figure}[H]
    \centering
    \includegraphics[width=0.5\linewidth]{background/TOPCOVIDURLS.png}
    \caption{Top spoofed websites in COVID-themed phishing attacks (global), where the percentage in each column is the percentage of phishing volume per site and category. Adapted from \cite{Unit42AtricleCovidPhishing2021}.}
    \label{fig:figFiveteen}
\end{figure}

Thus, this indicates a situation in which attackers frequently set up websites for COVID-19 themed phishing attacks as depicted in figure \ref{fig:figSixteen}. Many of these phishing pages are found on sites created less than 32 days, meaning that these sites are launched for specific purposes because of these imminent attacks. The strategy allows attackers to customise their messages and URLs to the current pandemic trends, indicating the dynamism behind such cyber threats.

\captionsetup{font= footnotesize}
\begin{figure}[H]
    \centering
    \includegraphics[width=0.5\linewidth]{background/AgeCovid.png}
    \caption{Statistic of lifespan distribution of COVID-19-related phishing content hosting sites when the sites are reported. Adapted from \cite{Unit42AtricleCovidPhishing2021}.}
    \label{fig:figSixteen}
\end{figure}

\subsection{Identification of Current Challenges}


Therefore, effective mitigation of DNS abuse requires the development of proactive measures that are continually updated in a way that adapts to the evolving landscape of cyber threats. Therefore, it is urgent to develop updated and evolving strategies with the evolving landscape. The cybersecurity sector needs to continue refining the tactics of the defence; otherwise, the bad actors will constantly refine their techniques to take advantage of DNS exposure. As for Internet and DNS abuses, since they transcend national frontiers, the only alternative to that serious problem is international cooperation. Now, the effectiveness of managing this kind of abuse would be through collaborative work between different nations, where experts in some geographical areas can come together and share their knowledge or resources \cite{altulaihan2022cybersecurity}. Among the many challenges would be the jurisdictions' differences related to the legal and regulatory frameworks. Therefore, reaching a consensus position is difficult in the case of regulations, standards, or enforcement actions. The next challenge that arises is that both need to be reduced, and that would include reducing the number of false positives and negatives to check DNS abuse. One has to find a balance because overly strict measures can reduce user experiences, while too liberal can cause less activity detection. This is not good news for the cybersecurity community, which should be gearing up even further to protect people as these bad actors' tactics only grow in sophistication. This would enable maintaining security, meaning that the good healthy state of the integrity of the DNS system will remain a factor towards the protection of this vital Internet infrastructure.


\subsection{Discussion on Future Research Directions and Technologies}


ICANN77 showcased further developments related to the mitigation of DNS abuse, including the drafting of changes that would require registrars and registries to respond to abuse notifications. In a legal response by Freenom, global abuse levels decreased. The ccNSO Domain Abuse Steering Committee argued for a proactive mitigation strategy, but gNSO analysis finds that EU ccTLDs have a very low abusive rate and considers its possible reason to be market maturity and non-profit models \cite{VanRoste2023}. Future activities will aim at building a new generation of tools with the help of new technologies, like artificial intelligence and machine learning, for enhanced domain security, to further support a larger global cooperation against shifting cyber threats. In other words, if better results are to be achieved, a high level of accuracy and wrong signals are not sent, more advanced AI and machine learning tools will be required, which will be able to understand the finer aspects of web traffic in detail \cite{ISG2023}.

\section{Summary of Findings}

Research on DNS abuse mitigation transparency points out how threats keep on changing, and therefore mitigation should also change. This is an indication of the value that community participation provides and the kind of transparency from which trust is built. Now, technological advancement is to be reviewed well, particularly in the field of AI and machine learning, to use for the detection of threats. Practical examples reflect the efficacy of various strategies. There is a balance between the introduction of new measures and the maintenance of effective teamwork and communication. Meanwhile, DNS abuse and subsequent staging should be effectively fought with a focus on such measures as technology, international cooperation, and standardisation of information exchange.





















