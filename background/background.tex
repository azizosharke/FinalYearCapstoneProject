\chapter{Background}
\label{Chapt2}


This chapter will explore the fundamental information relevant to this project, with an emphasis on the world of DNS abuse and transparency. It will include a detailed investigation of the domain name system (DNS), its function in the online community, and the variety of abuses it faces , the history of widely used policies and organisations aimed at mitigating DNS abuse, including a thorough examination of the DNS Abuse Institute and its achievements. A 'competition landscape' providing an examination of current market choices, from automated solutions to human tactics, will be provided as we navigate through the current methodology and technology deployed to mitigate DNS abuse. The reader will obtain a detailed understanding of the current situation of DNS abuse and the need for a more open, strong, and proactive strategy by analysing these various techniques and appreciating their strengths and weaknesses. This chapter emphasises the importance of the suggested solution in an era where digital authenticity is required, not only by providing information, but also by laying the groundwork for its presentation as a better and essential progression in the battle against DNS abuse.

\section{Understanding DNS \& Its Vulnerabilities}

The Domain Name System (DNS) is a significant part of the Internet infrastructure, serving as the key to converting computer-understandable IP addresses into human-friendly domain names. Although the DNS plays a vital role in maintaining ongoing online activities, privacy and security problems still arise. The ScienceDirect paper "Domain Name System Security and Privacy: A Contemporary Survey" provides a detailed analysis of these concerns that highlights the fundamental importance of DNS while illuminating the weaknesses that malicious actors may take advantage of \cite * {Sciencedirect2023dns}. There are a variety of security threats, ranging from DNS infrastructure targeting distributed denial-of-service (DDoS) assaults to cache poisoning and hijacking. Each of these attacks has the potential to do significant harm, including interruptions in service and the promotion of theft and spying. Due to the standard DNS design's lack of encryption, users' query data is vulnerable to abuse and eavesdropping, raising serious privacy problems. However, weaknesses do not mark the end of the story. In the same survey, new approaches are examined to improve DNS security and privacy. The use of DNSSEC (DNS Security Extensions), which authenticates DNS data and guarantees its integrity while repelling some types of attack, is an example of these advances in security measures. In addition, privacy-enhancing technologies are being used to encrypt DNS queries, preventing eavesdropping and manipulation, such as DNS over HTTPS (DoH) and DNS over TLS (DoT). The environment of DNS threats and defences is always changing in sync with the Internet. For systems to be robust and resilient, it is essential to understand these weaknesses and the continuous efforts being made to mitigate them. In this section, we provide an in-depth discussion of DNS vulnerability details, the effects of these safety concerns, and creative solutions that aim to bring in a new era of DNS security and privacy.

In a usual DNS lookup, three types of queries come into play to streamline the process and minimise the data journey. The first type is a recursive query, where the DNS client expects a direct answer or an error if the record cannot be found from the DNS server. Then there is an iterative query, which means if the server doesn't have the answer, it points the client to another server that might know, and the client keeps asking down the line until it gets an answer or hits a dead end. Lastly, a non-recursive query happens when the DNS server already knows the answer either because it is directly responsible for that piece of information or it has it saved from earlier inquiries. This method helps to reduce unnecessary internet traffic and reduce the load on the servers involved.

\section{Strategies \& Collaborations in Addressing DNS Abuse}

The DNS Abuse Institute, which will focus on DNS abuse to help increase safety and security through the domain name system, will be catered on these efforts to address DNS abuse with a comprehensive approach throughout the internet infrastructure. It helps the Internet community identify, report and mitigate DNS abuse in its mission to make the online environment more secure. Efforts by the institute, such as Compass Dashboards, provide vital data to registries and registrars that will enable proper decisions on combating DNS abuse. They show the commitment to transparency and education by issuing publications such as the "DNSAI 2022 Annual Report" or "DNSAI Bulletin 2023 04; Account Takeovers," which provide information on DNS abuse and how recommended mitigation practices \cite{dnsabuseinstitute2023}. Another such global strategy against DNS abuse has been contributed by the Internet Corporation for Assigned Names and Numbers (ICANN)\cite{icann2022dnsabuse} in collaboration with the entire DNS community, ICANN supports a synchronised method in the development of policies and standards on how to mitigate DNS abuse while ensuring the openness of the Internet. These participatory pillars hint at concerted efforts through policy development, technological developments, and stakeholder engagement as a central component in this collective approach to combating DNS abuse \cite{dnsai2022report}. 



\section{Different Forms of DNS Abuse}

DNS abuse takes many forms, each with its procedures and effects on users and the Internet as a whole. It is essential to understand these various pieces of evidence to create responses and regulations that work. This section will examine the comprehensive analysis of DNS abuse presented, describing the description, mechanism, and impact of each kind \cite{dotmagazine2022dnsabuse}.

\subsection{Phishing}
\begin{itemize}
    \item \textbf{Description:} Phishing is a technique aimed at deceiving individuals by creating website addresses that mimic those of companies, to trick users into revealing sensitive information such as login credentials, credit card numbers, or personal identification information \cite{webinarcare2023dnsstats}.
    \item \textbf{Mechanism:} This deception often occurs through emails or messaging services that direct users to websites similar to authentic ones \cite{jakobsson2006phishing}.
    \item \textbf{Impact:} Victims may suffer identity theft, financial fraud, and security compromise.
\end{itemize}

\subsection{Confusable Domains (Typosquatting)}
\begin{itemize}
    \item \textbf{Description:} Registering domain names that look visually similar to popular websites, taking advantage of typing errors or character similarities \cite{inta2023dnstypo}.
    \item \textbf{Mechanism:} Users may accidentally visit these websites when making a typo in a URL, which can expose them to malware or phishing attempts.
    \item \textbf{Impact:} Deception of users and potential harm to brand reputation \cite{edelman2008typosquatting}.
\end{itemize}

\subsection{Domain Hijacking}
\begin{itemize}
    \item \textbf{Description:} Unauthorised acquisition of domain names by exploiting security vulnerabilities in the domain registration system \cite{inta2023dnstypo}.
    \item \textbf{Mechanism:} Attackers may use tactics like social engineering, phishing, or exploiting security loopholes to gain control over a domain.
    \item \textbf{Impact:} Loss of control of the website, redirection to malicious sites, and potential data breaches.
\end{itemize}

\subsection{Botnets}
\begin{itemize}
    \item \textbf{Description:} Botnets involve controlling a group of computers infected with malware, used to carry out attacks or spread spam and malware \cite{citpyour}.
    \item \textbf{Mechanism:} Malware infects computers of unsuspecting users, incorporating them into a network under the attacker's control.
    \item \textbf{Impact:} Can result in large-scale DDoS attacks, mass spam campaigns, and widespread malware dissemination.
\end{itemize}

\subsection{Fast Flux Hosting}
\begin{itemize}
    \item \textbf{Description:} A technique used to conceal the location of websites associated with phishing and malware distribution \cite{lin2013genetic}.
    \item \textbf{Mechanism:} Involves a network of compromised hosts that regularly modify DNS records to evade detection.
    \item \textbf{Impact:} Makes tracking and shutting down malicious sites difficult.
\end{itemize}

\subsection{Domain Generation Algorithms (DGA)}
\begin{itemize}
    \item \textbf{Description:} DGAs generate domain names that act as meeting points for botnets \cite{antonakakis2012throw}.
    \item \textbf{Mechanism:} Malicious software uses algorithms to generate a sequence of domain names for command-and-control servers.
    \item \textbf{Impact:} Adds complexity to efforts to disrupt botnet command and control channels.
\end{itemize}
\captionsetup{font= footnotesize}
\begin{figure}[H]
\centering
\includegraphics[width=\textwidth]{background/DNSabuseForms.png}
\caption{Different Forms of DNS Abuse.}
\label{fig:figureThree}
\end{figure}



\section{How DNS Abuse Harms Users}

DNS abuse has serious and detrimental effects for both users and organisations, going beyond basic technological disruptions. Identity theft is among the most direct and direct effects. Phishing attacks, a common type of DNS abuse, use realistic websites to trick visitors into revealing sensitive data. Such attacks can produce information that results in financial theft, unauthorised access to accounts, and long-term damage to a person's reputation and credit \cite{godaddy2023dnsabuse}.

\subsection{Identity Theft}
\begin{itemize}
    \item \textbf{Phishing:} Phishing attacks often use domain names that imitate legitimate websites, fooling users into providing sensitive information such as usernames, passwords, or financial details, leading to potential identity theft.
\end{itemize}

\subsection{Financial Loss}
\begin{itemize}
    \item \textbf{Deceptive Transactions:} Users may be tricked into making payments to deceptive websites or unknowingly disclose their credit card information, resulting in financial losses \cite{bohme2013economics}.
\end{itemize}

\subsection{Data Breach}
\begin{itemize}
    \item \textbf{Malware:} Malicious software spread through compromised DNS systems can allow unauthorised access to corporate data, leading to data breaches \cite{fowler2016data}.
\end{itemize}

\subsection{System Compromise}
\begin{itemize}
    \item \textbf{Malware Infection:} Systems infected with malware due to DNS abuse can be exploited for further attacks, including the creation of botnets or the distribution of ransomware, resulting in system compromise \cite{saxe2018malware}.
\end{itemize}
\captionsetup{font= footnotesize}
\begin{figure}[H]
\centering
\includegraphics[width=\textwidth]{background/DNSabuseHarm.png}
\caption{How DNS Abuse Harms Users.}
\label{fig:figureFour}
\end{figure}


\section{Future Dangers of DNS Abuse}

As technology develops, so do bad actor strategies and tools, creating a dynamic environment for DNS abuse that could present new risks in the future. The sophistication of attacks has increased, which is a major issue. Bad actors are always creating increasingly sophisticated methods to take advantage of DNS, such as creating more convincing phishing schemes and using advanced virus distribution networks \cite{icann2022dnsabusetrends}.

\subsection{Increased Sophistication}
\begin{itemize}
    \item \textbf{Evolving Techniques:} Bad actors are constantly developing more sophisticated techniques to exploit DNS, such as advanced phishing schemes and malware distribution \cite{wrightson2014advanced}.
\end{itemize}

\subsection{IoT Vulnerabilities}
\begin{itemize}
    \item \textbf{Expanding Vulnerabilities:} The widespread adoption of Internet of Things (IoT) devices, which often lack robust security measures, presents a growing target for DNS-based attacks \cite{mahmoud2015internet}.
\end{itemize}

\subsection{Infrastructure Attacks}
\begin{itemize}
    \item \textbf{DNS as a Prime Target:} Attacks on DNS infrastructure can disrupt internet services on a large scale, including DDoS attacks targeting DNS providers or exploiting weaknesses in DNS protocols \cite{dooley2017dns}.
\end{itemize}

\subsection{Deepfakes \& AI}
\begin{itemize}
    \item \textbf{AI-Enhanced Phishing:} The use of AI technologies, such as deepfakes, has made phishing attacks more convincing and deceptive, manipulating audio and video content to impersonate trusted entities \cite{schick2020deep}.
\end{itemize}

\subsection{Cloud Computing Vulnerabilities}
\begin{itemize}
    \item \textbf{Targeting Cloud Services:} As organisations increasingly rely on cloud-based services, bad actors are exploiting DNS vulnerabilities to attack these platforms, potentially leading to data breaches and service disruptions \cite{mather2009cloud}.
\end{itemize}

\subsection{Mobile Device Exploitation}
\begin{itemize}
    \item \textbf{Mobile DNS Attacks:} The rising usage of mobile devices has led bad actors to target smartphones and tablets through DNS-based attacks, which can lead to data theft and the spread of malware \cite{au2016mobile}.
\end{itemize}

\subsection{Cryptocurrency \& Blockchain Exploitation}
\begin{itemize}
    \item \textbf{Crypto-Related DNS Attacks:} Attackers could exploit DNS vulnerabilities to redirect users to fake cryptocurrency exchanges or blockchain platforms, leading to financial fraud and theft of digital assets \cite{bashir2019advanced}.
\end{itemize}

\subsection{Political and Information Warfare}
\begin{itemize}
    \item \textbf{DNS in Cyber Warfare:} The manipulation of domain name systems can be used to spread misinformation or disrupt services during significant political events, serving as a tool for political and information warfare \cite{chapple2021cyberwarfare}.
\end{itemize}

\subsection{Exploiting Emerging Technologies}
\begin{itemize}
    \item \textbf{Abuse in New Tech Domains:} As new technologies such as 5G, AI, and quantum computing advance, tactics involving DNS abuse are likely to evolve, potentially leading to more sophisticated attacks \cite{brunner2021cybersecurity}.
\end{itemize}

\subsection{Supply Chain Attacks}
\begin{itemize}
    \item \textbf{DNS in Supply Chain Compromise:} DNS manipulation can also be employed as part of supply chain attacks, targeting software updates or cloud-based services to compromise organisations \cite{boyson2014cyber}.
\end{itemize}



\captionsetup{font= footnotesize} 
\begin{figure}  [H]
    \centering
    \includegraphics[width=\textwidth]{background/Future Dangers of DNS Abuse.png}
    \caption{Future Dangers of DNS Abuse.}
    \label{fig:LOLOLOL}
\end{figure}

By understanding these future dangers and emerging trends, stakeholders can better prepare and adapt their strategies to anticipate and counteract the evolving nature of DNS abuse.


\section{Foundational Mitigation Strategies \& Best Practices }

To address the broad nature of threats, mitigating DNS abuse requires an integrated strategy that integrates multiple strategies and best practices. The establishment of reporting and monitoring procedures is one fundamental tactic. Automated systems have the ability to track domain name registration patterns that may indicate DNS abuse, and protocols to report questionable actions can help ensure prompt intervention \cite{icannndnssec}. To confirm security and ensure that systems have not been compromised, regular audits of DNS configurations and domain registrations are also necessary \cite{lucas2021tls} .

\begin{enumerate}
    \item \textbf{Monitoring \& Reporting}
    \begin{itemize}
        \item Implementation: Use automated systems to monitor domain name registration for patterns that may indicate DNS abuse \cite{icannndnssec}. Establish procedures for reporting activities to authorities or cybersecurity organisations \cite{lucas2021tls}.
    \end{itemize}
    \item \textbf{Security Awareness Training}
    \begin{itemize}
        \item Implementation: Develop training programmes for users and IT staff with a focus on recognising phishing attempts, practising browsing habits, and understanding DNS security.
    \end{itemize}
    \item \textbf{DNS Security Extensions (DNSSEC)}
    \begin{itemize}
        \item Implementation: Deploy DNSSEC to ensure the integrity of the DNS data. This involves signing DNS records to protect against modification and DNS spoofing.
    \end{itemize}
    \item \textbf{Multi-Factor Authentication (MFA)}
    \begin{itemize}
        \item Implementation: Enforce multifactor authentication (MFA) for domain registrars and interfaces used to manage DNS \cite{icannndnssec}. This adds a layer of security beyond passwords, helping to prevent unauthorised domain transfers or alterations \cite{moghaddam2014ecco}.
    \end{itemize}
    \item \textbf{Blacklisting \& Takedown Services}
    \begin{itemize}
        \item Implementation: Collaborate with cybersecurity firms to identify and blacklist domains engaged in malicious activities. Establish response teams dedicated to removing domains involved in DNS abuse.
    \end{itemize}
    \item \textbf{Collaboration}
    \begin{itemize}
        \item Implementation: Foster collaboration among Internet service providers (ISPs), domain registrars, governments, and cybersecurity organisations. Share intelligence and best practices to collectively improve defence against DNS abuse \cite{skopik2017collaborative}.
    \end{itemize}
    \item \textbf{Regular Audits}
    \begin{itemize}
        \item Implementation: Conduct security audits of domain registrations and DNS configurations to verify their security and ensure that they have not been compromised \cite{coronado2014auditing}.
    \end{itemize}
    \item \textbf{Machine Learning}
    \begin{itemize}
        \item Implementation: Using AI and machine learning algorithms to analyse patterns in DNS traffic and proactively predict instances of DNS abuse \cite{icannndnssec}. This proactive approach enables the identification of threats before they materialise \cite{tsukerman2019machine}.
    \end{itemize}
    \item \textbf{Geo-Blocking \& IP Filtering}
    \begin{itemize}
        \item Implementation: Deploy geo-blocking and IP filtering techniques to limit access to DNS services from regions that have a history of DNS abuse. This can reduce the risk that attackers will use these services to carry out malicious activities or distribute malware \cite{meeseedited}.
    \end{itemize}
    \item \textbf{Enhanced Domain Validation Procedures}
    \begin{itemize}
        \item Implementation: Enhance the domain registration process by implementing validation procedures. This may involve verifying the identity of individuals or organisations that register domains, especially domains that resemble brands or fall into sensitive categories. By taking these measures, we can strengthen security and mitigate the risks associated with fraudulent domain registrations.
    \end{itemize}
\end{enumerate}


\captionsetup{font= footnotesize}
\begin{figure} [H]
    \centering
   \includegraphics[width=1.1\textwidth]{background/diagram (7).png}
    \caption{Mitigation Strategie.}
    \label{sadasdasdada}
\end{figure}

Each of these strategies plays a role in creating a comprehensive defence against DNS abuse. By integrating these tactics, organisations can establish robust, proactive measures to detect, prevent, and mitigate the ever-evolving threats posed by DNS abuse.

\section{Summary \& Synthesis}

After exploring the different forms of DNS abuse , How DNS abuse harms user , Future Dangers of DNS abuse and Mitigation Strategies and Best Practices. I have designed a table that has DNS abuses and the best possible mitigation strategies to help them against them, taking into account the transparency story behind it , user harm and reasoning. 


{

\footnotesize

\begin{longtable}{|p{2.5cm}|p{2.5cm}|p{4cm}|p{3cm}|p{4cm}|} 

\hline
\cellcolor{gray!50}\textbf{DNS Abuse } & 
\cellcolor{gray!50}\textbf{User Harm} & 
\cellcolor{gray!50}\textbf{Mitigation Strategy} & 
\cellcolor{gray!50}\textbf{Reasoning} & 
\cellcolor{gray!50}\textbf{Transparency Aspect} \\ \hline
\endfirsthead

\multicolumn{5}{c}%
{
\hline \cellcolor{gray!50}\textbf{DNS Abuse} & 
\cellcolor{gray!50}\textbf{User Harm} & 
\cellcolor{gray!50}\textbf{Mitigation Strategy} & 
\cellcolor{gray!50}\textbf{Reasoning} & \cellcolor{gray!50}
\textbf{Transparency Aspect} \\ \hline
\endhead

\hline \multicolumn{5}{|r|}{{\cellcolor{gray!50} Continued on next page}} \\ \hline
\endfoot

\hline
\endlastfoot
Phishing & \mbox{Identity Theft}, Financial Loss &  \mbox{Security Awareness} \mbox{Training, Enhanced Domain} Validation Procedures & \mbox{Training helps users} \mbox{recognize phishing} \mbox{attempts. Validation} prevents registration of mimic domains. & \mbox{Increases awareness and} \mbox{scrutiny during domain} registration. \\ \hline

\mbox{Confusable} Domains \mbox{(Typosquatting)} & Unauthorised Account Access & \mbox{Enhanced Domain} \mbox{Validation Procedures}, Regular Audits & \mbox{Prevents Registration} of Similar Domains. \mbox{Audits ensure} \mbox{compliance.} & \mbox{transparent domain} \mbox{registration process.} \\ \hline

\mbox{Domain} \mbox{Hijacking} & \mbox{System} \mbox{Compromise}, \mbox{Data Breach} & \mbox{Multi-Factor Authentication} (MFA), Regular Audits & \mbox{MFA secures domain} management. \mbox{Audits verify security} measures. & \mbox{Accountability in domain} management. \\ \hline

Botnets & \mbox{Malware} \mbox{Distribution} & Collaboration,Machine Learning & \mbox{Intelligence Sharing} \mbox{identifies botnet} \mbox{activities. AI predicts} \mbox{the formation of} \mbox{botnets}. & \mbox{Shared responsibility and} proactive detection. \\ \hline

\mbox{Fast Flux} \mbox{Hosting} & \mbox{System Infections} & Blacklisting and Takedown Services, Geo-Blocking & \mbox{Rapid response to} \mbox{malicious domains.} Restrict access from risky regions. & Responsive and transparent threat management. \\ \hline

\mbox{Domain} \mbox{Generation} Algorithms (DGA) & \mbox{Malware} \mbox{Distribution} & \mbox{Machine Learning, DNS} \mbox{Security Extensions} (DNSSEC) & AI detects abnormal \mbox{patterns. DNSSEC} \mbox{prevents spoofing.} & Integrity and trust in DNS data. \\ \hline

\mbox{IoT} \mbox{Vulnerabilities} & \mbox{Unauthorised} \mbox{Access, Data} \mbox{Breach} & \mbox{Security Awareness} \mbox{Training, Collaboration} & \mbox{Educates on security} \mbox{practices.} \mbox{Collaboration on best} \mbox{practices.} & \mbox{Open exchange of} \mbox{knowledge and efforts.} \\ \hline

Infrastructure Attacks & \mbox{DDoS Attacks}, \mbox{System Downtime} & DNSSEC, Collaboration & Protects DNS data integrity. Sharing of threat intelligence. & \mbox{Collective action}  \mbox{strengthens the DNS} infrastructure.  \\ \hline

Deepfakes and AI & \mbox{Identity Theft}, \mbox{Misinformation} & \mbox{Security Awareness} \mbox{Training, Monitoring} & \mbox{Recognising Phishing.} \mbox{Monitor} \mbox{AI threats.} & \mbox{Vigilance and prompt} \mbox{threat reporting.} \\  \hline

\mbox{Cloud} \mbox{Computing} Vulnerabilities & \mbox{Data Breach}, \mbox{Unauthorised} Access & Regular Audits, Enhanced Validation & \mbox{Secure DNS settings} \mbox{in cloud services.} \mbox{Prevents exploitation.} & \mbox{Framework for secure} \mbox{domain use in cloud.} \hline

\mbox{Mobile Device} Exploitation & Unauthorised Access, Financial Loss & \mbox{MFA, Security Awareness} Training & \mbox{Secures account} \mbox{access.} \mbox{ Raises awareness} of threats. & Mobile security awareness and protection. \\ \hline

\mbox{Political and} Information Warfare & Misinformation, Political \mbox{Manipulation} & Monitoring, Collaboration & \mbox{Monitoring abuse in} \mbox{campaigns. Unified} \mbox{response to } \mbox{misinformation.} & Transparency in monitoring and collective action. \\ \hline

\mbox{Exploiting} Emerging \mbox{Technologies} & system \mbox{Vulnerabilities} &\mbox{ Machine Learning,} \mbox{Collaboration} & \mbox{Analytics to predict} \mbox{DNS abuse. Share} \mbox{knowledge about} \mbox{threats.} & \mbox{Innovation in defense}  \mbox{strategies and sharing.} \\ \hline

\mbox{Supply Chain} \mbox{Attacks} & \mbox{System} \mbox{Compromise,} Data Breach & Regular Audits, Blacklisting & \mbox{Audits for DNS} \mbox{integrity. Rapid} \mbox{response to threats.} & \mbox{Transparency in supply} \mbox{chain security.} \\ \hline

\caption{Mitigation strategies against DNS abuse and its impact on users.} 

\end{longtable}

}


Finally, this chapter has examined all aspects of DNS abuse,  the various forms, the serious harm it does, as well as potential future threats. To create efficient regulations and countermeasures, it is essential to understand the extent and consequences of DNS abuse. Significant progress towards resolving these issues has been made by organisations like the DNS Abuse Institute and ICANN. However, as new technologies are incorporated into the equation and the threat environment changes in sophistication, it becomes increasingly important to adopt alert, flexible and cooperative strategies. The mitigation techniques and best practices discussed in this chapter provide a roadmap for mitigating DNS abuse. Every tactic contributes to a defence mechanism, from advanced technology solutions and improved methods for validation to monitoring and reporting. It is impossible to overestimate the value of cooperation, regular checks, and the application of cutting-edge technologies to anticipate and mitigate DNS abuse. After analysing the data, it is evident that a team effort is needed to comprehend, track, and mitigate DNS abuse. A complex strategy that integrates multiple techniques and encourages collaboration across industries is required instead of a single insufficient strategy. Our approaches to preserving the integrity and security of the DNS and, consequently, the larger Internet infrastructure must adapt, as does the digital environment.

By understanding the connections between different aspects of DNS abuse and reinforcing the collective effort required for effective mitigation, stakeholders can be better prepared to face the challenges ahead. This chapter sets the stage for further research and action, with the aim of contributing to a safer and more secure digital world.



\chapter{State of the Art}

This chapter explores the strategies used to mitigate DNS abuse and new developments in this field. Explore and evaluate the effectiveness and transparency of multiple mitigation techniques, including DNS filtering and threat intelligence, in which experts organise and analyse information about cyber attacks. Additionally, the use of domain-generating techniques and DoT and DoH are two novel forms of DNS abuse that are highlighted in this section. In addition, the role of AI and machine learning in identifying and mitigating DNS abuse is covered. The final half of the section includes a discussion on potential future research areas and technologies to improve DNS abuse mitigation. Case studies offer practical information on DNS abuse occurrences. 


\section{Current Strategies and Their Effectiveness in Relation to DNS Abuse}


DNS abuse presents a significant challenge for Internet entities involved in domain name management. Various approaches are employed to mitigate such abuse, including DNS filtering, which regulates access to specific websites and prevents you from accessing malicious sites that can administer phishing and ransomware. Additionally, threat intelligence methodologies use data analysis to identify potential risks, as exemplified by \cite{schmid2021thirty}. Anomaly detection plays a role in identifying suspicious DNS activities indicative of malicious intent using Packet Analysis to analyse individual packets for DNS allowing for real-time detection and statistical analysis, which involves performing statistical analysis on a large dataset of DNS traffic. However, these methods can face operational challenges, such as errors and the need for fast access to critical threat data. 

\subsection{Transparency in DNS Abuse Mitigation \& DNS Relevance}

\begin{enumerate}
    \item A Case Study of Cloudflare's Transparency Approach


Cloudflare is firmly committed to transparency \cite{cloudflare_transparency_2022}, the cornerstone of its relationship with customers, which guides its approach to DNS abuse reports and requests that may come from law enforcement. This reduces their actions and policies to mould a trustworthy environment while addressing internet safety and privacy concerns. Their approach to handling DNS abuse reports and law enforcement requests is grounded in three core principles:

\begin{enumerate}
    \item Require Due Process:  Whatever shall be lawfully requiring due process of law enforcement and Cloudflare shall adhere in letter and spirit. They are neutral in behaviour and do not intend to hinder or facilitate law enforcement efforts more than is required by law.
    \item Respect privacy: At Cloudflare, privacy is very important. They assure customers that anything of personal nature shared by them remains private and protected. The company makes a commitment not to sell, rent, or disclose personal information without specific and unambiguous consent from the individual, applying this policy to commercial and government or law enforcement requests.
    \item Provide Notice: Per the CloudFlare policy, they undertake to provide notice to any of their customers in case a subpoena or other legal process issues for customer or billing information relating to the use of its network, unless otherwise such disclosure is otherwise not permitted by law. This is to ensure that individuals and organisations are made aware before theirs can be distributed.
\end{enumerate}

The Cloudflare Transparency Report for the latter half of 2022 gives deep statistics and trends based on DNS abuse reports over Cloudflare's response. Highlights

\begin{enumerate}
    \item Abuse Reports: Cloudflare avidly responds to various abuse reports and has shown an enthusiastic commitment to maintaining a clean and lawful network. Some reported types of abuse include phishing, malware, and content that violates copyright laws, among others.
    \item Actions taken: Cloudflare not only reserves the right to review, accept, or decline clients, but also ensures decisive actions against reported abuses by terminating hosting services from the domains taking part in technical abuses, such as phishing or any malicious activities. Such terminations are not limited to actions taken by content-based abuse and are handled differently.
    \item Termination of services: Cloudflare suspends services to domains that do not take action to remedy reported instances of CSAM (Child Sexual Abuse Material) or are otherwise dedicated to distributing such material. Last year, in just the second half of 2022, alone, Cloudflare suspended service for 206 accounts and 530 domains connected to CSAM.
    \item IPFS and Ethereum Gateways: If a valid abuse report is received in regard to copyright, technical sanctions compliance, or otherwise, Cloudflare reserves the ability to disable access through its operated gateways to content on IPFS and the Ethereum network. 99 actions were taken on Ethereum gateways and 1142 for IPFS during the second half of 2022.
    \item UDRP Requests: 21 UDRP (Uniform Domain-Name Dispute Resolution Policy) responses resulted from verification requests to Cloudflare by an ICANN-approved dispute board in the second half of 2022, further illustrating its commitment to response in such legitimate concerns regarding domain name disputes.
    
\end{enumerate}

In addition , Cloudflare's careful description of compliance and due process with respect to handling law enforcement requests comes from their latest Transparency Report.
Below is a summary of the major areas covered.

\begin{enumerate}
    \item Legal Sufficiency Review: Each request is reviewed by Cloudflare for legal sufficiency before processing. This may range from ensuring compliance with necessary processes to all that is practically feasible within the purview of law to meet the need. They respect and safeguard the privacy of users and provide customer information to written requests from law enforcement that are validly issued based upon laws with valid legal process such as a subpoena or court order.

 \item Respect to International Privacy Laws: Cloudflare recognises the potential conflict of privacy laws of different countries, and when they receive requests from government, they legally challenge any request for data that is conflicting with the privacy laws of the country where the user stays.

 \item Emergency Disclosure Requests : Cloudflare takes very seriously all emergency disclosure requests. They may therefore make such disclosures to law enforcement without legal process when there appears to be an imminent danger of death or serious physical injury and requests that law enforcement obtain legal process when time permits, therefore ensuring that the use of emergency disclosures remains a carefully controlled exception.

 \item  National Security Requests and Non-Disclosure Obligations: Cloudflare has made a lot of effort to challenge FISA court orders or National Security Letters (NSLs) in case they feel that the company received one with which their desire for transparency or releasing transparency reports cannot be met. In this regard, there was a period in which the company fought legal prohibitions to report the receipt of NSLs, indicating its attitude of fighting for transparency and user privacy.

 \item  International Requests for Data: In the case of requests emanating from governments outside the United States, Cloudflare again evaluates them with strict adherence. The company responds to requests issued through U.S. courts by way of diplomatic processes like mutual legal assistance treaties (MLATs) and evaluates other international requests on a case-by-case basis. These include an analysis of local law, the request's compliance with international norms, and company policy.

 \item  Challenging Overly Broad or Inappropriate Requests: Over time, Cloudflare has long stated that it will challenge any law enforcement requests that are overbroad or issued wrongly and that act as an obstacle to their transparency with users. ., provided that due-procedure requirements are met or that the exercise is intended to protect user rights in any request they may receive in or outside the USA.
\end{enumerate}

Public reporting by Cloudflare and working closely with law enforcement, as well as other partners, form important elements in its strategy of mitigating DNS abuse such as: 

\begin{enumerate}

\item Reporting to the Public \& Transparency: Cloudflare maintains a high level of transparency in its reporting with regard to the types and volumes of abuse reports it receives and the measures that are put in place. This supports the creation of trust among clients and partners, demonstrating action in the fight against abuse.

\item Law Enforcement Cooperation: The report shows how Cloudflare interacts and cooperates with many law enforcement agencies in the most approachable manner and without touching on user privacy. It enables careful consideration of such a request for any action to be legally justified, and, by so doing, contributes to general mitigation efforts of DNS abuse.

\item Mitigation Actions: Cloudflare has taken affirmative action against DNS abuse. These actions include, but are not limited to, terminating such services when knowing of domains that are used for phishing, distributing malware, and performing other activities that would harm a greater world. The termination of the access is done on content at the many different access points provided by Cloudflare, including any relating to abuse reports and, indeed, including IPFS and Ethereum gateways. This shows that the company is serious about mitigating DNS abuse.

\item Challenges to Preventing DNS Abuse: Although Cloudflare provides these tools, the report still refers to the challenges associated with reducing abuse. The struggle for balance between protection and abuse of free expression, legal and technical challenges when reacting to abuse reports, and from what kind of cooperation between key shareholders are, it is underlined, ongoing challenges.

\item Efficiency of Efforts to Mitigate DNS Abuse: Cloudflare transparency practices, through the half-yearly publication of transparency reports, lend a hand in acquiring insights into the mitigation of DNS abuse. This clearly shows their commitment and forward-leaning policy to minimise problems related to DNS abuse. However, its efficacy also depends on the broader ecosystem's capacity to solve the initial cause of this DNS abuse, an undiversified market where most other options for hosting are very limited.

\end{enumerate}

Some of the challenges with which Cloudflare is confronted in its transparency efforts and in mitigating DNS abuse are mentioned in the Transparency Report. They include such matters as the complexity of DNS abuse, keeping the fine balance between transparency and privacy, legal/regulatory compliance, and limitations of technical ability in mitigating the misuse while keeping the fine line. With these insights in mind, the following are recommendations that Cloudflare could use to identify potential enhancements in its processes:

\begin{enumerate}
    \item Enhanced Cooperation with Stakeholders: Cloudflare will enhance cooperation with law enforcement, other service providers, and international organisations to exchange views on best practices and come up with standard operational procedures on how exactly they will address DNS abuse. Joint efforts reduce identification time and intensify the ability to mitigate abuse throughout the Internet ecosystem.
    \item Improve Abuse Detection Systems: Continuous investment in the best technologies and machine learning algorithms will improve abuse detection and enhance its ability to respond to DNS abuse. Better detection will be less time-consuming in identifying and bringing down abusive content, therefore improving the entire internet safety concern as a whole.
    \item Transparency Reporting Enhanced: The reports on transparency from Cloudflare are simple to understand, yet they need more details about the identification of types of abuse faced by the domain name system and evaluate the process with respect to checking its effectiveness on all counts for mitigation. It will keep stakeholders up to date by providing much more details when it comes to trend and pattern assessment in regard to abuse, which will lead them in the process to fine-tune the directions of best practices for abuse mitigation.
    \item Better User Education \& Awareness: Cloudflare would be in a position to prepare more materials and programmes to educate its users about cybersecurity and the risks of DNS abuse and what they should do to protect it. Enhanced user engagement in these can help build an enhanced internet environment.
    \item Advocate Policy and Legal Reforms: Cloudflare can do more to try to advocate for policies that will potentially be challenged in various legal jurisdictions and cause a potential conflict of privacy laws against law enforcement requests. In such a push for already formulated laws and put-in-place policies to balance user privacy against those interests supporting efforts in fighting DNS abuse, an improved offer may be realised. This policy helps offer protection against abuse, or even support for more coherent and efficient internet governance.
    \item Create a Multi-stakeholder Feedback Mechanism: A mechanism can be framed that ensures feedback from users, civil society, and other stakeholders that would indicate how far Cloudflare has been successful in its transparency efforts and reducing abuse. Thus, such suggestions can then guide any subsequent policy revisions or enhancement of organisational policy.
    \item Continue to Challenge Over-broad Requests: Cloudflare's willingness to continue fighting even with over-broad or inappropriate requests for user data in place remains praiseworthy. The possibility of being able to further prioritise the user and due process amongst this sort of situation implies some more badge of trust and a role model for the industry.

    
    
\end{enumerate}

In conclusion, the company emphasises its commitment to protecting legal processes and user privacy while navigating government and law enforcement requests. A critical aspect of these reports is Cloudflare's approach to DNS requests, particularly regarding content blocking through its 1.1.1.1 Public DNS Resolver.This was the key answer: Cloudflare, in no uncertain terms, "received legal requests to block content at our DNS servers" and stated its policy to first "exhaust legal remedies" that they could enforce. This is an indication of how very carefully Cloudflare has to adhere to the demands of the law, yet protect the openness of the Internet, bringing out just how major DNS is in all matters that pertain to the accessibility of content on the Internet and governance of the Internet.

\item Google Transparency Reports 

It has become evident that there lies a more dynamic in nature relationship between the governments of the world and internet governance, specifically through requests for removal of content in the Google services. In light of this, the function of the Domain Name System comes up as one of the mechanisms that are critical in realising how the requests can be translated into actions. The relevant data, for example, for Russia, contain tens of thousands of items to be redacted. Massive redaction requests, such as this one, go very far beyond the issue of focused content take-down and indicate potential for far broader action up to and perhaps including that which would be taken on the DNS level and through other means that may be ultimately settled here on the domains to be held in enforcement. Such instances further highlight all the more the role of DNS in enabling access to or blocking content on the Internet while serving much more effectively as the gateways through which governments indeed wish to exercise control, for which legal and regulatory pressure is employed so often on large technology companies like Google.

Further, the queries clarify the relevance of DNS; they do not directly mention "DNS manipulation", but the phrasing points to some kind of 'how-to' on technical compliance, which could also be the making of DNS changes. The compliant removal requests that Google yield indicate technical mechanisms that may be in place to comply with government mandates and most likely affecting how DNS resolves to certain domains or URLs. This indirectly points to the DNS as a critical infrastructure within the larger debate on Internet governance, censorship, and access to information. In light of that Google Transparency Report, it becomes very telling that DNS clearly breaks through this legal and policy structure not only as an underpinning element to the architecture of the Internet, but as a very hotly contested space to control both digital content and information flow \cite{Google2023}.

\item Amazon Transparency Reports 

Necessarily, such a role of DNS in servicing governments or other legal data demands does not trace directly to specific acts of manipulation in the DNS or intervention at the domain-level. The report explains about Amazon's observance of due process laws in handling requests for data such as subpoenas and search warrants, with a lot of emphasis on customer privacy and protection of data which can be mounted against the state or any other third party institution or person. It goes without saying that handling the domain or the services to do with this website means that a possibility of such a move like DNS changes can be in the offing. However, they do not give clear examples where DNS interventions have been taken, but describe the circumstances related to legal compliance and internet governance without direct reference to DNS \cite{Amazon2023}.

\item DNS- SB Transparency Reports

xTom reported nil compliance, for the most part, within the international statistics of content data requests, requests for information on subscribers, requests to have content taken down, requests to have content blocked, and domain name dispute resolutions in 2020. These zero compliances are placed to highlight the fact that the organisation, in reality, sets the protection level of user data and content integrity too high, which is part of a general position on how DNS and domains are managed for the protection of users and the achievement of operable thresholds \cite{DNSSB2023}.

\item The CyberGhost Transparency Report

An obvious upward trend of the recursion without DMCA complaints, along with flagging malicious activities, flashes up in each year, record by record, before a sudden spike around 2023. Given the growing level of claims and requests, CyberGhost still regards the No Logs policy as a strong sweat so they keep a keen eye and hence stays guardedly strong on the user's privacy and any request relating to DNS. The report is categorical with such an idea that even in the case of mitigating malicious activity, they do not involve logging of DNS queries or respective user activity; therein, the integrity of user data and an assurance towards compliance in privacy. DNS somehow plays a function in this case: It becomes evident that the design of the CyberGhost infrastructure is supposed to be resistant to infiltrators and, hence, capable of withstanding invasions and pressures in no less than those that would compromise an individual's anonymity and right to freely receive information via the Internet \cite{CyberGhostVPN2023}.

\item The Meta-Transparency Reports

This will also touch on the enforcement of intellectual property on social media platforms like Facebook and Instagram, and usher in overall holistic measures to combat copyright, counterfeit, and trademark violations. The DNS is important to these functions in these two aspects. First, with regard to it being an underpinning of the distribution of information online and, in another sense, a checkpoint in the process of enforcement. For instance, Facebook removed 447,123 pieces of content on copyright grounds and on Instagram, 297,356 in the first half of 2022. When these volumes are taken in such high volumes, one would easily conclude that beyond the platform level of moderating content, other interventions had to be made at the DNS level. Those could vary from steps like de-indexing websites from search, to editing DNS records in such a manner that requests to domain names of abusive sites are not resolved or that access to infringing content is denied.

Results since the second halves of 2020 and 2021 seem to suggest that the rates of removals have been self-sustaining, due to the mechanisms of DNS dependency. Last year, in 2020, Facebook stuff removed 432,854 pieces of content for copyright reasons, but this number decreased to 273,325 counterfeit items removed in 2021. This is a huge amount, proving that if something was taken down, then Meta has not only removed nasty content from offers, but most likely reached an agreement with DNS providers too, to not allow access to offending domains. This clearly elaborates on the integral part of the DNS in enforcement; hence they are used in upholding intellectual property rights, effectively reducing the spread of counterfeit goods, and protecting the interests of creators and owners of the trademarks \cite{Facebook2023}.

\item  T-Mobile Transparency Report

It outlines how the company complies with directions of the law in the management of requests for information from consumers, thus highlighting staying within customers' privacy and legal compliance. Details the approach and policies of the company in response to lawful requests on records of customers within T-Mobile, Metro by T-Mobile, and Sprint, now collectively T-Mobile USA, Inc. (TMUS). At the same time, it provides information about what TMUS does to protect consumers from unauthorised data access, including first-party requests made by the company itself, such as subpoenas, court orders, and warrants, with all processes required following the same. When sharing details surrounding the number and types of requests received in 2022, the report marks a heavy emphasis on TMUS’s efforts to take care of customer privacy and complying with applicable legal obligations. In the case of T-Mobile, it handled 301,388 subpoenas, mostly related to orders to disclose information about the subscriber, such as names and addresses, and 94,599 different types of warrants or search warrants, which can be after historical location data or the content of messages \cite{TMobile2022TransparencyReport}.

\item IBM 1H 2021 Law Enforcement Requests Transparency Report
 
 It shows how IBM ethically handles data and is transparent about it. Building on a tradition of over a century, a new standard in earning client trust, IBM sets the principles that will dictate IBM’s management of client data and places an emphasis on client data ownership and promotion of fair and non-discriminatory government policy toward data. IBM: The finalisation of this report was to clarify that IBM, under no circumstances, has been handing over their clients' data to any government surveillance under surveillance programmes involving bulk collection and any other surveillance programme for that matter. It underlines the IBM policy related to compelling governments to work directly with enterprise clients in relation to data requests and adherence to rigid guidelines through legal routes including Mutual Legal Assistance Treaties (MLATs) with respect to international data requests. IBM received a total of 27 law enforcement requests during the period from January 1, 2021, to June 30, 2021, two of which were requests for information on an IBM account, and all information was accepted. The information pursued was, for the most part, basic subscriber contact information, such as a name, email, and business address that would allow law enforcement to contact directly with our customers. The report strongly emphasises a key IBM principle: that the customer owns their data and that requests for those data are matters of great gravity and extreme rarity. Meanwhile, within that time, there was no request for any data of the clients that was at all met, which clearly reflected that IBM has a stricter policy regarding the ownership of the data of the clients and less participation in providing specific or private information to investigative bodies. This approach shows the commitment of IBM to unwavering client privacy and data protection amidst legal and governmental inquiries \cite{IBMTransparencyReport2023}.

 \item Trade Me Transparency Report

This report shows the ongoing commitment to transparency and openness of Trade Me, which gives a breakdown of its interactions with New Zealand government agencies. This is the 11th annual report by Trade Me and builds on last year's report of walking this tightrope of how they balance legal compliance with protecting their member's privacy. By being a step ahead in proactively sharing information with government agencies on data releases, Trade Me sets a tone with transparency and what they believe in, spilling the beans on the intended use and release of member information. There is also emphasis on the cumbersome process the Trust and Safety staff go through with regard to ensuring the relevancy and coverage by the law of such information released, definitely with good intentions of keeping the community safe within the laws. This ensures the confederation of trust among Trade Me's members and guarantees a safer online community. What stands out in this report is the clarity of numbers on requests and releases of member data to government agencies for the periods between July 1, 2022, and June 30, 2023, indicating how transparent Trade Me operates. For example, it states that there is a 36\% year-on-year decrease in voluntary releases of information under the Privacy Act to the New Zealand Police. This signals the carefulness of the organisation towards the exposure of information \cite{TradeMeTransparencyReport2023}.

\item Xiaomi Transparency Report: Government Requests for User Information (2022)

This Report is a reflection of how the company treats the requests of the government when it comes down to the user-related data. In a statement by Xiaomi, the company shows that it had been straightforward in handling judicial, enforcement, and other government data requests. These are implemented by industry standard technical and organisational guidelines and observing the law and regulation of the world. The report hereafter is a general review of transparency by Xiaomi when it comes to requests from different governments in as far as users' data are concerned from device-based to financial-identifier-based to account-based, reflecting the bottom line and the bedrock principle upon which Xiaomi built trust with consumers in relation to privacy and data protection. In 2022, the government of India reached out to most companies, with at least 51 devices-based requests for Xiaomi Inc., which involved 49,683 devices. More specifically, 32 out of these requests were complied with by Xiaomi, which means that 62. 75\% was the compliance rate in that country. Such data here talk about the extent of inquiries made by the government against Xiaomi in regions with a strong operational presence and give a perspective on what kinds of questions were posted \cite{XiaomiTransparencyReport}.
 
\item eBay Global Transparency Report 

 This report highlights eBay's commitment to a safe and trusted experience for our global buying and selling community. "Safety" is a word the company takes quite seriously, and the report articulates the ongoing work of eBay to secure its marketplace from fakes, fraud, and other abuses. With sophisticated AI technologies and image detection, combined with recent efforts, eBay continues with its commitment to everything from proactively identifying and mitigating potential threats, soon to be announced, which include better collaboration with rights owners and law enforcement agencies. This supports eBay investments in technology, partnerships, and other efforts that support that investment and hence maintain the platform's integrity on so many levels. Reflecting back on policy, procedure, and the far-reaching impact of eBay initiatives from 2002 through 2022, at its core, the report enshrines the company's foundational philosophy: that through transparency, a marketplace of economic opportunity is created for people around the globe. In 2022, eBay proactively removed a massive number of harmful listings. To put that into perspective, precisely 295 million prohibited items listed on eBay were blocked by eBay's AI tools, putting into question exactly how instrumental the technology at that company has been in damaging controlled substances. The unbelievable post-sale services on luxury watches, handbags, jewellery, sneakers, trading cards, and much more through the Authenticity Guarantee programme on eBay \cite{eBayGlobalTransparencyReport2022}.

 \item Cisco Transparency Report: Government Data Demands (First Half of 2023)

 This report shows the public declaration of the company's commitment to open up about requests made to it by different governments for customer data within various jurisdictions across the world. This semiannual report, of indispensable importance in knowing the landscape of data privacy and government surveillance, will include the nature and volumes of requests that come to Cisco, the kind of task, and demands for both content and non-content data. Publishing how many requests were compiled, rejected, and met with no data found, Cisco stands by its principled approach to balancing legal obligations with customer privacy. In particular, the report indicates national security demands in the view of the United States and that Cisco complies with explicit legal frameworks such as the United States Freedom Act of 2015. In that same reporting period, there were 16 demands for noncontent data to Cisco from US government agencies, 7 of which resulted in data disclosure. This reflects an approximately 44\% compliance rate in which data was indeed disclosed in NCDRs by Cisco after "MIND" demands in which no information was found, or demands were rejected. Such statistics are important to be aware of in that they help paint a picture of how a tech company, like Cisco, interfaces with government requests for data, supplying a view of the balance that exists amidst governmental interests versus privacy rights. Furthermore, international requests showed a total of 27 requests from Germany to Cisco with 25 disclosures, fostering the idea of how much government interest there is among the governments of the world for data \cite{CiscoTransparencyReport}.

 \item Apple Transparency Report : Government \& Private Party Requests 

 It involves a summary containing both legal requests from government agencies worldwide and US private parties. Shows how careful Apple tends to be with the data it holds from its users by defining requests according to devices, financial identifiers, accounts, and types of requests. It is reflective of a very tiring process through which Apple passes each and every request to ensure that it is within the laws and just how extremely committed the company is to user privacy and information security. Therefore, it is high that it is important for Apple to be transparent and allow all these requests to form an opinion about the operations and trust in it. This report is required to be read by those who may wish to understand the relation of technology, privacy, and law enforcement in the digital era. The report outlines the types and amounts of requests available. For example, Apple received 5,660 requests of devices from the US itself, 82\% were met with data provision. For the most part, such requests arise in the investigations of lost or stolen devices and fraud inquiries. Similarly, for account requests, Apple in the US received 7,944 requests; for the last reporting period, data was provided in 47\% of the cases \cite{AppleTransparencyReportGB}.


\end{enumerate}

\subsection{Effectiveness of Current DNS Abuse Mitigation Strategies}

Different methods are used to mitigate DNS abuse, including the implementation of blocking tools, awareness of potential threats, and identification of anomalous behaviour. DNS filtering entails the regulation of website access based on predetermined rules, which can have varied outcomes depending on the context in which it can happen in different environments such as register and registry in which it implements mechanisms to compare DNS names to the block list and given set of rules then takes the necessary action such as homograph attacks in which DNS filtering mechanism play a role in mitigating them by comparing domain names against block lists and predefined rule to identify potentially malicious homographs as stated earlier. Threat intelligence plays a role in identifying potential dangers and detecting unusual activities within the DNS, as noted \cite{rizvi2022application}, such as allowing proactive identification and assessment of potential threats and malicious activities, including detecting patterns indicative of phishing, domain hijacking, malware distribution, and other forms of DNS abuse. Evaluating the effectiveness of these methods requires careful consideration of their performance in real-world scenarios. For example, while DNS filtering can effectively block malicious content, it may inadvertently permit harmful elements to bypass the filtering process, potentially impacting the user experience. Similarly, the effectiveness of threat intelligence relies on the timeliness and accuracy of the data used. However, identifying anomalous behaviour poses challenges, as distinguishing between malicious actions and legitimate activities performed in innovative ways can be challenging.


\section{Emerging Trends in DNS Abuse}

Trends in DNS abuse had declined among some categories, such as botnets, malware, phishing, and spam. Much of this decline could be attributed to the multipronged approaches that ICANN itself launched around data analysis, community tools, and enforcement of registry and registrar obligations \cite{icann_dns_security_threat}. Although continuing to be slow, adopting organisations did so under the compulsion of situations that left them no choice but to use technology or by those for whom TLS adoption was a matter of technological innovation, choice, or desire for the embrace of technologies simpler and more robust from misdirection. One of the major issues has continued to be privacy, due to the fact that DNS queries have been accidentally found to give away user behaviours. One such move to enhance user privacy is the Query Name Minimisation. The main concern has been how to remain vigilant against DNS abuses while improving privacy without altering service efficiency.

\subsection{Evolving New Forms of DNS Abuse}

The field of cybersecurity is rapidly advancing, bringing forth new challenges as it evolves, and constantly moving the goalposts for defence mechanisms. The introduction of DNS over TLS (DoT) and DNS over HTTPS (DoH) is like a double-edged sword. Although these encryption protocols were designed to enhance privacy and security by encrypting DNS queries, they unintentionally provide attackers with means to disguise malicious traffic. This expands the attack surface, affecting everything from individual devices to corporate networks. For example, attackers could take advantage of DoT and DoH in enterprise settings to avoid outdated security controls and establish hidden communication channels. Furthermore, Domain Generation Algorithms (DGAs) play an important role in cyber threats by automatically generating a large number of random domain names, making it extremely difficult to identify and shut down malicious sites. \cite{kaur2023artificial}. This tactic, integral to botnet command and control (C2) operations, significantly complicates cybersecurity defence efforts to predict and mitigate threats.

The adoption of DoT and DoH offers several benefits, such as enhanced privacy by preventing the surveillance of DNS queries and improved security through the encryption of DNS traffic, which weaken hackers' attempts to intercept or manipulate data. However, these protocols also allow attackers to hide their malicious activities, which poses challenges for traditional DNS security systems in detecting and filtering harmful content. Furthermore, these protocols could accidentally bypass content filtering policies, leading to potential security breaches within organisations.
Conversely, DGAs provide attackers with a method to evade detection and maintain C2 communications, as the dynamically generated domains are difficult to predict and preemptively block. This results in an overwhelming number of domain names for security mechanisms to monitor, complicating the threat intelligence process and necessitating continuous vigilance and blacklist updates. The widespread adoption of these technologies underscores the need for cybersecurity professionals to adopt a proactive and informed stance, understand their potential for exploitation, and develop comprehensive strategies. These strategies must strike a balance between the benefits of encryption and domain generation and the imperative to prevent DNS abuse, ensuring the integrity and security of the online environment.


\subsection{Predictive Measures \& Their Transparency}

Efforts to mitigate DNS abuse are set toward immediately slowing such activities by utilising complex systems and advanced machine learning algorithms to detect patterns indicative of DNS abuse. Articulating and sharing insights about the decision-making processes in predictive modelling is considered significant, as well as the efforts by registrars and registries, acting together, in the context of DNS Abuse Transparency are comprehensive. These entities will invoke a wide range of mitigation measures to minimise damage and losses related to DNS, which will ensure the development of a more secure and trusted Internet environment. Some key mitigation strategies are account-based remediation in the way that maliciously generated accounts are locked out and further validated, in addition to monitoring third-party feeds and reports from cybersecurity organisations, law enforcement, and the public to discover and address abuse early. Moreover, this mitigation involves malware analysis, which comes from attacks on the communication infrastructure and the corresponding IP addresses, through suppression or sinkholes in the context of botnets and the use of domain generation algorithms (DGA) that direct botnet traffic \cite{ M3AAWG2024}. Most specifically, sinkholing is an authoritative measure that directs traffic from abusive domains to harmless servers and allows studies to be conducted on the sources of traffic and the extent of compromise. Compliance with legal and contractual requirements further underscores the actions of registrars and registries against DNS abuse, ensuring that their actions in mitigation are within the context of the ICANN agreements and local laws. 

The evident evaluation of real-time black hole lists (RBLs), in addition to the responsible role of trusted notifiers, further increases the effectiveness and accuracy of mitigating actions, to filter and validate reports on abuse, so that proper responses may be made. This multipronged approach on the part of the registrars and the registries towards the mitigation of DNS abuse does not only emphasise the proactive and reactive measures, but also the possibilities of increased transparency as far as reporting and publicising the actions in place against DNS abuse are concerned. Such transparency is key to building trust, open to accountability, and creating an environment conducive to stakeholders' collaboration for the more effective fight against abuse in the DNS ecosystem. This transparency helps to understand the rationale behind the predictions, map the data used for model training, and clarify the methods that guide decision-making, as highlighted in \cite{hussain2022software}. Striking a balance between the complexity of predictive models and their interpretability is a significant challenge. Therefore, it is essential to approach this challenge with caution, ensuring that the models are not only effective in identifying DNS abuse but also accessible for thorough examination and accountability.


\captionsetup{font= footnotesize}
\begin{figure}[H]
\centering
    \includegraphics[width=1.0\linewidth]{background/DNSECO.png}
    \caption{DNS Ecosystem Contractually Related to ICANN (image
courtesy of Verisign and originally published in SSAC 115 adapted from \cite{SSAC2023SAC115})}
    \label{fig:fig14}
\end{figure}
\clearpage

\section{Technological Advancements}

The mitigation of DNS abuse is increasingly influenced by the integration of artificial intelligence (AI) and machine learning technologies \cite{goethals2021enabling}. At the helm of this evolution are innovative tools such as the iQ Domain Risk Score, which employs machine learning and string analytics to proactively detect potential domain abuses now of registration \cite{dnsabuseAI2023}. This tool aims to act as a mitigation measure by analysing domains against criteria indicative of malicious intent, thereby attempting to stop abuse before it even starts. Additionally, the field is witnessing a transformative shift in analysing abuse report evidence through the adoption of Large Language Models (LLMs), such as generative pre-trained transformers (GPTs). These models are highly adept at parsing and understanding complex data patterns that could be missed by human investigators, enhancing the efficiency and automation of DNS abuse mitigation efforts, and forming a more dynamic defence against cyber threats. However, this progress also highlights an emerging challenge: the potential for malicious entities to exploit AI technologies themselves \cite{halvorsenAI2023}.  Consequently, the intersection of AI and machine learning with DNS abuse mitigation not only heralds significant advancements in cybersecurity strategies, but also emphasises the need for vigilance to prevent these technologies from being used for harmful purposes. This pivotal moment in the fight against DNS abuse underscores the need for ongoing innovation and adaptation to effectively secure digital ecosystems.

\subsection{Role of AI \& ML}

The introduction of AI and machine learning technologies into DNS abuse mitigation marks the beginning of an innovative era focused on proactive detection and neutralisation of cyber threats \cite{tariq2023critical}.  This approach facilitates the rapid analysis of large datasets to uncover patterns indicative of malicious intent in DNS queries. For example, machine learning techniques have been highly effective in analysing DNS queries to classify domain names, significantly improving the detection of domains linked to malware \cite{LiMaliciousDomainDetection2020}. Furthermore, the application of neural network models, such as the Extreme Learning Machine (ELM), has achieved accuracy rates above 95\% in the identification of malicious domains, demonstrating the predictive power of AI in combating cyber threats \cite{ZouDNSGraphMining2015}. Additionally, the technique of DNS graph mining has illuminated AI's potential within cybersecurity frameworks, with methodologies like belief propagation algorithms achieving high precision in identifying infected hosts and malicious domains. These examples underscore the vital role of AI and machine learning in supporting DNS abuse, paving new avenues for early detection and swift mitigation of potential abuses. However, the complexity of AI models and the demand for transparency in their decision-making processes present ongoing challenges. Integrating AI into DNS abuse mitigation strategies improves security measures, but also requires careful attention to ethical considerations and the establishment of governance frameworks \cite{AntonakakisMalwareDomainsUpperDNS2011}. AI and machine learning can help improve DNS abuse mitigation, but experts must be clear about the problem.  It is important to understand how AI models make certain decisions. This helps build trust and ensures that people are responsible for them. There are difficulties in making things clear, such as needing to write down what data was used for training, telling others about the things that affect choices, and explaining how models change to face new risks. It is still difficult to find the right balance between the complexity needed for good threat detection and the openness needed for blame.

In summary,AI and ML are valuable in protecting against rapidly evolving cyber threats and a wide range of devices, including those in the IoT. However, their predictive accuracy can be limited by the quality and quantity of data used for training. Sophisticated attacks designed to evade detection algorithms present a notable challenge, underscoring the importance of continuous learning and adaptation in AI/ML models to maintain their effectiveness.


\section{Case Studies and Real-World Applications}

In recent years, technology has become so widespread that we have witnessed an unmatched number and complexity of cyber threats. A significant vulnerability that can be exploited is the DNS domain name system, a critical part of the internet infrastructure that translates human-readable names into IP addresses \cite{kumari2021sac115}. 

\begin{enumerate} 

\item\textbf{ Case Study 1: OilRig DNS Tunneling Attack }

The case of OilRig reflects the use of custom DNS tunnelling protocols for command and control (C2) operations, thus making it dual-use in nature, both in normal operation and on a fallback communication channel \cite{paloaltonetworks2021dnsattacks}.The xHunt campaign \cite{unit42_xhunt_2021} followed a similar trend of including Snugy backdoor implants in targets of Middle Eastern government organisations and keeping track of them using DNS tunnelling for communication with its C2.  These are examples that underscore the strategic use by adversaries of DNS tunneling techniques for stealthiness and resilience within the context of their operations \cite{unit42_2021}.

\captionsetup{font= footnotesize}
\begin{figure}[H]
    \centering
    \includegraphics[width=\textwidth]{background/DNSTuu.png}
    \caption{DNS tunneling communication between the attacker's command and control (C2) infrastructure and the victim's network.}
    \label{fig:figTen}
\end{figure}



\item\textbf{ Case Study 2: SUNBURST Use of DGAs}

SUNBURST backdoor associated with the breach of the SolarWinds supply chain represents a case in which the use of DGAs is critical, if not only, to conceal communications and system details \cite{paloaltonetworks2021dnsattacks}. The SUNBURST backdoor applies the deep use of DNS manipulation for evasion purposes and subsequent attack stages by encoding basic system identifiers and the usage of DGAs for C2 check-ins \cite{unit42_solarstorm_2021}.

\captionsetup{font= footnotesize}
\begin{figure}[H] 
    \centering
    \includegraphics[width=0.8\linewidth]{background/SUNDNS.png}
    \caption{SUNBURST backdoor's utilization of DGAs and its associated components.}
    \label{fig:figEleven}
\end{figure}

\item\textbf{ Case Study 3: Fast Flux Techniques}
The presence of several C2 domains related to the Smoke Loader malware family using Fast Flux techniques only further underscores the difficulties associated with the tracking and eradication of DNS-enabled threats. \cite{paloaltonetworks2021dnsattacks}.The major takeaway in the rapid rotation of IP addresses of this method points to the dynamism of strategies used in malicious communications, thus improving the means of defence by cybersecurity \cite{unit42_fastflux_2021}.

\captionsetup{font= footnotesize}
\begin{figure}[H]
    \centering
    \includegraphics[width=0.8\linewidth]{background/FastFluDNS.png}
    \caption{The usage of Fast Flux techniques by the Smoke Loader malware family for dynamic C2 domain communications.}
    \label{fig:figTweleve}
\end{figure}


\item\textbf{Case Study 4:  Malicious Newly Registered Domains (NRDs)}

Malicious NRDs crafted opportunistically in the context of the pandemic expose how threat actors exploit current events to engineer targeted attacks. \cite{paloaltonetworks2021dnsattacks} From domains that mirror the information resources of COVID-19 to those that feign government relief programmes, the evolution of such attacks reflects a calculated approach to exploiting public interest and vulnerabilities  \cite{unit42_covid19_phishing_2021} .

\captionsetup{font= footnotesize}
\begin{figure}[H]
    \centering
    \includegraphics[width=0.8\linewidth]{background/PandemicTime.png}
    \caption{The usage of Fast Flux techniques by the Smoke Loader malware family for dynamic C2 domain communications.}
    \label{fig:figThirteen}
\end{figure}

\end{enumerate}

In the coronavirus pandemic, too, phishing attacks changed to initially targeting PPE and testing kits, then turning to government stimulus programmes and subsequently enlisting vaccine distribution. Several of them, in fact, employed sophisticated tools, such as MFA pretending as the US Federal Trade Commission and brands such as Pfizer and BioNTech, to steal credentials. where it emphasised that there was a 530\% surge in vaccine-related phishing attempts and a 189\% increase in attacks on pharmacies and hospitals from December last year to February this year. Advice was given to individuals and organisations that includes being cautious in email and website transactions, advancing security awareness training, and adopting multifactor authentication.

Since January 2020, a total of 69,950 COVID-19 related phishing URLs have been received, of which 33,447 are specifically dedicated to COVID-19. The data have been normalised in such a way that the peak of each topic is at 100\%. The results showed much steadier phishing when it came to topics such as pharmaceuticals and virtual meeting platforms (e.g., Zoom) with vaccines and testing showing sharper rises and falls in the attention of scammers.

\captionsetup{font= footnotesize}
\begin{figure}[H]
    \centering
    \includegraphics[width=0.8\linewidth]{background/CovidPhising.png}
    \caption{Development trends in the majority of COVID-19-related phishing content hosting sites during the period from January 2020 to February 2021. Adapted from \cite{Unit42AtricleCovidPhishing2021}.}
    \label{fig:figFourteen}
\end{figure}

It is evident that a large portion of COVID-19 themed phishing pages targeted leading brands for phishing business credentials, such as Microsoft login, Webmail, and Outlook login. For example, about 23\% of these phishing URLs were posed as Microsoft login pages. This threat has particularly highlighted the shift towards remote work in the pandemic, and hence magnified the relevance of these attacks as one of the foremost methods that bad actors are taking on.


\captionsetup{font= footnotesize}
\begin{figure}[H]
    \centering
    \includegraphics[width=0.8\linewidth]{background/TOPCOVIDURLS.png}
    \caption{Top spoofed websites in COVID-themed phishing attacks (global), where the percentage in each column is the percentage of phishing volume per site and category. Adapted from \cite{Unit42AtricleCovidPhishing2021}.}
    \label{fig:figFiveteen}
\end{figure}

Thus, this clearly indicates a situation whereby the attackers set up websites frequently for COVID-19 themed phishing attacks. Many of these phishing pages are found on sites created less than 32 days, meaning that these sites are launched with specific purposes in view of these imminent attacks. The strategy allows attackers to customise their messages and URLs to the current pandemic trends, indicating the dynamism behind such cyber threats.

\captionsetup{font= footnotesize}
\begin{figure}[H]
    \centering
    \includegraphics[width=0.8\linewidth]{background/AgeCovid.png}
    \caption{Statistic of lifespan distribution of COVID-19-related phishing content hosting sites when the sites are reported. Adapted from \cite{Unit42AtricleCovidPhishing2021}.}
    \label{fig:figSixteen}
\end{figure}


\section{Challenges \& Future Directions}

Mitigating DNS abuse demands an immediate stop to the rapid evolution of cyber threats, underscoring the critical need for rapid global cooperation and the implementation of advanced technology. The key challenge is to achieve a fine balance between reducing false positives and accurately identifying genuine threats, while simultaneously advancing beyond the limitations of outdated technologies \cite{pour2023comprehensive}. The future of this domain largely depends on researchers' ability to enhance technological solutions, particularly focussing on the improvement of AI algorithms for deeper analysis of DNS traffic patterns. This opens a promising pathway for the creation and application of locally developed tools, providing innovative strategies to strengthen DNS defences. The ability to navigate the complex landscape of DNS abuse will require stakeholders to be agile in responding to emerging threats and developing novel solutions. The collective push towards the evolution of technology and methodologies will play a pivotal role in shaping effective DNS abuse management strategies in the years ahead.


\subsection{Identification of Current Challenges}

Mitigating DNS abuse involves developing strategies that should not only be proactive, but kept constantly up to date to handle the changing environment of cyber threats. The fluid nature of these threats means updating current protocols as well as developing new defence methods. With bad actors constantly reviewing their methods to take advantage of the vulnerability of DNS, it has become imperative that the cybersecurity industry continuously updates its defence mechanisms \cite{bhattacharya2023dns}. Being a global phenomenon, the Internet and hence DNS abuse being transnational in character, there is no other alternative than international cooperation. The effectiveness of DNS abuse management would be based on collaborative work across national borders, where experts in different geographical areas come together to share their knowledge and resources \cite{altulaihan2022cybersecurity}. The legal and regulatory framework varies in the various jurisdictions, making it difficult to reach a consensus on the regulations, standards, and enforcement actions. Another big challenge is that, to mitigate DNS abuse, the requirement is necessary to eliminate both false positives and negatives. Balance must be established in such a way that rather strict measures may reduce user experience, while, at the same time, being liberal might bring less detection of malicious activities. The cybersecurity community must continue to advance its detection and response capabilities, due to the increasing levels of sophistication used by DNS abusers. This will keep the security and integrity of the DNS system in good shape, thus protecting this vital part of the Internet infrastructure.

\subsection{Discussion on Future Research Directions and Technologies}

At the ICANN77 meeting, developments on mitigating DNS abuse were presented. These included the draughting of changes mandating that registrars and registries respond to abuse notifications, which contributed to a decline in global abuse levels after Freenom's legal response. Although the ccNSO Domain Abuse Steering Committee argued for a proactive mitigation strategy, the gNSO's analysis found minimal abuse rates in EU ccTLDs, which it attributed to market maturity and non-profit models \cite{VanRoste2023}. To improve domain security and enable international cooperation against changing cyberthreats, future initiatives will focus on creating cutting-edge tools and using technologies such as artificial intelligence and machine learning. This means that we need to look at more complex AI and machine learning tools that can understand the details of web traffic, which will make the results more accurate and stop the sending of wrong signals \cite{ISG2023}.

\section{Summary of Findings}

Research on the transparency of DNS abuse mitigation emphasises how threats are always changing and how mitigation techniques must evolve as well. It highlights how important community involvement and transparency are to fostering trust. Although technological advancements, especially in AI and machine learning, are essential for threat detection, their implementation must be carefully considered. Practical examples provide insight into the efficacy of various strategies. Maintaining a balance between new mitigation measures and effective teamwork and communication is a constant issue. To effectively address DNS abuse, future efforts should focus on using technology, international cooperation, and standardised information exchange.






















