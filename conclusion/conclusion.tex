\chapter{Conclusion}

\section{Brief Review}

This project looked at DNS abuse, a situation in which malicious actors exploit domain names for malicious activities such as phishing. DNS infrastructure providers, including registrars and registries, have been the focal points of attention due to their significant roles in controlling and potentially reducing this abuse. The investigation included the complaints these providers receive and the measures they take to mitigate abuse, such as the deletion or blocking of domain name registrations. A critical aspect of this research was the concept of transparency: the extent to which these actions are disclosed and documented by the providers. It was found that the practice of issuing comprehensive transparency reports is not as widespread as necessary, despite the crucial role that openness plays in fostering trust and accountability in the digital realm. 


\section{Main Results}

\subsection{Related back to Project Objectives:}

The project led to the identification of serious transparency gaps in the mitigation of DNS abuse adopted by infrastructure providers. Although some reporting and communication measures of mitigation action against DNS abuse have been adopted to trace the information on the mitigation measures undertaken, it is largely inconsistent. This goes hand in hand with our first objective, to better understand practices on transparency in the area and set a clear requirement for standardised transparency measures.

\subsection{Summary of Proposals:}

Throughout the research project, several strategies have been discovered to improve transparency:


\begin{enumerate}

    \item Regular Transparency Reporting: Urge all DNS infrastructure providers to release reports on a regular basis describing the steps they have taken to mitigate DNS abuse.

    \item Stakeholder Engagement: Encourage more cooperation and communication on transparency regulations between DNS providers, users, and legislators.

    \item Public Accountability Mechanisms: Provide and implement systems that allow the general public to monitor and evaluate DNS abuse mitigation efforts.
    
    \item Innovation in Defence Strategies and Sharing: Emphasise the importance of preparing new methods to combat DNS abuse by developing and encouraging the sharing of these innovative strategies among stakeholders.
    
    \item Transparency in Monitoring and Collective Action: Encourage open observation of DNS activity and cooperative efforts from all parties within the DNS ecosystem to ensure a unified strategy to mitigate abuse.
    
\end{enumerate}

These strategies seek to expand on the fundamental measures outlined in this research project, focusing not only on the individual efforts of DNS infrastructure providers but also on the collective efforts and shared responsibilities throughout the DNS ecosystem. By using these tactics, the Internet community can work towards improving trust and transparency, which will result in a DNS system that is more secure and resistant to abuse.


\subsection{Assessment of Contribution :}

These results can be used to show the importance of the application, as the responsibility may fall on the DNS infrastructure providers to have a better and more transparent practice in mitigating DNS abuses. Eventually, the adopted strategies give a chance for better accountability and greater trust from the interested users and parties. In the larger context of cybersecurity and Internet governance, where transparency is becoming more widely acknowledged as essential to promoting a safer online environment, this contribution is vital.


\section{Future Work}



\subsection{Further Research Directions:} Future research should examine how AI and machine learning are combined to detect predictive DNS abuse and how well international regulatory frameworks enforce transparency requirements. In addition, further research may examine how user trust and behaviour are affected by transparency, as well as how various degrees of openness influence how DNS infrastructure providers are seen. Furthermore, studies could assess how well different transparency techniques mitigate DNS abuse in the real world.

\subsection{Practical Next Steps for Developing Transparency Best Practices: }

\begin{enumerate}
    \item Framework Development: Collaborate with leading industry players to develop a uniform transparency framework that DNS infrastructure providers can use anywhere.

    \item Technology Solutions: Looking into technical options that automate the gathering and sharing of DNS abuse data to improve transparency.

    \item Policy Recommendations: Draughting policy suggestions should require transparency in these activities to encourage legislative support for DNS abuse mitigation initiatives.
    
    \item Stakeholder Collaboration: Governing bodies that maintain the DNS infrastructure, regulatory organisations, and cybersecurity communities need to join hands to combat and find an amicable solution to these challenges.
    
    \item Transparency Standardisation: Standardise transparency reports across the industry to ensure uniformity in disclosing DNS abuse mitigation efforts.
    \item Real-Time Monitoring: Implement real-time abuse monitoring dashboards to enable swift detection and response to DNS threats.
    \item  Public Awareness: Promote user education on DNS security to enhance public awareness and safeguard against potential abuses.
\end{enumerate}


\subsection{Enhanced Transparency Practices for DNS Abuse Mitigation: }

Building on these initial steps, registries and registrars are urged to implement improved transparency measures such as the following to strengthen the DNS ecosystem's resistance to abuse:

\begin{enumerate}
    \item Public Reporting: Establish detailed and consistent transparency reports that provide information on the number of DNS abuse reports received, the steps taken, and the results of those steps. In addition to increasing user trust, this transparency makes the organisation responsible for efficient abuse mitigation.

    \item Stakeholder Engagement: Provide forums or advisory committees to discuss and evaluate mitigation solutions for DNS abuse that involve a wide range of stakeholders, such as government representatives, cybersecurity professionals, and members of civil society. This guarantees that decision-making procedures take into account a wide range of points of view.

    \item Abuse Point of Contact: Clearly identify the abusive contact and make it public. This makes it easier for the community, including end users and cybersecurity researchers, to report and handle abuse problems effectively.

    \item Best Practice Sharing: Encourage a transparency environment by sharing best practices, resources, and innovations to mitigate DNS abuse with colleagues in the DNS ecosystem. Workshops for the entire industry or collaborative platforms can help to promote this conversation.

    \item User Education: Recreate and distribute instructional materials to help domain owners and end users identify and stop DNS misuse. Empowering people with knowledge can drastically reduce the effectiveness of phishing and other abusive techniques.

    \item Automated Abuse Detection: Make use of AI and machine learning technology to automatically identify possible DNS abuse behaviours. Exchange anonymous indicators of compromise (IoCs) with reliable partners to increase the resilience of the ecosystem as a whole.

\end{enumerate}


\subsection{Future Directions in DNS Abuse Mitigation: }

Future studies and practical initiatives should focus on the following areas to better address the dynamic nature of DNS abuse and proactively counter new threats:

\begin{enumerate}
    \item Emerging Technologies: Exploring the possibility of DNS abuse and creating focused mitigation solutions in AI-generated content and the growth of IoT devices.

    \item AI and Machine Learning for Proactive Defence: Using data analysis to find possible abuse vectors and advancing AI and machine learning models to anticipate and handle DNS abuse before it happens.

    \item Enhanced IoT Security: Establishing security guidelines for IoT device makers to stop device exploitation in DNS abuse, encouraging industry-wide adoption through partnerships and laws.

    \item Global Policy and Regulation Dialogue: Participating in policy discussions to coordinate mitigation measures for DNS abuse and promote laws that promote security, privacy, and openness.

    \item Transparency Evolution: Advancing transparency standards in line with technology, emphasising real-time data sharing, blockchain-based log reporting, and user-friendly interfaces to prevent unauthorised access to data.

    
\end{enumerate}


\subsection{Contributions to Future Transparency Practices:}

This research contributes to the ongoing development of best practices for transparency in DNS abuse mitigation. By emphasising the significance of transparent, consistent reporting and promoting stakeholder interaction, it seeks to enable better informed policy-making and promote a safer online environment. Identifying existing challenges and proposing feasible solutions, this study is a first step toward improving openness in the DNS ecosystem. The desired results are a reduction in DNS abuse, an informed and active user base, and the development of a more reliable Internet ecosystem.


\section{Reflection}
\subsection{Personal Learning: } 

With this project, I have appreciably learnt the complexities of DNS abuse and how these complexities could pose a challenge in conceiving systems that address the potential lack of transparency. In that sense, DNS abuse is said to be very dynamic, always changing to expose new tactics by bad actors; mitigation strategies should, therefore, become adaptive.  I learnt that transparency isn't just about sharing information; it is about building trust within the community, improving the effectiveness of abuse mitigation efforts, and impacting broader internet governance and security positively.

\subsection{Evaluation of Research Process:  }

In other words, this research project has brought to light the complicated hurdles in studying DNS abuse mitigation transparency, from the reluctance to share sensitive information due to privacy and security concerns to possible biases in data self-reporting. The process made clear the delicate balance that must be struck between maintaining security and releasing just enough information to be transparent. Although the technique allowed a thorough analysis to be performed, it also highlighted areas that needed to be improved, such as determining more accurate ways to assess how transparency practices affect the reduction of DNS abuse.



\subsection{Perspective on Research Findings and Contributions: }

This research project offers a more comprehensive view of current procedures and their effectiveness, contributing to ongoing conversations on the mitigation and transparency of DNS abuse. Through gap analysis and practical strategy recommendations, the study highlights the necessity of a coordinated approach to openness. It calls for the creation of guidelines and best practices that improve cooperation between all parties involved in the DNS ecosystem. While great progress has been made, my work emphasises the ongoing need for attention and effort in this area and suggests that the road towards a more transparent, safe, and abuse-resistant DNS landscape is far from over.


In summary, this project has improved my knowledge of DNS abuse mitigation and the pursuit of transparency, while also providing insightful information that will guide future research in this area of cybersecurity and Internet governance.


