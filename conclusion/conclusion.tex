\chapter{Conclusion}

\section{Brief Review}

This project has looked at the abuse of the DNS, in which a situation leads to malicious actors using domain names for their malicious activities, such as phishing. DNS infrastructure providers, including registries and registrars, have thus focused on playing a role in controlling, if not reducing, this abuse. The research included complaints received from providers and steps taken to abuse by deleting or blocking name registrations. Therefore, the core of the study lies in the concept of transparency and the degree to which providers disclose and document these actions. Obviously, the issuing of comprehensive transparency reports does not go as far as the requirement, and it is a very key aspect towards promoting trust and accountability in the digital world. 


\section{Main Results}

\subsection{Related back to Project Objectives:}

This project led to the serious transparency gaps in the mitigation of DNS abuse adopted by infrastructure providers. Mitigation action against DNS abuse includes technical measures, such as the traceability of responsible individuals and strict reporting and communication of mitigation action. This goes parallel with our first objective: to understand how the area practices on transparency and, when necessary, establish a requirement for standardised measures.

\subsection{Summary of Proposals:}

Throughout the research project, several strategies have been discovered to improve transparency:


\begin{enumerate}

    \item Regular Transparency Reporting: Ask DNS infrastructure providers to provide information to the public regularly about the actions they have taken in their fight to mitigate against DNS abuse.

    \item Stakeholder Engagement: Increase cooperation and communication to regulate transparency between DNS providers, users, and legislators. 
.
    \item Public Accountability Mechanisms: Members of the general public should have ways through which they can monitor and evaluate DNS efforts to mitigate abuse. 
    
    \item Innovation in Defence Strategies and Sharing: Create a framework that indicates the development of strategies to enable one to be ready and share the strategies among the participants to enhance innovation in their efforts in the fight to end different abuses.
    
    \item Transparency in Monitoring and Collective Action: Open observation of DNS activity and the collective participation of all parties in the DNS ecosystem to work towards a single approach to end the abuses.
    
\end{enumerate}

These strategies try to supplement the fundamental measures of the research project already listed by also taking into account the providers of the efforts of the DNS infrastructure, as well as the efforts and shared responsibilities over the DNS ecosystem. The Internet community can achieve a much more reliable system by enhancing the level of trust and transparency, thus enforcing a DNS system that protects against abuse more effectively.

\section{Future Work}

\subsection{Further Research Directions:} Future research needs to work towards integrating AI and machine learning in predictive DNS abuse detection, as well as the extent to which international regulatory frameworks manage to establish transparency requirements. This could be further researched into how the difference in degrees of openness and transparency changes user trust and behaviour, in addition to transparency, and the perception of infrastructure DNS providers. Furthermore, studies could assess how well different transparency techniques mitigate DNS abuse in the real world.

\subsection{Practical Next Steps for Developing Transparency Best Practices: }

\begin{enumerate}
    \item Framework Development: Develop a common transparency framework that could be adopted by DNS infrastructure providers around the world and work closely with the respective leading industry partners on this matter.

    \item Technology Solutions: Explore technical mechanisms for automating the transparent collection and dissemination of DNS abuse data.

    \item Policy recommendations: The requirement of transparency in those activities should be a prerequisite for the formulation of policies and incentives for legislative support for DNS abuse mitigation efforts.
    
    \item Stakeholder Collaboration: Regulatory bodies responsible for maintaining DNS infrastructure and cybersecurity communities unite their efforts to fight and find a solution to these issues.
    
    \item Transparency standardisation: Industry-wide standardisation of the reporting process would go a long way to maintaining a healthy level of uniformity in the matters of disclosure regarding behavioural efforts when addressing DNS abuse.
    
    \item Real-Time Monitoring: Ensures timely detection and response to threats and utilisation of real-time dashboards
    
    \item  Public Awareness: User education on DNS security can boost the general public’s knowledge of the subjects, thereby protecting them from potential abuses.
\end{enumerate}


\subsection{Enhanced Transparency Practices for DNS Abuse Mitigation: }

Building on these initial steps, registries and registrars are urged to implement improved transparency measures such as the following to strengthen the DNS ecosystem's resistance to abuse:

\begin{enumerate}
    \item Public Reporting: Issue a comprehensive and consistent transparency report that includes the number of abuse reports received, the actions they recommend, and the follow-up action on them. Moreover, transparency builds trust among users and holds the organisation accountable for efficient abuse mitigation.

    \item Stakeholder Engagement: Create forums or advisory boards to discuss and assess DNS solutions to mitigate abuse. This should involve a broader selection of stakeholders, including government officials, cyber security specialists, and representatives of civil society. This can help guide the decision-making process with input from all sides.

    \item Best Practice Sharing: Encourage a transparent environment by sharing best practices, resources, and innovations to mitigate DNS abuse with colleagues in the DNS ecosystem. 

    \item User Education: Recreate and share educational materials to assist domain owners and end users in recognising and avoiding DNS abuse. When people are well informed, they become intelligent for most abuses.

    \item Automated Abuse Detection: Make use of AI and machine learning technology to automatically identify possible DNS abuse behaviours. Exchange anonymous indicators of compromise (IoCs) with reliable partners to increase the resilience of the ecosystem as a whole.

\end{enumerate}


\subsection{Future Directions in DNS Abuse Mitigation: }

Future studies and practical initiatives should focus on the following areas to better address the dynamic nature of DNS abuse and proactively counter new threats:

\begin{enumerate}
    \item Emerging Technologies: Research DNS, the abuse that may arise and build a solution-targeted mitigation around AI-generated content and the growth in IoT devices.

    \item AI and Machine Learning for Proactive Defence: Data analysis to identify possible abuse trends and support AI and machine learning solutions that can anticipate and address DNS abuse before it occurs.

    \item Enhanced IoT Security: advocates for the spread of security measures through partnerships and laws for manufacturers of IoT devices to hopefully change the fact that IoT devices have a homogeneous trigger and seek a tendency from which an abuser can attack a device.

    \item Global Policy and Regulation Dialogue: Engage in policy discussion and implementing Database Coordination and laws that can bring growth and satisfy cyber infrastructure security, privacy, and freedom.

    \item Transparency Evolution: Evolution of transparency in line with evolving tech, consider setting up real-time data sharing, blockchain log reporting, and user-friendly setups to deter unauthorised access to data.

    
\end{enumerate}


\subsection{Contributions to Future Transparency Practices:}

This research further contributes to the ongoing development of best-practice for transparency in reporting on DNS abuse mitigation. This emphasises the importance of transparent and consistent reporting and interaction with stakeholders so that policy makers can rely on reporting for informed policy formulation and to be able to contribute to safer cyberspace. Therefore, this has been one of the main findings of this study in the remaining challenges of the field. The ways out include informed and active user bases, reduction of DNS abuse, and even more reliable Internet ecosystems.


\section{Reflection}
\subsection{Personal Learning: } 

This project really opened my eyes and taught me a lot about the complexity of DNS abuse and how this could pose a challenge in the development of systems due to the potential lack of transparency. DNS abuse can be viewed as dynamic, in the sense that it continues to evolve to show new tactics used by the bad actors, so that the mitigation strategies also need to be adaptive. And what I have learnt is that transparency in work is not only sharing information, but, in fact, it helps build trust within the community, increases efficiency of efforts aimed at abuse mitigation, and has an overall positive influence on governance of the Internet and security.

\subsection{Evaluation of Research Process:  }

In other words, this research project exposed serious challenges to the study of transparency in mitigating DNS abuse, from secrecy in sharing sensitive data, dreaded by privacy and security considerations, to the threat of data self-reporting biases. What it made clear was the fact that the process made it clear that a delicate balance needs to be achieved between the need for security and releasing just enough information to be, basically, transparent. However, these approaches allowed in-depth research to be carried out, and at the same time, they criticised the general area of what needs to be improved, including methods to identify more accurate ways by which transparency practices influence the reduction of DNS abuse.



\subsection{Perspective on Research Findings and Contributions: }

This research project provides a broader look at current practices and how effective they are within the ongoing discussions on how to mitigate and possibly even yield more effective transparency of DNS abuse. This research suggests that there should be an organised approach to openness from gap analysis and practical strategy recommendations. This calls for designing best-practice guidelines that strengthen cooperation among all the parties in the DNS ecosystem. Although very good progress is being made, my work highlights the continued need for attention and effort in this area, suggesting that the road toward a more transparent, safe, and abuse-resistant DNS landscape is very far from complete.


This project has broadly enriched my knowledge in relation to mitigation of DNS abuse and transparency pursued, while provoking very important information that should take this field into account in future research in the area of cybersecurity and Internet governance.




