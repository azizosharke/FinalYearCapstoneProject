\chapter{Conclusion}

\section{Brief Review}

This project has looked at the abuse of the DNS, in which a situation leads to malicious actors using domain names for their malicious activities, such as phishing. DNS infrastructure providers, including registries and registrars, have thus focused on playing a role in controlling, if not reducing, this abuse. The research included complaints received from providers and steps taken to abuse by deleting or blocking name registrations. Therefore, the core of the study lies in the concept of transparency and the degree to which providers disclose and document these actions. Obviously, the issuing of comprehensive transparency reports does not go as far as the requirement, and it is a very key aspect towards promoting trust and accountability in the digital world. 


\section{Main Results}

\subsection{Related back to Project Objectives:}

This project led to the serious transparency gaps in the mitigation of DNS abuse adopted by infrastructure providers. Mitigation action against DNS abuse includes technical measures, such as the traceability of responsible individuals and strict reporting and communication of mitigation action. This goes parallel with our first objective: to understand how the area practices on transparency and, when necessary, establish a requirement for standardised measures.

\subsection{Summary of Proposals:}

Throughout the research project, several strategies have been discovered to improve transparency:


\begin{enumerate}

    \item Regular Transparency Reporting: Request the providers of DNS infrastructure to report on a regular basis with respect to actions undertaken in the mitigation of DNS abuse.

    \item Stakeholder Engagement: Increase cooperation and communication toward the regulation of transparency between DNS providers, users, and legislators.
.
    \item Public Accountability Mechanisms: Develop a means for members of the general public to monitor and evaluate DNS abuse mitigation efforts.
    
    \item Innovation in Defence Strategies and Sharing: Highlight the need to develop strategies to prepare for the fight against DNS abuse and share these strategies among participants to further promote innovation in their approaches to the fight against different types of abuses.
    
    \item Transparency in Monitoring and Collective Action: Open observation of DNS activity and the collaborative work of all parties involved in the DNS ecosystem to power a single strategy to stop abuse.
    
\end{enumerate}

These strategies seek to add to the basic measures outlined in this research project, considering the individual efforts of the providers of DNS infrastructure, but also the collective efforts and shared responsibilities throughout the DNS ecosystem. With these measures, definitely the internet community can help improve the level of trust and transparency, thus ensuring that a much more effective DNS system is in place securing against abuse.

\section{Future Work}

\subsection{Further Research Directions:} Future research needs to work towards integrating AI and machine learning in predictive DNS abuse detection, as well as the extent to which international regulatory frameworks manage to establish transparency requirements. This could be further researched into how the difference in degrees of openness and transparency changes user trust and behaviour, in addition to transparency, and the perception of infrastructure DNS providers. Furthermore, studies could assess how well different transparency techniques mitigate DNS abuse in the real world.

\subsection{Practical Next Steps for Developing Transparency Best Practices: }

\begin{enumerate}
    \item Framework Development: The development of a common transparency framework that can be applied by DNS infrastructure providers across the globe and work with the leading industry partners in this respect.

    \item Technology Solutions: Investigate technical possibilities for the automation of DNS abuse data collection and sharing in a more transparent manner.

    \item Policy Recommendations: Transparency in such activities should be a requirement for draughting policies and should encourage legislative support for DNS abuse mitigation initiatives.
    
    \item Stakeholder Collaboration: The regulatory bodies, which maintain the DNS infrastructure, and the cybersecurity communities come together to fight the battle and find an amicable solution to these challenges.
    
    \item Transparency Standardisation: Uniformity in the standardisation of reporting across the industry would help maintain uniformity in the level of disclosure of mitigation efforts for DNS abuse.
    
    \item Real-Time Monitoring: This ensures abuse is detected in time and threats are responded to in time, and real-time dashboards are used.
    
    \item  Public Awareness: User education on DNS security to improve public awareness and protect them from possible abuses.
\end{enumerate}


\subsection{Enhanced Transparency Practices for DNS Abuse Mitigation: }

Building on these initial steps, registries and registrars are urged to implement improved transparency measures such as the following to strengthen the DNS ecosystem's resistance to abuse:

\begin{enumerate}
    \item Public Reporting: Create a detailed and consistent transparency report in which we issue the number of abuse reports received, the steps to be taken, and the result of those steps. In addition, this transparency increases the confidence of users and makes the organisation responsible for efficient mitigation of abuse.

    \item Stakeholder Engagement: Provide forums or advisory committees to discuss and evaluate mitigation solutions for DNS abuse that involve a wide range of stakeholders, such as government representatives, cybersecurity professionals, and members of civil society. This guarantees that decision-making procedures take into account a wide range of points of view.

    \item Best Practice Sharing: Encourage a transparency environment by sharing best practices, resources, and innovations to mitigate DNS abuse with colleagues in the DNS ecosystem. 

    \item User Education: Recreate and distribute instructional materials to help domain owners and end users identify and mitigate DNSabuse. Empowering people with knowledge can drastically reduce the effectiveness of phishing and other abusive techniques.

    \item Automated Abuse Detection: Make use of AI and machine learning technology to automatically identify possible DNS abuse behaviours. Exchange anonymous indicators of compromise (IoCs) with reliable partners to increase the resilience of the ecosystem as a whole.

\end{enumerate}


\subsection{Future Directions in DNS Abuse Mitigation: }

Future studies and practical initiatives should focus on the following areas to better address the dynamic nature of DNS abuse and proactively counter new threats:

\begin{enumerate}
    \item Emerging Technologies: Research DNS, the possibility of abuse, and develop a targeted mitigation solution within AI-generated content and growth in IoT devices.

    \item AI and Machine Learning for Proactive Defence: Using data analysis to find possible abuse vectors and advancing AI and machine learning models to anticipate and handle DNS abuse before it happens.

    \item Enhanced IoT Security: Establishing security standards with the support of partnerships and laws for manufacturers of IoT devices to reduce the tendency to exploit the device in DNS abuse.

    \item Global Policy and Regulation Dialogue: Participate in discussions that deal with policies, coordinating mitigation for DNS abuse, and laws that can enhance security, privacy, and openness.

    \item Transparency Evolution: Advancing transparency standards in line with technology, emphasising real-time data sharing, blockchain-based log reporting, and user-friendly interfaces to prevent unauthorised access to data.

    
\end{enumerate}


\subsection{Contributions to Future Transparency Practices:}

This research further contributes to the ongoing development of best-practice for transparency in reporting on DNS abuse mitigation. This emphasises the importance of transparent and consistent reporting and interaction with stakeholders so that policy makers can rely on reporting for informed policy formulation and to be able to contribute to safer cyberspace. Therefore, this has been one of the main findings of this study in the remaining challenges of the field. The ways out include informed and active user bases, reduction of DNS abuse, and even more reliable Internet ecosystems.


\section{Reflection}
\subsection{Personal Learning: } 

This project really opened my eyes and taught me a lot about the complexity of DNS abuse and how this could pose a challenge in the development of systems due to the potential lack of transparency. DNS abuse can be viewed as dynamic, in the sense that it continues to evolve to show new tactics used by the bad actors, so that the mitigation strategies also need to be adaptive. And what I have learnt is that transparency in work is not only sharing information, but, in fact, it helps build trust within the community, increases efficiency of efforts aimed at abuse mitigation, and has an overall positive influence on governance of the Internet and security.

\subsection{Evaluation of Research Process:  }

In other words, this research project exposed serious challenges to the study of transparency in mitigating DNS abuse, from secrecy in sharing sensitive data, dreaded by privacy and security considerations, to the threat of data self-reporting biases. What it made clear was the fact that the process made it clear that a delicate balance needs to be achieved between the need for security and releasing just enough information to be, basically, transparent. However, these approaches allowed in-depth research to be carried out, and at the same time, they criticised the general area of what needs to be improved, including methods to identify more accurate ways by which transparency practices influence the reduction of DNS abuse.



\subsection{Perspective on Research Findings and Contributions: }

This research project provides a broader look at current practices and how effective they are within the ongoing discussions on how to mitigate and possibly even yield more effective transparency of DNS abuse. This research suggests that there should be an organised approach to openness from gap analysis and practical strategy recommendations. This calls for designing best-practice guidelines that strengthen cooperation among all the parties in the DNS ecosystem. Although very good progress is being made, my work highlights the continued need for attention and effort in this area, suggesting that the road toward a more transparent, safe, and abuse-resistant DNS landscape is very far from complete.


This project has broadly enriched my knowledge in relation to mitigation of DNS abuse and transparency pursued, while provoking very important information that should take this field into account in future research in the area of cybersecurity and Internet governance.




