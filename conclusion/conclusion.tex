\chapter{Conclusion}

\section{Brief Review}

This project looked into DNS abuse, a situation in which malicious actors exploit domain names for malicious activities such as phishing. DNS infrastructure providers, including registrars and registries, have been focal points of attention due to their significant roles in controlling and potentially reducing this abuse. The investigation included the complaints these providers receive and the measures they undertake to stop abuse, such as the deletion or blocking of domain name registrations. A critical aspect of this research was the concept of transparency: the extent to which these actions are disclosed and documented by the providers. It was found that the practice of issuing comprehensive transparency reports is not as widespread as necessary, despite the crucial role that openness plays in fostering trust and accountability in the digital realm. 


\section{Main Results}

\subsection{Related back to Project Objectives:}

The project led to the identification of serious transparency gaps in the mitigation of DNS abuse adopted by infrastructure providers. Although some reporting and communication measures of mitigation action against DNS abuse have been embraced to trace the information on the mitigation measures undertaken, it is largely inconsistent. This goes hand in hand with our first objective, to further understand practices on transparency in the area and set a clear requirement for standardised transparency measures.

\subsection{Summary of Proposals:}

Throughout the research project, a number of strategies for improving transparency have been discovered:


\begin{enumerate}

    \item Regular Transparency Reporting: Urge all DNS infrastructure providers to release reports on a regular basis describing the steps they have taken to mitigate DNS abuse.

    \item Stakeholder Engagement: Encourage more cooperation and communication on transparency regulations between DNS providers, users, and legislators.

    \item Public Accountability Mechanisms: Provide and implement systems that let the general public monitor and evaluate DNS abuse mitigation efforts.
    
    \item Innovation in Defence Strategies and Sharing: Emphasise the importance of preparing new methods of fighting against DNS abuse by developing and encouraging the sharing of these innovative strategies among stakeholders.
    
    \item Transparency in Monitoring and Collective Action: Encourage the open observation of DNS activity and cooperative efforts from all parties within the DNS ecosystem to guarantee a unified strategy for mitigating abuse.
    
\end{enumerate}

These strategies seek to expand upon the fundamental measures outlined in this, emphasising not just the individual efforts of DNS infrastructure providers but also the collective efforts and shared responsibilities throughout the DNS ecosystem. By using these tactics, the internet community can work towards improving trust and transparency, which will result in a DNS system that is more secure and resistant to abuse.


\subsection{Assessment of Contribution :}

These results can be used to show the importance of the application, as the responsibility may fall on the DNS infrastructure providers to have a better and more transparent practice in mitigating DNS abuses. Eventually, the adopted strategies give a chance for better accountability and more trust from the interested users and parties. In the larger context of cybersecurity and internet governance, where transparency is becoming more widely acknowledged as essential to promoting a safer online environment, this contribution is vital.


\section{Future Work}



\subsection{Further Research Directions:} Future research should examine how AI and machine learning are combined to detect predictive DNS abuse and how well international regulatory frameworks enforce transparency requirements. In addition further research may examine how user trust and behaviour are affected by transparency, as well as how various degrees of openness influence how DNS infrastructure providers are seen. Additionally, studies could assess how well different transparency techniques mitigate DNS abuse in the real world.

\subsection{Practical Next Steps for Developing Transparency Best Practices: }

\begin{enumerate}
    \item Framework Development: Collaborate with leading industry players to develop a uniform transparency framework that DNS infrastructure providers can use anywhere.

    \item Technology Solutions: Looking into technical options that automate the gathering and sharing of DNS abuse data to improve transparency.

    \item Policy Recommendations: Drafting policy suggestions should require transparency in these activities in order to encourage legislative support for DNS abuse mitigation initiatives.
    
    \item Stakeholder Collaboration: Governing bodies maintaining DNS infrastructure, regulatory organisations, and cybersecurity communities need to join hands in combating and finding an amicable solution for the above challenges.
    
    \item Transparency Standardisation:  Standardise transparency reports across the industry to ensure uniformity in disclosing DNS abuse mitigation efforts.
    \item Real-Time Monitoring: Implement real-time abuse monitoring dashboards to enable swift detection and response to DNS threats.
    \item  Public Awareness: Promote user education on DNS security to enhance public awareness and safeguard against potential abuses.
\end{enumerate}

\subsection{Contributions to Future Transparency Practices:}

This research contributes to the ongoing development of best practices for transparency in DNS abuse mitigation.By emphasising the significance of transparent, consistent reporting and promoting stakeholder interaction, it seeks to enable better informed policy-making and promote a safer online environment. Identifying existing challenges and proposes feasible solutions, this study is a first step towards improving openness in the DNS ecosystem. A reduction in DNS abuse, an informed and active user base, and the development of a more reliable internet ecosystem are the desired results.


\section{Reflection}
\subsection{Personal Learning: } 

With this project, I have appreciably learned the complexities of DNS abuse  and how these complexities might pose a challenge in coming up with systems that address the potential lack of transparency. In that sense, DNS abuse is said to be very dynamic, always changing to expose new tactics by bad actors; mitigation strategies should, therefore, become adaptive.  I learned that transparency isn't just about sharing information; it's about building trust within the community, improving the effectiveness of abuse mitigation efforts, and impacting broader internet governance and security positively.

\subsection{Evaluation of Research Process:  }

In other words, this research project has brought to light the complicated hurdles in studying DNS abuse mitigation transparency, from the reluctance to share sensitive information due to privacy and security concerns, to possible biases in data self-reporting. The process made clear the delicate balance that must be struck between maintaining security and disclosing just enough information to be transparent. Although the technique made it possible to conduct a thorough analysis, it also pointed up areas that needed to be improved, such as determining more accurate ways to assess how transparency practices affect the reduction of DNS abuse.



\subsection{Perspective on Research Findings and Contributions: }

This research project offers a more comprehensive view of current procedures and their effectiveness, contributing to continuing conversations on DNS abuse mitigation and transparency. Through gap analysis and practical strategy recommendations, the study highlights the necessity of a coordinated approach to openness. It makes the case for the creation of guidelines and best practices that improve cooperation between all parties involved in the DNS ecosystem. While great progress has been made, my work emphasises the ongoing need for attention and effort in this area and suggests that the road towards a more transparent, safe, and abuse-resistant DNS landscape is far from over.


In summary, this project has improved my knowledge of DNS abuse mitigation and the pursuit of transparency while also offering insightful information that will guide future research in this  area of cybersecurity and internet governance.


