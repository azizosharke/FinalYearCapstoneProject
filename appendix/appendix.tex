\chapter{Appendix}


The Domain Name System (DNS) plays a role in the infrastructure of the Internet by converting user domain names into IP addresses. However, due to its use and importance it has become a target for actors seeking to exploit it. These abuses range from setting up phishing websites to taking advantage of DNS for activities such as typosquatting. The responsibility for mitigating abuse primarily lies with DNS infrastructure providers, such as registrars and registries. These entities respond to reports of abuse by taking down confirmed domain names or proactively blocking the registration of harmful ones. While these actions are essential for maintaining the security and integrity of DNS, they also raise questions about how transparent these measures are. 

Transparency in the context of mitigating DNS abuse refers to the disclosure of actions taken by registries and registrars including the criteria and reasoning behind their decisions. Currently, there is prevalence in publishing transparency reports related to this matter leading to a lack of clarity and understanding about the processes involved in combating DNS abuse. This project aims to address this issue through a survey involving registries, registrars and other stakeholders actively engaged in mitigating DNS abuse.

The main objective of the survey is to collect organise and describe the transparency reports they are presently accessible. This will help us gain an understanding of the status of transparency, in mitigating DNS abuse.

\section{Appendix numbering}
Appendices are numbered sequentially, A1, A2, A3\ldots The sections, figures and tables within appendices are numbered in the same way as in the main text. For example, the first figure in Appendix A1 would be Figure A1.1. Equations continue the numbering from the main text.
